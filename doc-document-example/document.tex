\documentclass{template/socthesis}

\usepackage{subcaption}
\usepackage{amsmath}
\usepackage{enumitem}


\addbibresource{text.bib}

\titlecz{Coopmaster - systém pro kontrolu a automatizaci kurníku}
\titleen{Coopmaster - system for coop control and automation}
\author{Jaroslav Němec}
\field{18}
\school{Střední škola a vyšší odborná škola aplikované kybernetiky}
\mentor{Ing. David Podzimek}
\mentorstatement{Ing. Davida Podzimka}
\newcommand{\mentorthanks}{Ing. Davidu Podzimkovi}

% Změňte, pokud se liší
%\region{Jihomoravský}
%\placefooter{Brno 2017}

\begin{document}

    \maketitle

    \makecopyrightstatement{V~Hradci Králové}

    \makethanks{Děkuji svému školiteli \mentorthanks{} za obětavou pomoc, podnětné připomínky a nekonečnou trpělivost, kterou mi během práce poskytoval.}

    \pagestyle{empty}

    \section*{Anotace}
    Ve své práci jsem se zabýval návrhem a realizací kompletního řešení pro malé i střední zemědělce a hospodáře, které má pomoci plnit běžné činnosti. Cílem této práce bylo vytvořit a realizovat systém pro kontrolu a automatizaci chovu hospodářských zvířat.

    \subsection*{Klíčová slova}
    programování, automatizace, zemědělství, neuronové sítě, python, mikroservisní architektura

    \vspace{20mm}

    \section*{Annotation}
    In my work, I have been involved in the design and implementation of a complete solution for small and medium farmers and homesteaders to help them carry out their day-to-day activities. The aim of this work was to design and implement a system for control and automation of livestock farming.

    \subsection*{Keywords}
    programming, automation, agriculture, neural networks, python, microservice architecture

    \pagestyle{plain}

    \tableofcontents % vysází obsah

    %%% Začátek práce
    \setcounter{figure}{0}
    \setcounter{table}{0}
%    \newpage

    %%% Úvod
    \chapter{Úvod}
V dnešní době je populární využívat moderní technologie a umělou inteligenci v různých aplikacích jako například monitoring pohybu zákazníků v obchodech nebo ve formě textových modelů čímž je například známe ChatGPT. Zároveň dnes ubývá lidí, kteří by se chtěli zabývat staráním se o hospodářská zvýřata a tudíž je potřeba, aby něco zaujmulo jejich místo. Zároveň jsem sám člověk, který rád zkoumá nové věci v oblasti informatiky a velice ho baví automatizování nejrůznějších úloh. Na základě těchto faktů jsem se rozhodl vytvořit téma práci popisující využití moderních technologií při chovu hospodářských zvýřat. Nápad na tuto práci vznikl díky mojí babičce, která chová doma slepice. Když odjede na dovolenou, stávám se já tím, kdo se o ně musí starat. Slepice je třeba dojet ráno a večer zkontrolovat a spočítat. Tato činnost je časově náročná kvůli cestování a kolikrát i zbytečná, protože většinou se nic neděje. A vzhledem k tomu, že jsem povoláním informatik, budu toto práci věnovat tomu jaké je moje řešení automatizace babiččina chovu.
\newline
Práce se dělí na teoretickou a praktickou část. V teoretické části jsou popsány základní pojmy a principy, jež jsou důležité pro plné chápání práce. Praktická část popisuje konkrétní implementaci a nasazení asistenčního systému pro chov hospodářských zvýřat. Čtenář může využít znalosti, které nasbírá během čtení teoretické části a jež mu následně pomůže chápat praktická část, jako návod k realizaci vlastního systému dle jeho potřeb.
\newline
Svojí konkrétní implementaci jsem zaměřil na chov Kura domácího z výše popsaných osobních důvodů, a protože byl pro mě nejdostupnější testovací zvíře. Systém jsem realizoval mikroservisní architekturou a jednotlivé služby jsou psány v populárnímu jazyce Python. Jako GUI rozhraní je šikovně použit open source software pro chytrou domácnost Home Assistant.
\newline
Ukázkový běh aplikace je dostupný na url adrese https://coopmaster.jarousnemec.cz/ s přihlašovacími údaji do Home Assistanta uživatel: \textbf{visitor} a heslo: \textbf{Heslo1234}. Zdrojový kód k jednotlivým službám a konfiguraci Home Assistanta je dostupný na serveru github.com.


    \chapter{Teoretická část}
\section{Kontejnerizace a docker compose}
\section{Mikroservisní architektura}
\section{Jazyk Python}
\section{Jazyky Wire a C++ }
\section{Git}
\section{Github workflow}
\section{Mqtt}
\section{Home Assistant}
\section{Arduino}
\section{Ip kamera a RTSP}
\section{Cloudflare tunneling}
\section{Flask}
\section{Strojové učení}
\section{Yolo Ultralytics}
\section{Tenzometrický senzor}
\section{Poe}
\section{Wifi extender}
\section{TailScale vpn}
\section{Raspberry PI}
\section{Raspberry PI OS}
\section{Yaml}
\section{GUI}






    \chapter{Praktická část}

\section{Rozvržení person}
- bylo potřeba rozvrhnout pro koho je aplikace určena\newline
- zjistilo se že prymárně pro mě ale babička by s tím asi taky neměla mít problém\newline
- takže aplikace je primárně pro člověka který rozumý chovu slepic a měla by šetřit jeho čas ve smyslu že zautomatizuje každodenní činnosti

\section{Výběr technologií}
- vybrali jsme si python a pro clasifikaci framework od ultralitics což znamená modely YOLO\newline
- jako databáze pro sběr statistik byl vybrán postgres pro svou robustnost a dobrou cenu\newline
- mohlo se to psát klidně i v C# a použít třeba azure pro classifikaci\newline
- pro kontejnerizaci jsme zvolili docker; dalo se i třeba podman nebo podobné ale toto se učíme ve škole a rád si to zopakuji a zlepším tak své dovednosti\newline
- python pro mě byla trochu výzva ale dopadlo to dobře
\section{Rozmišlení konkrétní implementace}
- budeme potřebovat váhu pro kontrolu hnízd zda tam slepice je a nebo kolik je tam vajec\newline
- budeme potřebovat ideálně ip kamery s vhodným IP krytím abychom mohly monitorovat kurník vevnitř a venku\newline
- budeme potřebovat ovladačku asi arduino pro dveře, světlo, senzory teploty a vlhkosti\newline
- předchozí věci je potřeba dostat na síť takže nějaké rpi pro předávání komunikace a směrování\newline
- a všechno to musí řídit něco s dostatečným výkonem pro klasifikace třeba my tu máme nucka s RTX2080\newline
- uživatelské rozhraní se rozhodlo že je zbytečné vyrábět vlastní a ztrácet tím čas lepší bude použít HA který disponuje všemi funkcemi je open source a má komunitu která ho udržuje
- první řešení ale bude na stole takže se nemusíme zaobírat detaily kontkrétní instalace, alespoň pro zatím

\section{Návrh architektury}
- aplikace se rozvrhla do několika modulů\newline
- jako první moduly pro komunikaci s hardware nazvané drivery; takže vzniknul scale driver pro komunikaci s váhou, kamerami a arduinem pro dveře, světlo a senzory\newline
- následně vzniky služby na základě požadovaných funkcí nest watcher, chicken watch guard, dog alarm, room assistant a ještě drobná službyčka health checker pro kontrolu a reportování systému a jeho chyb\newline
- drivery komunikují s výkonými službami pomosí http protokolu \newline
- výkonné služby komunikují s home assistantem pomocí mqtt realizovaným mosquitto serverem

\section{Implementace jednotlivých modulů}
- implementace probíhala v jazyce python za využítí vypsaných knihoven viz readmečka modulů\newline
- jak se konfiguroval logger, flask blueprinty, jak propojit python a arduino, jak na to s cronem/schedulerem, jak posílat mqtt a přijímat(jaká je struktura našich topiců, jak se to pojmenovává)
- verzuje se to na github
- na githubu běží workflow které vytváří jednolivé docker image pro každý modul a uploaduje je na můj docker hub kvůli snadnému deployi a distribuci po internetu

\section{Tvorba GUI rozhraní}
- vybral se teda ten home assistant a ted ho nakonfit\newline
- konfiguruje se to přes yaml configuration.yaml což je hlavní konfigurák HA\newline
- byla výzva přijít na to jak se přidávají vlastní mqtt senzory viz seznam závad v trellu\newline
- po přidání senzorů jsem si hrál s vizualizací jednotlivých hnízd tak aby to pro uživatele bylo přívětivě\newline
- napsal jsem proto vlastní komponentu pro HA viz foto a trello\newline
- zajímavé zjištění bylo že state který jsem využíval pro předávání textových zpráv má maximální velikost 255 bytů\newline
- takže jsem z jsonu přešel na csv\newline
- pokecat trochu o tom jak vytvořit takovou komponentu jaké to má části a předpoklady a jak se to následně přidává a konfiguruje v HA\newline
- následně pak konfigurace automations pro dveře, přepínačů pro manuální ovladání světla a dvěří, obrázků z kamer které taky chodí pomocí mqtt, a následně ještě dog alert aby se poslalo upozornění pokud je mobilní apka a na dashboardu vyskočil daný obrázek a výstrahou\newline
- závěrem pak výpis teploty, vlhkosti a počtu slepic v kurníku které systém poznal\newline

\section{První pracovní zapojení a běh}
- zjistilo se že bez poe je to špatná volba \newline
- kamery žerou dost proudu\newline
- bylo potřeba odladit nastavení kamer a jejich statické ip adresy, kamery měli zabezpečení na blokování ip address a tak\newline
- byl problém s kontakty na váze takže se nakonec museli vyměnit lisované za pájené\newline
- na stole to většinou funguje dobře \newline

\section{Nasazení do kurníku}
- bylo potřeba natahat elektřinu a internet \newline
- elektřina nebyla problém vzala se z kůlny zkrz díru ve zdi\newline
- horší to bylo s netem protože wifi signál do kurníku nedosáhne\newline
- vyřešilo se to wifi extenderem od tplinku který má zárověň i ethernet výstup díky čemuž nemuselisme dlouhého tahati kabelu a šlo nám to vzduhem přes dvůr a ve stodole pak drátem\newline
- kamery bylo potřeba napájet a taky připojit přes internet; zprvu jsem si myslel že poe nebude potřeba ale taha 230 by bylo zbytečné námahy takže jsem pořídil poe switch od tplinku a ten připojuje kamery\newline
- bylo někde potřeba udělat místo kam se nainstaluje celá technologie kurníku\newline
- zvolila se plechová bedna ze starého domovního rozvaděče do níž se pro jednotlivé prvky vytvožili na míru držáčky a pouzdra ps.: fotky z tisku a pak z rozvaděče\newline
- jak se zrealizovala váha\newline
- jak vypadá řídící jednotka pri room assistanta\newline
- jaké tam jsou dveře pro slepice\newline
- celé je to propojené s rpi 5 které to v kurníku řídí a s ním pak komunikuje nuc pomocí tail scailu ale to zas v další kapitole\newline
- bylo potřeba zkalibrovat hmotnosti slepic a vajec

\section{Konfigurace vzdáleného přístupu}
- rpi propojene s nuckem a dev kompama přes tailscail\newline
- v docker compose na nuckoj se vyrobila nová network jenom pro home assistanta a jeho propagaci na venek\newline
    - do vzniklé sítě se přidal ještě kontejner Cloudflared který zajišťuje tunel ven na cloudflare a přes jeho firewall a proxyny do internetu\newline
- cloudflare tunel bylo potřeba namapovat na doménu kterou vlastníme\newline
    - pronajmul jsem si doménu u doméhového registrátora forpsi\newline
- zmínění jaká plynou nebezpečí z tohoto řešení

\section{Rozvržení práce do budoucna}
- jak by se řešení dalo zoptimalizovat\newline
- vylepšení modelu pro classifykaci\newline
- vylepšení a zpřesnění vah\newline
- kontrola napajedla\newline
- kontrola krmítka


    \chapter{Zamyšlení nad možnými dalšími příležitostmi pro automatizaci}

\section{Co vše se dá automatizovat?}
\begin{itemize}
    \item skleník
    \item akvarium
    \item kravín a pastviny
    \item záhony a zahrada
\end{itemize}

\section{Automatizace skleníku}
- kontrola vlhkosti, větrání na základě teploty ve skleníku, přidávání hnojiva do zálivky

\section{Chytré akvárium}
- chytré topení, filtrace, krmítko, kontrola kvality vody

\section{Monitoring kravína a pastvin}
- zranění kráva\newline
- cizí pes\newline
- cizí člověk

\section{Monitoring záhonů}
- pes krade mrkev\newline
- utekly slepice\newline
- zajíc\newline
- plíseň na bramborách\newline
- sucho

    %%% Závěr
    \newpage
\chapter{Vyhodnocení}\label{ch:vyhodnoceni}

\addcontentsline{toc}{section}{Závěr}
Závěrem lze říct, že jsme úspěšně vyvinuli systém, který je schopný v reálném čase pomáha plnit každodení úkony, které by hospodář musel jinak dělat sám. Díky naší aplikaci může člověk kontrolovat situaci v kurníku případně i ve výběhu. Systém ho sám informuje, pokud je něco v nepořádku, například v případě, že je ve výběhu vetřelec, který by mohl představovat nebezpečí pro chované slepice. Dále se nám na podobném principu povedlo implementovat kontrolu počtu slepic v kurníku, což lze použít například při automatizaci zavírání dvířek. Systém také poskytuje aktuální záběry z bezpečnostních kamer díky, kterým si může hospodář sám kontrolovat situaci svého chovu a nemusí díky tomu být přítomen fyzicky. V poslední řadě jsme také implementovali část systému, která je schopna kontrolovat stavy jednotlivých hnízd a zjistit zda v něm sedí slepice případně kolik vajec v hnízdě je. Tato data lze také použít ke tvorbě statistik, díky nimž může člověk analizovat různé aspekty svého chovu.


\addcontentsline{toc}{section}{Použitá Literatura}
\printbibliography[title=Použitá Literatura]

\addcontentsline{toc}{section}{Seznam obrázků}
\listoffigures

\addcontentsline{toc}{section}{Seznam tabulek}
\listoftables

\addcontentsline{toc}{section}{Seznam rovnic}
\listoflistedequation

\end{document}
