\externaldocument{text-02-teoreticka}


\chapter{Praktická část}\label{ch:prakticka-cast}
Tato část práce předvede čtenáři konkrétní realizaci myšlenky probírané v této prací.
Zároveň v se v ní čtenář dozví jak v praxi využít poznatky nabyté v teoretické části.


\section{Rozvržení person}\label{sec:rozvrzeni-person}
Před tím než se započne tvorba jakékoli aplikace, je třeba určit, pro koho danou aplikaci tvoříme a jak bychom chtěli, aby jí tento uživatel používal.
V našem případě jde o to, že systém by měl být schopný používat člověk znalý v chovu hospodářských zvířat, ale často méně zdatný v používání informačních technologií a mobilních aplikací.
Na základě toho, že už víme, kdo je cílový uživatel můžeme se zamýšlet jakou funkcionalitu do řešení implementovat.
Jedná se tedy o to, aby aplikace umožňovala uživateli vzdálený dohled na jeho chov a pomáhala mu šetřit čas zautomatizováním každodenních činnosti.
Rozhodlo se tedy, že aplikace bude uživateli poskytovat následující funkce
\begin{itemize}
    \item Vzdálený přístup skrze internet odkudkoli
    \item Pohledy z bezpečnostních kamer ve vnitřním i venkovním výběhu
    \item Dálkové ovládání a automatizace světla a bezpečnostních dvířek v kurníku
    \item Intuitivní vizualizace stavů jednotlivých hnízd(sedící slepice, v opačném případě počet vajec)
    \item Automatické upozornění notifikacemi v telefonu na vetřelce ve výběhu
    \item Vizualizace aktuálních povětrnostních podmínek v kurníku (teplota a vlhkost)
\end{itemize}

\section{Návrh architektury a volba technologií}\label{sec:navrh-architektury-a-volba-technologii}
Díky tomu, že si přesně určíme, kdo je cílový uživatel naší aplikace, a k tomu dáme do hromady, co uživatel od naší aplikace očekává.
Jsme nyní schopni bez větších problémů teoreticky navrhnout architekturu našeho systému a přesně popsat požadovanou funkcionalitu, jakou budou disponovat jeho jednotlivé části.
Celý ekosystém se skládá z několika samostatných celků
\begin{itemize}
    \item Fyzická zařízení jako kamery a senzory
    \item Architektura systému obsahující hlavní funkcionalitu(sekce~\ref{sec:backend})
    \item GUI(sekce~\ref{sec:gui})
    \item Komunikace mezi GUI a backendem systému
    \item Zpřístupnění aplikace z internetu
\end{itemize}

\subsection{Architektura hlavního systému}\label{subsec:microservices}
Jako teoretický model na základě kterého, budeme organizovat a třídit funkcionalitu, jsem zvolil mikroservisní architekturu(sekce~\ref{sec:microservice-architecture}).
Tento způsob rozvržení zodpovědnosti jednotlivých částí systému jsem zvolil, kvůli velké možnosti rozšíření, zapouzdření funkcionality a snadné úpravě jednotlivých služeb bez nutnosti kompletního restartu systému případně znovunasazení.
Jako programovací jazyk pro tvorbu služeb je zvolen programovací jazyk Python(sekce~\ref{sec:ipcamera-rtsp}).
Python byl vybrán kvůli jeho univerzálnosti, snadné syntaxi a podpoře ze strany vývojářů a komunity.
Pro realizaci, konfiguraci a síťování mikroservistní architektury je použit Docker Engine(sekce~\ref{sec:kontejnerizace}) společně s rozšířením Docker Compose.
Na základě určeného způsobu zapouzdření funkcionality je architektura rozdělena do následujících služeb, které jsou pojmenovány dle jejich účelu
\begin{itemize}
    \item Camera driver
    \item Scale driver
    \item Room driver
    \item Health checker
    \item Room assistant
    \item Nest watcher
    \item Dog alarm
    \item Chicken watch guard
\end{itemize}

\subsection{GUI}\label{subsec:gui}
Aplikace musí mít rozhodně i grafické rozhraní.
Pro tento účel byla zvolena open source aplikace Home Assistant(sekce~\ref{sec:home-assistant}).
Home assistant byl zvolen kvůli jeho univerzálnosti, rozsáhlé podpoře, komunitě bohaté na custom řešení a obrovské možnosti konfigurace a přispůsobení, čehož pro ovládání a vizualizaci dat z našeho systému hojně využijeme.

\subsection{Komunikace mezi Backendem a Frontendem}\label{subsec:komunikace-mezi-backendem-a-frontendem}
Je třeba zajistit komunikaci mezi Home Assistantem a Službami.
Tuto úlohu musíme přijmout velice zodpovědně a navrhnout řešení, které půjde opět snadno rozšířit a modifikovat.
Nelze proto použít klasické HTTP(sekce~\ref{sec:http-rest}), z toho důvodu že bychom komunikaci vázali na buď doménové jméno a port nebo ip adresu a port.
Toto řešení má problém v tom, že pokud bychom potřebovali změnit, buď umístění částí služeb nebo port jedné ze služeb, bylo by třeba překonfigurovat i home assistanta.
Dalším problém nastane, když potřebujeme z některé ze služeb poslat například notifikaci do Home Assistanta, je pro to potřeba, aby jednotlivé služby věděli, kde na síti Home Assistant běží, což je věc, která se může měnit, a museli bychom naopak přenastavovat jednotlivé služby.
Jako řešení se nabízí použít messaging konkrétně třeba technologii MQTT(sekce~\ref{sec:mqtt}).
Tato technologie se běžně používá u IoT zařízení a pro naše použití bude vynikajícím řešením.

\subsection{Hardwarová zařízení}\label{subsec:hardwarova-zarizeni}
Tuto sekci je velmi důležité dobře a správně navrhnout, protože náš systém bude pracovat s daty z reálného světa.
Tato data musejí získávat k tomu určená zařízení a na jejich přesnosti závisý efektivita a správnost chování zbytku systému.
Musíme tedy navrhnout a realizovat několik zařízení, která společně budou plnit následující úlohy
\begin{itemize}
    \item snímání stavů hnízd
    \item ovládatní dvířek
    \item snímání teploty a vlhkosti
    \item snímání scén v kurníku a ve výběhu
\end{itemize}
\subsubsection{Snímání stavů hnízd}
Všechny případy, které potřebujeme v kontextu našeho hnízda řešit se dají rozpoznat na základě hmotnosti hnízda.
Pokud je v hnízdě slepice, hnízdo bude hodně zatížené oproti počátečnímu prázdnému stavu.
V případě, že bude hnízdo zatížené málo a nebo středně, může to pro nás ve většině případů spolehlivě znamenat, že se v hnízdě nacházejí vejce.
Nestandartní situace jako neobviklý nepořánek v hnízdě případně výkal, řešit nebudeme, protože by to exponenciálně zvýšilo složitost řešení a nepřineslo by to o moc lepší data.
Kurník ja navíc v našem případě pravidelně udržován, takže šance na výskyt chybového faktoru v podobě nežádoucí zátěže je nízká.
Jako nejjednodužší prostředek pro splnění tohoto úkolu na základě předchozí analýzy se ukázala obyčejná digitální váha.
Tu bude třeba vytvořit vlastní konstrukce vzhledem ke skutečnosti, že ji musíme zabudovat do již existujícího hnízda.
Popis konkrétního zapojení a konstrukce je sepsán níže v sekci~\ref{subsec:digitalni-vaha-do-hnizda}.

\subsubsection{Ovládatní dvířek, snímání teploty a vlhkosti}
Pro ovládatní dvířek a snímání povětrnostních podmínek je nejrozumnější vytvořit řídící jednotku, která tyto požadavky obsáhne.
Další částí částí jsou samotná elektrická dvířka, jejichž funkce bude na základě způsobu elektrického signálu být otevřená nebo zavřená.
Teplota a vlhkost jsou snímány senzory, jež jsou souřástí řídící jednotky.
Implementace je popsána v sekci~\ref{subsec:ridici-jednotka}.
\subsubsection{Snímání scén v kurníku a ve výběhu}
Ke snímání scén v daných oblastech je třeba vyprat zařízení s dostatečnou odolností proti povětrnostním vlivům.
Speciálně v kurníku je hodně agresivní prostředí kvůli vysoké vlhkosti a pro dlouhodobý provoz je třeba vybrat kvalitní a odolnou techniku.
Zařízení musí být připojeno přes ethernet a musí být napájené ze sítě nikoli pomocí baterie.
Na základě předchozí analýzy je vhodné použít venkovní ip kameru.
Ip kamera s dostatečným krytím, dobrou kvalitou obrazu a velkým zorným polem bude snadná na instalaci, je cenově dostupná a poskytuje dostatečně kvalitní data pro naše použití při detekci objektů.
Podrobnosti ohledně výběru kamery se nachází v sekci~\ref{subsec:kamerovy-system}

\subsection{Zpřístupnění aplikace z internetu}\label{subsec:zpristupneni-aplikace-z-internetu}
Aby systém mohl sloužit svému účelu musí být vzdáleně přístupný.
To znamená, že server, na kterém systém poběží, musí mít veřejnou ip adresu a v ideálním případě mít tuto adresu propojenou s doménovým názvem pro snadnější použití.
Tato problematika se dá řešit několika způsoby.\newline
Jeden ze způsobů je zařízení si přímo veřejné ip adresy pro svůj server.
Toto řešení je, technicky i teoreticky náročné, protože v případě, se kterým pracujeme my, běží server na lokální síti a veřejná ip adresa tedy není přímo pro náš server, ale celou síť.
To vytváří v určitých situacích vytváří velmi vážná bezpečnostní rizika, díky čemuž je tato metoda velice náročná na hardware a implementaci.
Navíc poskytování veřejné ip adresy je placená služba poskytovatel internetového připojení.\newline
Další způsob je nasazení části systému do některého z cloudových řešení jako je Microsoft Azure nebo Amazon AWS.
Tato metody by v podstatě vyhovovala našim požadavkům, ale pokud bychom se bavili o ceně, tak to rozhodně není levné řešení. \newline
Metoda, kterou využíváme v řešení my, je snadná, jednodužší na implementaci než zřizování veřejné ip adresy a levnější nez využití cloudové služba.
Tento způsob využívá takzvané tunely neboli šifrovaná spojení.
Funguje na principu, při němž se zařízení šifrovaně propojíte s veřejným serverem a tento server bude skrze sebe vystavovat službu, která běží na lokálním počítači například u nás doma.
Výhoda je, že vzdálený server disponuje kvalitním a odborně nastaveným firewalem, ktery zajistí dobrou ochranu našeho lokálního serveru.
Tuto službu poskytuje zdarma a bez omezení například společnost Cloudflare.
Tento způsob nejvíce odpovídá našim požadavkům, které byli cena a jednoduchost společně s rychlost nasazení.\newline
Poslední nutnostní, kterou ovšem je třeba zaplatit, je pronájem vlastního doménového jména.
Doménu si lze pronajmout u doménových registrátorů jako jsou GoDaddy, Hostinger nebo například Forpsi.
Ceny domén se odvíjí na základě jejího druhu a pohybují od 20 do 1500 korun za kus.
My v našem řešení využíváme službu Forpsi, protože s ní máme již předchozí zkušenost.\newline
Dále je tato část rozvedena v sekci~\ref{sec:nasazeni-do-kurniku}, kde je popsán konkrétní způsob nasazení do funkčního projektu.

\section{Detailní popis služeb a jejich úloh v řešení}

\subsection{Camera driver}
Camera driver je služba zodpovědná za komunikaci s ip kamerou(sekce~\ref{sec:ipcamera-rtsp}).\newline
S ní komunikuje pomocí protokolu RTSP(sekce~\ref{sec:ipcamera-rtsp}).
Url ip kamery, k níž je driver přiřazen, je službě předávána pomocí environment proměnných(sekce~\ref{sec:environment-variables}).
Pro komunikaci se zbytkem služeb poskytuje Camera driver své REST(sekce~\ref{sec:http-rest}) api.
Konkrétně při zavolání na endpoint driver stáhne nejnovější obrázek z kamery a vrátí ho jako odpověď na volání.
%\begin{itemize}
%    \item GET /api/image
%\end{itemize}

\subsection{Scale driver}
Scale driver zodpovídá za komunikaci mezi fyzickou váhou a službami, které využívají data o vážení.\newline
Protože jako řídící jednotka váhy je použito Arduino(\ref{sec:arduino}) tento modul komunikuje přes serialový port pomocí protokolu USB a na dotaz přijmutý RESTovým api poskytne jako odpověď hodnotu načtenou z váhy.
Pro služby v systému tento driver poskytuje data o hmotnosti opět pomocí REST api.
%\begin{itemize}
%    \item GET /api/weight
%\end{itemize}

\subsection{Room driver}
Room driver zařizuje komunikaci mezi ostatními službami a řídící jednotkou v kurníku, která ovládá dveře a světlo.\newline
Jako mozek řídící jednotky je použito opět Arduino a tomu je třeba přizpůsobit architekturu služby.
Tento modul má tedy za úkol přes serialový port pomocí protokolu USB posílat příkazy a načítat stavy řídící jednotky na základě requestů příchozích na REST api služby.
Pro služby v systému služba na vystavuje GET a POST endpointy
%\begin{itemize}
%    \item GET /api/temperature (dej teplotu)
%    \item GET /api/humidity (dej vlhkost)
%    \item GET /api/lamp/state (dej stav světla vypnuto/zapnuto)
%    \item GET /api/door/state (dej pozici dvířek otevřeno/zavřeno)
%    \item POST /api/lamp/on (rozsviť)
%    \item POST /api/lamp/off (zhasni)
%    \item POST /api/door/open (otevři)
%    \item POST /api/door/close (zavři)
%\end{itemize}

\subsection{Health checker}
Health checker je malá služba určená pro správce systému.\newline
Poskytuje informace o tom zda všechny potřebné služby běží, aby správce nebyl nucen přihlašovat se vzdáleně na server a manuálně kontrolovat každou službu.
Výpis stavů jednotlivých služeb je poskytován RESTovým api [GET] /status
%\begin{itemize}
%    \item GET /status
%\end{itemize}

\subsection{Room assistant}
Room assistant je zpřístupňuje komunikaci mezi GUI tedy konkrétně Home Assistantem a Room driverem.\newline
S Home Assistantem je komunikace realizována pomocí MQTT(\ref{sec:mqtt}) a s Room driverem pomocí HTTP(\ref{sec:http-rest}) protokolu.
Základní funkcí je zpracování a přeposlání příkazů do Room Driveru odebíraných z témat
%\begin{itemize}
%    \item coopmaster/room/door/cmnd
%    \item coopmaster/room/lamp/cmnd
%\end{itemize}
Služba očekává, že na tato témata budou chodit zprávy open / close pro dveře a on / off jako příkazy pro světlo.
Další úlohou je periodické načítání a aktualizace informací o stavu dveří, světla a dat z teplotního a vlhkostního senzorů.
Tato data by měl Room assistant pravidelně načítat a posílat přes MQTT do Home Assistanta.
%\begin{itemize}
%    \item coopmaster/room/temperature
%    \item coopmaster/room/humidity
%    \item coopmaster/room/door/state
%    \item coopmaster/room/lamp/state
%\end{itemize}

\subsection{Nest watcher}
Tato služba interpretuje stavy jednotlivých hnízd v kurníku pro Home Assistanta.
Hlavní funkcí je načítání a analýza dat z jednotlivých vah v hnízdech, která jsou reprezentována Scale Drivery.\newline
Data jsou načítána několikrát do minuty a ukládána do databáze s časovým údajem, kdy byl záznam vytvořen.
Následně se jednou za minutu vyhodnotí průměrná hodnota během posledních několika vážení.
Na základě tohoto údaje jsme schopni zjisti několik případů
\begin{itemize}
    \item hnízdo je prázdné (hodnota na váze nepřevyšuje 50 g )
    \item v hnízdě se nacházejí vejce (hmotnost jednoho vejce je průměrně 50 g)
    \item v hnízdě sedí slepice (hmotnost slepice se pohybuje okolo 1200 g a více)
\end{itemize}
Pokud je na váze průměrně méně než 50 g, vzhledem k možným chybám měření, takový případ vyhodnotíme jako, že je hnízdo prázdné.
Jestliže se hodnota pohybuje mezi 50 a 1200 gramy, znamená to, že v hnízdě jsou pravděpodobně vejce, a jejich počet je vypočítán vydělením celkové hmotnosti a hmotnosti jednoho vejce.
V případě, že je na váze více jak 1200 g, vyhodnotí služba, že v hnízdě sedí slepice.
Tyto tři zmíněné informace služba následně pomocí MQTT předává do Home Assistanta.

\subsection{Dog alarm}
Služba Dog alarm má detekovat nebezpečí ve výběhu a poslat tuto zprávu do Home Assistanta.\newline
Aktuální záběry jsou pomocí GET requestů stahovány z konkrétní instance služby Camera driver, která je přiřazena ke kameře ve výběhu.
Analýza probíhá v určitých intervalech za pomocí umělé inteligence, kde je konkrétně použita metoda detekce objektů.
Jakmile jako výsledek klasifikace vyjde jednoznačně, že v záběru byl spatřen pes nebo jiný predátor, je zpráva poslána pomocí MQTT do Home Assistanta společně s konkrétním záběrem, na němž byl predátor detekován.
Tato služba zároveň přes MQTT posílá do Home Assistanta aktuální záběr z kamery.

\subsection{Chicken watch guard}
Úkolem služby Chicken watch guard je sledovat stav a počet slepic v kurníku.\newline
Aktuální záběry jsou stejně jako u Dog alarmu stahovány z konkrétního Camera driveru v kurníku.
Následně po získání záběru proběhne detekce objektů a počet těchto objektů udává počet detekovaných slepic v obraze.
Tato hodnota je publikována pomocí MQTT do Home Assistanta.
Vedlejší funkcí služby je průběžné posílání aktuálního pohledu z kamery v kurníku do Home Assistanta.

\section{Implementace služeb}\label{sec:implementace-sluzeb}
- implementace probíhala v jazyce python za využítí vypsaných knihoven viz readmečka modulů\newline
- jak se konfiguroval logger, flask blueprinty, jak propojit python a arduino, jak na to s cronem/schedulerem, jak posílat mqtt a přijímat(jaká je struktura našich topiců, jak se to pojmenovává)
- verzuje se to na github
- na githubu běží workflow které vytváří jednolivé docker image pro každý modul a uploaduje je na můj docker hub kvůli snadnému deployi a distribuci po internetu

\section{Implementace hardwarových prvků}\label{sec:implementace-hardwarovych-prvku}
\subsection{Digitální váha do hnízda}\label{subsec:digitalni-vaha-do-hnizda}

\subsection{Řídící jednotka}\label{subsec:ridici-jednotka}

\subsection{Kamerový systém}\label{subsec:kamerovy-system}


\section{Tvorba GUI rozhraní}\label{sec:tvorba-gui-rozhrani}
- vybral se teda ten home assistant a ted ho nakonfit\newline
- konfiguruje se to přes yaml configuration.yaml což je hlavní konfigurák HA\newline
- byla výzva přijít na to jak se přidávají vlastní mqtt senzory viz seznam závad v trellu\newline
- po přidání senzorů jsem si hrál s vizualizací jednotlivých hnízd tak aby to pro uživatele bylo přívětivě\newline
- napsal jsem proto vlastní komponentu pro HA viz foto a trello\newline
- zajímavé zjištění bylo že state který jsem využíval pro předávání textových zpráv má maximální velikost 255 bytů\newline
- takže jsem z jsonu přešel na csv\newline
- pokecat trochu o tom jak vytvořit takovou komponentu jaké to má části a předpoklady a jak se to následně přidává a konfiguruje v HA\newline
- následně pak konfigurace automations pro dveře, přepínačů pro manuální ovladání světla a dvěří, obrázků z kamer které taky chodí pomocí mqtt, a následně ještě dog alert aby se poslalo upozornění pokud je mobilní apka a na dashboardu vyskočil daný obrázek a výstrahou\newline
- závěrem pak výpis teploty, vlhkosti a počtu slepic v kurníku které systém poznal\newline


\section{Rozmyšlení konkrétní implementace}\label{sec:rozmysleni-konkretni-implementace}
- budeme potřebovat váhu pro kontrolu hnízd zda tam slepice je a nebo kolik je tam vajec\newline
- budeme potřebovat ideálně ip kamery s vhodným IP krytím abychom mohly monitorovat kurník vevnitř a venku\newline
- budeme potřebovat ovladačku asi arduino pro dveře, světlo, senzory teploty a vlhkosti\newline
- předchozí věci je potřeba dostat na síť takže nějaké rpi pro předávání komunikace a směrování\newline
- a všechno to musí řídit něco s dostatečným výkonem pro klasifikace třeba my tu máme nucka s RTX2080\newline
- uživatelské rozhraní se rozhodlo že je zbytečné vyrábět vlastní a ztrácet tím čas lepší bude použít HA který disponuje všemi funkcemi je open source a má komunitu která ho udržuje
- první řešení ale bude na stole takže se nemusíme zaobírat detaily kontkrétní instalace, alespoň pro zatím


\section{První pracovní zapojení a běh}\label{sec:prvni-pracovni-zapojeni-a-beh}
- zjistilo se že bez poe je to špatná volba \newline
- kamery žerou dost proudu\newline
- bylo potřeba odladit nastavení kamer a jejich statické ip adresy, kamery měli zabezpečení na blokování ip address a tak\newline
- byl problém s kontakty na váze takže se nakonec museli vyměnit lisované za pájené\newline
- na stole to většinou funguje dobře \newline


\section{Nasazení do kurníku}\label{sec:nasazeni-do-kurniku}
- bylo potřeba natahat elektřinu a internet \newline
- elektřina nebyla problém vzala se z kůlny zkrz díru ve zdi\newline
- horší to bylo s netem protože wifi signál do kurníku nedosáhne\newline
- vyřešilo se to wifi extenderem od tplinku který má zárověň i ethernet výstup díky čemuž nemuselisme dlouhého tahati kabelu a šlo nám to vzduhem přes dvůr a ve stodole pak drátem\newline
- kamery bylo potřeba napájet a taky připojit přes internet; zprvu jsem si myslel že poe nebude potřeba ale taha 230 by bylo zbytečné námahy takže jsem pořídil poe switch od tplinku a ten připojuje kamery\newline
- bylo někde potřeba udělat místo kam se nainstaluje celá technologie kurníku\newline
- zvolila se plechová bedna ze starého domovního rozvaděče do níž se pro jednotlivé prvky vytvožili na míru držáčky a pouzdra ps.: fotky z tisku a pak z rozvaděče\newline
- jak se zrealizovala váha\newline
- jak vypadá řídící jednotka pri room assistanta\newline
- jaké tam jsou dveře pro slepice\newline
- celé je to propojené s rpi 5 které to v kurníku řídí a s ním pak komunikuje nuc pomocí tail scailu ale to zas v další kapitole\newline
- bylo potřeba zkalibrovat hmotnosti slepic a vajec


\section{Konfigurace vzdáleného přístupu}\label{sec:konfigurace-vzdaleneho-pristupu}
- rpi propojene s nuckem a dev kompama přes tailscail\newline
- v docker compose na nuckoj se vyrobila nová network jenom pro home assistanta a jeho propagaci na venek\newline
- do vzniklé sítě se přidal ještě kontejner Cloudflared který zajišťuje tunel ven na cloudflare a přes jeho firewall a proxyny do internetu\newline
- cloudflare tunel bylo potřeba namapovat na doménu kterou vlastníme\newline
- pronajmul jsem si doménu u doméhového registrátora forpsi\newline
- zmínění jaká plynou nebezpečí z tohoto řešení


\section{Rozvržení práce do budoucna}\label{sec:rozvrzeni-prace-do-budoucna}
- jak by se řešení dalo zoptimalizovat\newline
- vylepšení modelu pro classifykaci\newline
- vylepšení a zpřesnění vah\newline
- kontrola napajedla\newline
- kontrola krmítka
- implementace autonomního chování do Room Assistanta


\section{Ekonomická stránka projektu}\label{sec:ekonomicka-stranka-projektu}

