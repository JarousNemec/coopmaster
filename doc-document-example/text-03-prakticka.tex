\chapter{Praktická část}
Tato část práce předvede čtenáři konkrétní realizaci myšlenky probírané v této prácí.
Zárověň v se v ní čtenář dozví jak v praxi využít poznatky nabyté v teoretické části.

\section{Rozvržení person}
Před tím než se započne tvorba jakékoli aplikace, je třeba určit, pro koho danou aplikaci tvoříme a jak bychom chtěli, aby jí tento uživatel používal.
V našem případě jde o to, že systém by měl být schopný používat člověk znalý v chovu hospodářských zvířat, ale často méně zdatný v používání informačních technologií a mobilních aplikací.
Na základě toho, že už víme, kdo je cílový uživatel můžeme se zamýšlet jakou funkcionalitu do řešení implementovat.
Jedná se tedy o to, aby aplikace umožňovala uživateli vzdálený dohled na jeho chov a pomáhala mu šetřit čas zautomatizováním každodenních činnosti.
Rozhodlo se tedy, že aplikace bude uživateli poskytovat následující funkce
\begin{itemize}
    \item Vzdálený přístup zkrze internet odkudkoli
    \item Pohledy z bezpečnostních kamer ve vnitřním i venkovním výběhu
    \item Dálkové ovládání a automatizace světla a bezpečnostních dvířek v kurníku
    \item Intuitivní vizualizace stavů jednotlivých hnízd(sedící slepice, v opačném případě počet vajec)
    \item Automatické upozornění notifikacemi v telefonu na vetřelce ve výběhu
    \item Vizualizace aktuálních povětrnostních podmínek v kurníku (teplota a vlhkost)
\end{itemize}

\section{Návrh architektury}
- aplikace se rozvrhla do několika modulů\newline
- jako první moduly pro komunikaci s hardware nazvané drivery; takže vzniknul scale driver pro komunikaci s váhou, kamerami a arduinem pro dveře, světlo a senzory\newline
- následně vzniky služby na základě požadovaných funkcí nest watcher, chicken watch guard, dog alarm, room assistant a ještě drobná službyčka health checker pro kontrolu a reportování systému a jeho chyb\newline
- drivery komunikují s výkonými službami pomosí http protokolu \newline
- výkonné služby komunikují s home assistantem pomocí mqtt realizovaným mosquitto serverem
\subsection{Výběr technologií}
- vybrali jsme si python a pro clasifikaci framework od ultralitics což znamená modely YOLO\newline
- jako databáze pro sběr statistik byl vybrán postgres pro svou robustnost a dobrou cenu\newline
- mohlo se to psát klidně i v C\# a použít třeba azure pro classifikaci\newline
- pro kontejnerizaci jsme zvolili docker; dalo se i třeba podman nebo podobné ale toto se učíme ve škole a rád si to zopakuji a zlepším tak své dovednosti\newline
- python pro mě byla trochu výzva ale dopadlo to dobře

\section{Implementace jednotlivých modulů}
- implementace probíhala v jazyce python za využítí vypsaných knihoven viz readmečka modulů\newline
- jak se konfiguroval logger, flask blueprinty, jak propojit python a arduino, jak na to s cronem/schedulerem, jak posílat mqtt a přijímat(jaká je struktura našich topiců, jak se to pojmenovává)
- verzuje se to na github
- na githubu běží workflow které vytváří jednolivé docker image pro každý modul a uploaduje je na můj docker hub kvůli snadnému deployi a distribuci po internetu

\section{Tvorba GUI rozhraní}
- vybral se teda ten home assistant a ted ho nakonfit\newline
- konfiguruje se to přes yaml configuration.yaml což je hlavní konfigurák HA\newline
- byla výzva přijít na to jak se přidávají vlastní mqtt senzory viz seznam závad v trellu\newline
- po přidání senzorů jsem si hrál s vizualizací jednotlivých hnízd tak aby to pro uživatele bylo přívětivě\newline
- napsal jsem proto vlastní komponentu pro HA viz foto a trello\newline
- zajímavé zjištění bylo že state který jsem využíval pro předávání textových zpráv má maximální velikost 255 bytů\newline
- takže jsem z jsonu přešel na csv\newline
- pokecat trochu o tom jak vytvořit takovou komponentu jaké to má části a předpoklady a jak se to následně přidává a konfiguruje v HA\newline
- následně pak konfigurace automations pro dveře, přepínačů pro manuální ovladání světla a dvěří, obrázků z kamer které taky chodí pomocí mqtt, a následně ještě dog alert aby se poslalo upozornění pokud je mobilní apka a na dashboardu vyskočil daný obrázek a výstrahou\newline
- závěrem pak výpis teploty, vlhkosti a počtu slepic v kurníku které systém poznal\newline

\section{Rozmyšlení konkrétní implementace}
- budeme potřebovat váhu pro kontrolu hnízd zda tam slepice je a nebo kolik je tam vajec\newline
- budeme potřebovat ideálně ip kamery s vhodným IP krytím abychom mohly monitorovat kurník vevnitř a venku\newline
- budeme potřebovat ovladačku asi arduino pro dveře, světlo, senzory teploty a vlhkosti\newline
- předchozí věci je potřeba dostat na síť takže nějaké rpi pro předávání komunikace a směrování\newline
- a všechno to musí řídit něco s dostatečným výkonem pro klasifikace třeba my tu máme nucka s RTX2080\newline
- uživatelské rozhraní se rozhodlo že je zbytečné vyrábět vlastní a ztrácet tím čas lepší bude použít HA který disponuje všemi funkcemi je open source a má komunitu která ho udržuje
- první řešení ale bude na stole takže se nemusíme zaobírat detaily kontkrétní instalace, alespoň pro zatím

\section{První pracovní zapojení a běh}
- zjistilo se že bez poe je to špatná volba \newline
- kamery žerou dost proudu\newline
- bylo potřeba odladit nastavení kamer a jejich statické ip adresy, kamery měli zabezpečení na blokování ip address a tak\newline
- byl problém s kontakty na váze takže se nakonec museli vyměnit lisované za pájené\newline
- na stole to většinou funguje dobře \newline

\section{Nasazení do kurníku}
- bylo potřeba natahat elektřinu a internet \newline
- elektřina nebyla problém vzala se z kůlny zkrz díru ve zdi\newline
- horší to bylo s netem protože wifi signál do kurníku nedosáhne\newline
- vyřešilo se to wifi extenderem od tplinku který má zárověň i ethernet výstup díky čemuž nemuselisme dlouhého tahati kabelu a šlo nám to vzduhem přes dvůr a ve stodole pak drátem\newline
- kamery bylo potřeba napájet a taky připojit přes internet; zprvu jsem si myslel že poe nebude potřeba ale taha 230 by bylo zbytečné námahy takže jsem pořídil poe switch od tplinku a ten připojuje kamery\newline
- bylo někde potřeba udělat místo kam se nainstaluje celá technologie kurníku\newline
- zvolila se plechová bedna ze starého domovního rozvaděče do níž se pro jednotlivé prvky vytvožili na míru držáčky a pouzdra ps.: fotky z tisku a pak z rozvaděče\newline
- jak se zrealizovala váha\newline
- jak vypadá řídící jednotka pri room assistanta\newline
- jaké tam jsou dveře pro slepice\newline
- celé je to propojené s rpi 5 které to v kurníku řídí a s ním pak komunikuje nuc pomocí tail scailu ale to zas v další kapitole\newline
- bylo potřeba zkalibrovat hmotnosti slepic a vajec

\section{Konfigurace vzdáleného přístupu}
- rpi propojene s nuckem a dev kompama přes tailscail\newline
- v docker compose na nuckoj se vyrobila nová network jenom pro home assistanta a jeho propagaci na venek\newline
    - do vzniklé sítě se přidal ještě kontejner Cloudflared který zajišťuje tunel ven na cloudflare a přes jeho firewall a proxyny do internetu\newline
- cloudflare tunel bylo potřeba namapovat na doménu kterou vlastníme\newline
    - pronajmul jsem si doménu u doméhového registrátora forpsi\newline
- zmínění jaká plynou nebezpečí z tohoto řešení

\section{Rozvržení práce do budoucna}
- jak by se řešení dalo zoptimalizovat\newline
- vylepšení modelu pro classifykaci\newline
- vylepšení a zpřesnění vah\newline
- kontrola napajedla\newline
- kontrola krmítka
