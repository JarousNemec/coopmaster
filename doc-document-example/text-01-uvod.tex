\chapter{Úvod}
V dnešní době je populární využívat moderní technologie a převážně umělou inteligenci v různých aplikacích jako například monitoring pohybu zákazníků v obchodech nebo ve formě textových modelů čímž je například známe ChatGPT. Zároveň dnes ubývá lidí, kteří by se chtěli zabývat staráním se o hospodářská zvýřata a tudíž je potřeba je nějak nahradit. Na základě těchto faktů jsem se rozhodl napsat práci popisující konkrétní využití umělé inteligence a dalších technologií při chovu hospodářských zvýřat a jak nám tyto technologie může pomoci.
\newline
Práce se dělí na teoretickou a praktickou část. V teoretické části jsou popsány základní pojmy a principy, jež jsou důležité pro plné chápání práce. Praktická část popisuje konkrétní implementaci a nasazení asistenčního systému pro chov hospodářských zvýřat. Čtenář může znalosti, které nasbírá během čtení teoretické části a jež mu následně pomůže chápat praktická část, může využít jako návod k realizaci vlastního systému dle jeho potřeb.
\newline
Svojí konkrétní implementaci jsem zaměřil na chov Kura domácího, protože byl pro mě nejdostupnější testovací zvíře. Systém jsem realizoval mikroservisní architekturou a jednotlivé služby jsou sány v populárnímu jazyce Python. Jako GUI rozhraní je šikovně použit open source software pro chytrou domácnost Home Assistant.
\newline
Ukázkový běh aplikace je dostupný na url adrese https://coopmaster.jarousnemec.cz/ s přihlašovacími údaji do Home Assistanta uživatel: \textbf{visitor} a heslo: \textbf{Heslo1234}. Zdrojový kód k jednotlivým službám a konfiguraci Home Assistanta je dostupný na serveru github.com.

\newpage