\chapter{Úvod}
V dnešní době je populární využívat moderní technologie a umělou inteligenci v různých aplikacích jako například monitoring pohybu zákazníků v obchodech nebo ve formě textových modelů čímž je například známe ChatGPT. Zároveň dnes ubývá lidí, kteří by se chtěli zabývat staráním se o hospodářská zvýřata a tudíž je potřeba, aby něco zaujmulo jejich místo. Zároveň jsem sám člověk, který rád zkoumá nové věci v oblasti informatiky a velice ho baví automatizování nejrůznějších úloh. Na základě těchto faktů jsem se rozhodl vytvořit téma práci popisující využití moderních technologií při chovu hospodářských zvýřat. Nápad na tuto práci vznikl díky mojí babičce, která chová doma slepice. Když odjede na dovolenou, stávám se já tím, kdo se o ně musí starat. Slepice je třeba dojet ráno a večer zkontrolovat a spočítat. Tato činnost je časově náročná kvůli cestování a kolikrát i zbytečná, protože většinou se nic neděje. A vzhledem k tomu, že jsem povoláním informatik, budu toto práci věnovat tomu jaké je moje řešení automatizace babiččina chovu kura domácího.
\newline
Práce se dělí na teoretickou a praktickou část. V teoretické části jsou popsány základní pojmy a principy, jež jsou důležité pro plné chápání práce. Praktická část popisuje konkrétní implementaci a nasazení asistenčního systému pro chov hospodářských zvýřat. Čtenář může využít znalosti, které nasbírá během čtení teoretické části a jež mu následně pomůže chápat praktická část, jako návod k realizaci vlastního systému dle jeho potřeb.
\newline
Svojí konkrétní implementaci jsem zaměřil na chov Kura domácího z výše popsaných osobních důvodů, a protože byl pro mě nejdostupnější testovací zvíře. Systém jsem realizoval mikroservisní architekturou a jednotlivé služby jsou psány v populárnímu jazyce Python. Jako GUI rozhraní je šikovně použit open source software pro chytrou domácnost Home Assistant.
\newline
Ukázkový běh aplikace je dostupný na url adrese https://coopmaster.jarousnemec.cz/ s přihlašovacími údaji do Home Assistanta uživatel: \textbf{visitor} a heslo: \textbf{Heslo1234}. Zdrojový kód k jednotlivým službám a konfiguraci Home Assistanta je dostupný na serveru github.com.
