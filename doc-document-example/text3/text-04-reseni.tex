%\externaldocument{text-02-teoreticka}
%
%
%\chapter{Praktická část}\label{ch:prakticka-cast}
%Tato část práce předvede čtenáři konkrétní realizaci myšlenky probírané v této prací.
%Zároveň v se v ní čtenář dozví jak v praxi využít poznatky nabyté v teoretické části.
%
%
%\section{Rozvržení person}\label{sec:rozvrzeni-person}
%Před tím než se započne tvorba jakékoli aplikace, je třeba určit, pro koho danou aplikaci tvoříme a jak bychom chtěli, aby jí tento uživatel používal.
%V našem případě jde o to, že systém by měl být schopný používat člověk znalý v chovu hospodářských zvířat, ale často méně zdatný v používání informačních technologií a mobilních aplikací.
%Na základě toho, že už víme, kdo je cílový uživatel můžeme se zamýšlet jakou funkcionalitu do řešení implementovat.
%Jedná se tedy o to, aby aplikace umožňovala uživateli vzdálený dohled na jeho chov a pomáhala mu šetřit čas zautomatizováním každodenních činnosti.
%Rozhodlo se tedy, že aplikace bude uživateli poskytovat následující funkce
%\begin{itemize}
%    \item Vzdálený přístup skrze internet odkudkoli
%    \item Pohledy z bezpečnostních kamer ve vnitřním i venkovním výběhu
%    \item Dálkové ovládání a automatizace světla a bezpečnostních dvířek v kurníku
%    \item Intuitivní vizualizace stavů jednotlivých hnízd(sedící slepice, v opačném případě počet vajec)
%    \item Automatické upozornění notifikacemi v telefonu na vetřelce ve výběhu
%    \item Vizualizace aktuálních povětrnostních podmínek v kurníku (teplota a vlhkost)
%\end{itemize}
%
%\section{Návrh architektury a volba technologií}\label{sec:navrh-architektury-a-volba-technologii}
%Díky tomu, že si přesně určíme, kdo je cílový uživatel naší aplikace, a k tomu dáme do hromady, co uživatel od naší aplikace očekává.
%Jsme nyní schopni bez větších problémů teoreticky navrhnout architekturu našeho systému a přesně popsat požadovanou funkcionalitu, jakou budou disponovat jeho jednotlivé části.
%Celý ekosystém se skládá z několika samostatných celků
%\begin{itemize}
%    \item Fyzická zařízení jako kamery a senzory
%    \item Architektura systému obsahující hlavní funkcionalitu(sekce~\ref{sec:backend})
%    \item Uživatelské rozhraní
%    \item Komunikace mezi GUI a backendem systému
%    \item Zpřístupnění aplikace z internetu
%\end{itemize}
%
%\subsection{Architektura hlavního systému}\label{subsec:microservices}
%Jako teoretický model na základě kterého, budeme organizovat a třídit funkcionalitu, jsem zvolil mikroservisní architekturu(sekce~\ref{sec:microservice-architecture}).
%Tento způsob rozvržení zodpovědnosti jednotlivých částí systému jsem zvolil, kvůli velké možnosti rozšíření, zapouzdření funkcionality a snadné úpravě jednotlivých služeb bez nutnosti kompletního restartu systému případně znovunasazení.
%Jako programovací jazyk pro tvorbu služeb je zvolen programovací jazyk Python(sekce~\ref{sec:ipcamera-rtsp}).
%Python byl vybrán kvůli jeho univerzálnosti, snadné syntaxi a podpoře ze strany vývojářů a komunity.
%Pro realizaci, konfiguraci a síťování mikroservistní architektury je použit Docker Engine(sekce~\ref{sec:kontejnerizace}) společně s rozšířením Docker Compose.
%Na základě určeného způsobu zapouzdření funkcionality je architektura rozdělena do následujících služeb, které jsou pojmenovány dle jejich účelu
%\begin{itemize}
%    \item Camera driver
%    \item Scale driver
%    \item Room driver
%    \item Health checker
%    \item Room assistant
%    \item Nest watcher
%    \item Dog alarm
%    \item Chicken watch guard
%\end{itemize}
%
%\subsection{Uživatelské rozhraní}\label{subsec:gui}
%Aplikace musí mít rozhodně i grafické rozhraní.
%Pro tento účel byla zvolena open source aplikace Home Assistant(sekce~\ref{sec:home-assistant}).
%Home assistant byl zvolen kvůli jeho univerzálnosti, rozsáhlé podpoře, komunitě bohaté na custom řešení a obrovské možnosti konfigurace a přispůsobení, čehož pro ovládání a vizualizaci dat z našeho systému hojně využijeme.
%
%\subsection{Komunikace mezi Backendem a Frontendem}\label{subsec:komunikace-mezi-backendem-a-frontendem}
%Je třeba zajistit komunikaci mezi Home Assistantem a Službami.
%Tuto úlohu musíme přijmout velice zodpovědně a navrhnout řešení, které půjde opět snadno rozšířit a modifikovat.
%Nelze proto použít klasické HTTP(sekce~\ref{sec:http-rest}), z toho důvodu že bychom komunikaci vázali na buď doménové jméno a port nebo ip adresu a port.
%Toto řešení má problém v tom, že pokud bychom potřebovali změnit, buď umístění částí služeb nebo port jedné ze služeb, bylo by třeba překonfigurovat i home assistanta.
%Dalším problém nastane, když potřebujeme z některé ze služeb poslat například notifikaci do Home Assistanta, je pro to potřeba, aby jednotlivé služby věděli, kde na síti Home Assistant běží, což je věc, která se může měnit, a museli bychom naopak přenastavovat jednotlivé služby.
%Jako řešení se nabízí použít messaging konkrétně třeba technologii MQTT(sekce~\ref{sec:mqtt}).
%Tato technologie se běžně používá u IoT zařízení a pro naše použití bude vynikajícím řešením.
%
%\subsection{Hardwarová zařízení}\label{subsec:hardwarova-zarizeni}
%Tuto sekci je velmi důležité dobře a správně navrhnout, protože náš systém bude pracovat s daty z reálného světa.
%Tato data musejí získávat k tomu určená zařízení a na jejich přesnosti závisý efektivita a správnost chování zbytku systému.
%Musíme tedy navrhnout a realizovat několik zařízení, která společně budou plnit následující úlohy
%\begin{itemize}
%    \item snímání stavů hnízd
%    \item ovládatní dvířek
%    \item snímání teploty a vlhkosti
%    \item snímání scén v kurníku a ve výběhu
%\end{itemize}
%\subsubsection{Snímání stavů hnízd}
%Všechny případy, které potřebujeme v kontextu našeho hnízda řešit se dají rozpoznat na základě hmotnosti hnízda.
%Pokud je v hnízdě slepice, hnízdo bude hodně zatížené oproti počátečnímu prázdnému stavu.
%V případě, že bude hnízdo zatížené málo a nebo středně, může to pro nás ve většině případů spolehlivě znamenat, že se v hnízdě nacházejí vejce.
%Nestandartní situace jako neobviklý nepořánek v hnízdě případně výkal, řešit nebudeme, protože by to exponenciálně zvýšilo složitost řešení a nepřineslo by to o moc lepší data.
%Kurník ja navíc v našem případě pravidelně udržován, takže šance na výskyt chybového faktoru v podobě nežádoucí zátěže je nízká.
%Jako nejjednodužší prostředek pro splnění tohoto úkolu na základě předchozí analýzy se ukázala obyčejná digitální váha.
%Tu bude třeba vytvořit vlastní konstrukce vzhledem ke skutečnosti, že ji musíme zabudovat do již existujícího hnízda.
%Popis konkrétního zapojení a konstrukce je sepsán níže v sekci~\ref{subsec:digitalni-vaha-do-hnizda}.
%
%\subsubsection{Ovládatní dvířek, snímání teploty a vlhkosti}
%Pro ovládatní dvířek a snímání povětrnostních podmínek je nejrozumnější vytvořit řídící jednotku, která tyto požadavky obsáhne.
%Další částí částí jsou samotná elektrická dvířka, jejichž funkce bude na základě způsobu elektrického signálu být otevřená nebo zavřená.
%Teplota a vlhkost jsou snímány senzory, jež jsou souřástí řídící jednotky.
%Implementace je popsána v sekci~\ref{subsec:ridici-jednotka}.
%\subsubsection{Snímání scén v kurníku a ve výběhu}
%Ke snímání scén v daných oblastech je třeba vyprat zařízení s dostatečnou odolností proti povětrnostním vlivům.
%Speciálně v kurníku je hodně agresivní prostředí kvůli vysoké vlhkosti a pro dlouhodobý provoz je třeba vybrat kvalitní a odolnou techniku.
%Zařízení musí být připojeno přes ethernet a musí být napájené ze sítě nikoli pomocí baterie.
%Na základě předchozí analýzy je vhodné použít venkovní ip kameru.
%Ip kamera s dostatečným krytím, dobrou kvalitou obrazu a velkým zorným polem bude snadná na instalaci, je cenově dostupná a poskytuje dostatečně kvalitní data pro naše použití při detekci objektů.
%Podrobnosti ohledně výběru kamery se nachází v sekci~\ref{subsec:kamerovy-system}
%
%\subsection{Zpřístupnění aplikace z internetu}\label{subsec:zpristupneni-aplikace-z-internetu}
%Aby systém mohl sloužit svému účelu musí být vzdáleně přístupný.
%To znamená, že server, na kterém systém poběží, musí mít veřejnou ip adresu a v ideálním případě mít tuto adresu propojenou s doménovým názvem pro snadnější použití.
%Tato problematika se dá řešit několika způsoby.\newline
%Jeden ze způsobů je zařízení si přímo veřejné ip adresy pro svůj server.
%Toto řešení je, technicky i teoreticky náročné, protože v případě, se kterým pracujeme my, běží server na lokální síti a veřejná ip adresa tedy není přímo pro náš server, ale celou síť.
%To vytváří v určitých situacích vytváří velmi vážná bezpečnostní rizika, díky čemuž je tato metoda velice náročná na hardware a implementaci.
%Navíc poskytování veřejné ip adresy je placená služba poskytovatel internetového připojení.\newline
%Další způsob je nasazení části systému do některého z cloudových řešení jako je Microsoft Azure nebo Amazon AWS.
%Tato metody by v podstatě vyhovovala našim požadavkům, ale pokud bychom se bavili o ceně, tak to rozhodně není levné řešení. \newline
%Metoda, kterou využíváme v řešení my, je snadná, jednodužší na implementaci než zřizování veřejné ip adresy a levnější nez využití cloudové služba.
%Tento způsob využívá takzvané tunely neboli šifrovaná spojení.
%Funguje na principu, při němž se zařízení šifrovaně propojíte s veřejným serverem a tento server bude skrze sebe vystavovat službu, která běží na lokálním počítači například u nás doma.
%Výhoda je, že vzdálený server disponuje kvalitním a odborně nastaveným firewalem, ktery zajistí dobrou ochranu našeho lokálního serveru.
%Tuto službu poskytuje zdarma a bez omezení například společnost Cloudflare.
%Tento způsob nejvíce odpovídá našim požadavkům, které byli cena a jednoduchost společně s rychlost nasazení.\newline
%Poslední nutnostní, kterou ovšem je třeba zaplatit, je pronájem vlastního doménového jména.
%Doménu si lze pronajmout u doménových registrátorů jako jsou GoDaddy, Hostinger nebo například Forpsi.
%Ceny domén se odvíjí na základě jejího druhu a pohybují od 20 do 1500 korun za kus.
%My v našem řešení využíváme službu Forpsi, protože s ní máme již předchozí zkušenost.\newline
%Dále je tato část rozvedena v sekci~\ref{sec:nasazeni-do-kurniku}, kde je popsán konkrétní způsob nasazení do funkčního projektu.
%
%\section{Popis jednotlivých modulů}\label{sec:popis-jednotlivych-modulu}
%
%\externaldocument{text-02-teoreticka}

\subsection{Camera driver}\label{subsec:camera-driver}
Camera driver je služba zodpovědná za sprostředkování komunikace mezi ip kamerami a službami zpracovánajícími obraz (Dog alarm, Chicken watch guard).\newline

\subsubsection{Funkcionalita}
Po startu služby naběhne RESTové api na portu 9001 vytvořené za pomoci pythonového frameworku Flask.
Api poskytuje endpoint pro získání aktuálního snímku z každé kamery, která je službě přiřazena v konfiguraci.
S kamerami služba komunikuje pomocí HTTP protokolu.
Byla zvažována i možnost využití protokolu RTSP, ale protože nemáme prozatím žádne vysoké nároky na živý přenos a stačí nám pouze jednou za čas aktuální snímek, ukazala se proto lepší volba a to využití REST api kamery pro stažení obrázku.
Jakmile je přijat požadavek na získání aktuálního snímku z konkrétní kamery, služba za pomocí get requestu stáhne obraz z restového api síťové kamery a vrátí ho ve Full HD rozlišení jako odpověď na http dotaz.
Url, z kterého má služba obrázek stahovat, získá z konfigurace předávané pomocí systémových promenných.
Kamery mají vlastní zabezpečení, proti neoprávněnému přístupu k datům.
V případě ověřování HTTP komunikace se používá metoda Digest Access Authentication, pomocí které je kamerě předáváno uživatelské jméno a heslo.

\subsubsection{Technologie}
\begin{itemize}
    \item Jazyk Python a jeho knihovny
    \item REST api
    \item HTTP protokol
\end{itemize}

\subsubsection{Použíté knihovny}
\begin{itemize}
    \item Flask: Lehký webový framework pro rychlý vývoj webových aplikací.
    \item colorama: Manipulace s barvami v textovém výstupu na terminálu.
    \item waitress: Rychlý WSGI server pro produkční nasazení webových aplikací.
    \item python-dotenv: Načítání konfigurace z .env souborů.
    \item Requests: Jednoduché HTTP požadavky (GET, POST, atd.).
\end{itemize}

%# coopmaster-camera-driver
%- komunikator mezi IP camerou a obraz zpracovavajici aplikaci (dog alarm, chicken watch guard)
%
%## funkcionalita
%- nese si informaci o konfiguraci jednotlivych IP kamerach aktuálně je zde pevně zahardcoděne používání dvou kamer 1. pro dog alarm a 2. pro chicken watch guard
%- vystavuje REST API pro ziskani aktualniho obrazu z chicken kamery a kamery v ohrade
%- vraci obrazek ve FULL HD rozliseni
%- pro každý call se načte nový obrázek z kamery
%
%## technologie
%- python
%- rtsp - real-time streaming protocol
%- knihovny
%
%
%## hardware
%- IP camera Hilook by Hikvision IPC-B180HA-LU
%
%\externaldocument{text-02-teoreticka}

\subsection{Scale driver}\label{subsec:scale-driver}
Scale driver zodpovídá za komunikaci mezi fyzickou váhou a službami, které využívají data o vážení.\newline
Protože jako řídící jednotka váhy je použito Arduino(\ref{sec:arduino}) tento modul komunikuje přes serialový port pomocí protokolu USB a na dotaz přijmutý RESTovým api poskytne jako odpověď hodnotu načtenou z váhy.

\subsubsection{Funkcionalita}
Nejprve naběhne hlavní vlákno, které se stará o poskytování RESTového api na portu 9004.
Toto api obsahuje endpoint pro vrázení aktuální hmotnosti na váze pomocí požadavku s metodou get.
Následně je spuštěno nové vlákno, které má za úkol číst data z váhy a předávat je aplikačnímu rozhraní jako aktuálně naměřenou hmotnost na váze, jež se bude od teď poskytovat, pokud o ní někdo GET requestem požádá.
Čtení informací z váhy se prování v nekonečném cyklu.
Před začátkem cyklu se zaloguje že je nové vlákno úspěšně spuštěno a následně proběhne otevření komunikace s váhou přes seriový port s využitím knihovny PySerial.
Po úspěšném otervření spojení se váze vždy po uplinutí časového intervalu 2 sekundy, odešle znak w.
Tento znak je ve firmwaru váhy vedeno jako příkaz, při jehož přijetí má váha vrátit aktuální naměřenou hodnotu.
Časové spoždění je implementováno kvůli tomu, že není třeba číst data tak často a zároveň by mohlo docházet k zahlcení váhy.
Jakmile je celá odpověď od váhy přijata programem zpět, jsou data zvalidována a následně, pokud je to možné, převedena na datový typ Int.
Datový typ Int je zvolen z důvodu, že váha posílá data v gramech a je zbytečné v tomto případě používat desetinná čísla, protože gramy poskytují dostatečné rozlišení pro naše účeli.
Pokud se nepodaří převézd přečtenou hodnotu na číslo, je výtupní promněnná nastavena na -1, což zbytku systému značí, že váha není v pořádku.
Na závěr se zalogují aktuálně získaná data a následuje další volání cyklu.

\subsubsection{Technologie}
\begin{itemize}
    \item Jazyk Python a jeho knihovny
    \item REST api
    \item HTTP protokol
    \item Serialový port
\end{itemize}

\subsubsection{Použíté knihovny}
\begin{itemize}
    \item Flask: Lehký webový framework pro rychlý vývoj webových aplikací.
    \item colorama: Manipulace s barvami v textovém výstupu na terminálu.
    \item waitress: Rychlý WSGI server pro produkční nasazení webových aplikací.
    \item pyserial: Komunikace se sériovými zařízeními přes sériovéporty.
    \item python-dotenv: Načítání konfigurace z .env souborů.
\end{itemize}


%
%\externaldocument{text-205-teoreticke-pojmy}

\subsection{Room driver}\label{subsec:room-driver}
Room driver zařizuje komunikaci mezi službou Room Assistant a řídící jednotkou v kurníku.
Tato jednotka se stará o ovládání dveří, světla a poskytování dat o aktuální teplotě a vlhkosti.
Ovládání nízkoúrovňových komponent a komunikaci s připojeným počítačem zajišťuje Arduino Nano.
Arduino je připojeno pomocí usb sběrnice k počítači, který na základě domluvených pravidel jednoduchého komunikačního protokolu posílá znaky, na než Arduino příslušně reaguje dle programu.
Tato služba má tedy za úkol přes serialový port pomocí USB rozhraní posílat příkazy a načítat stavy řídící jednotky na základě requestů přijmutých na REST api.

\subsubsection{Funkcionalita}
Po startu služby se nejdříve vytvoří instance třídy ArduinoReader.
Jedná se o vlastní třídu, při inicializaci vytvoří instanci serialového spojení k arduinu a to drží po celou dobu běhu aplikace.
Třída má jedinou metodu a tou je run\_command.
Tato metoda přijímá jako parametr jeden znak, který při provolání metody pošle zkrze serialové připojení do arduina a vrátí data typu String, kterými odpoví arduino, zpět.
Jakmile se povede provést inicializaci serialového spojení je spuštěno REST api.
Toto api poskytuje endpointy pro vyslání příkazů k zavření a otevření dvíře, zhasnutí a rozsvícení světla, vracení informací o aktuálních stavech jednotky a vrácení informací o teplotě a vlhkosti.
Informace o aktuálním stavu jsou dobré pro případ, kdy se budou muset služby restartovat a bude třeba načíst reálný aktuální stav.
V případě, kdy přijde požadavek na konkrétní endpoint, jako první věc se provede odeslání příslušného znaku jako příkaz arduinu.
Následně se počká na jeho odpověď, pokud je požadavek typu POST.
Odpověď je následně převedena do JSON objektů a vrácena v HTTP odpovědi s typem těla odpovědi JSON.

\subsubsection{Příkazy ovládacího protokolu}
\begin{itemize}
    \item o = otevřít dvířka
    \item c = zavřít dvířka
    \item l = rozsvítit světlo
    \item d = zhasnout světlo
    \item j = vrátí json s daty o teplotě a vlhkosti
    \item s = vrátí json s daty o stavech jednotlivých ovládaných prvku (dvěře, světlo)
\end{itemize}

\subsubsection{Technologie}
\begin{itemize}
    \item Jazyk Python a jeho knihovny
    \item REST api
    \item HTTP protokol
    \item Serialový port
\end{itemize}

\subsubsection{Použíté knihovny}
\begin{itemize}
    \item Flask: Lehký webový framework pro rychlý vývoj webových aplikací.
    \item colorama: Manipulace s barvami v textovém výstupu na terminálu.
    \item waitress: Rychlý WSGI server pro produkční nasazení webových aplikací.
    \item pyserial: Komunikace se sériovými zařízeními přes sériovéporty.
    \item python-dotenv: Načítání konfigurace z .env souborů.
\end{itemize}


%# coopmaster-room-driver
%
%Aplikace provazuje core aplikace(konkrétně službu room assistant) s arduinem ovladajicim svetla, dvere a poskytujici informaci o teplote a vzdusne vlhkosti.
%Součástí projektu je i firmware arduina.

%## funkcionalita
%- umoznuje ovladat osvetleni kurniku
%- resi otevírání a zavírání dvířek v kurníku
%- ziskava informace ze senzoru o teplote a vlhkosti v kurniku
%- komunikuje se zbytkem systému pomocí rest api
%- s arduinem komunikuje pomocí serialového portu a pomocí zasílání příkazů v podobě jednoho písmene
%- příkazy: o = otevřít dvířka, c = zavřít dvířka, l = rozsvítit světlo, d = zhasnout světlo, j = vrátí json s daty o teplotě a vlhkosti, s = vrátí json s daty o stavech jednotlivých ovládaných prvku (dvěře, světlo)
%
%## technologie
%- python
%- knihovny pro python
%- **Flask**: Lehký webový framework, flexibilní a rychlý vývoj webových aplikací.
%- **colorama**: Manipulace s barvami v textovém výstupu na terminálu.
%- **waitress**: Rychlý WSGI server pro produkční nasazení webových aplikací.
%- **requests**: Jednoduché HTTP požadavky (GET, POST, atd.).
%- **Werkzeug**: WSGI nástroje pro webové aplikace (routování, správa relací).
%- **python-dotenv**: Načítání konfigurace z `.env` souborů.
%- **pyserial**: Komunikace se sériovými zařízeními přes sériové porty.

%
%\externaldocument{text-205-teoreticke-pojmy}
\subsection{Health checker}\label{subsec:health-checker}
Health checker je malá služba určená pro správce systému.\newline
Poskytuje informace o tom zda všechny potřebné služby běží, aby správce nebyl nucen přihlašovat se vzdáleně na server a manuálně kontrolovat každou službu.
Výpis stavů jednotlivých služeb je poskytován RESTovým api [GET] /status
%\begin{itemize}
%    \item GET /status
%\end{itemize}
%
%\externaldocument{text-02-teoreticka}
\subsection{Room assistant}\label{subsec:room-assistant}
Room assistant je zpřístupňuje komunikaci mezi GUI tedy konkrétně Home Assistantem a Room driverem.\newline
S Home Assistantem je komunikace realizována pomocí MQTT(\ref{sec:mqtt}) a s Room driverem pomocí HTTP(\ref{sec:http-rest}) protokolu.
Základní funkcí je zpracování a přeposlání příkazů do Room Driveru odebíraných z témat
%\begin{itemize}
%    \item coopmaster/room/door/cmnd
%    \item coopmaster/room/lamp/cmnd
%\end{itemize}
Služba očekává, že na tato témata budou chodit zprávy open / close pro dveře a on / off jako příkazy pro světlo.
Další úlohou je periodické načítání a aktualizace informací o stavu dveří, světla a dat z teplotního a vlhkostního senzorů.
Tato data by měl Room assistant pravidelně načítat a posílat přes MQTT do Home Assistanta.
%\begin{itemize}
%    \item coopmaster/room/temperature
%    \item coopmaster/room/humidity
%    \item coopmaster/room/door/state
%    \item coopmaster/room/lamp/state
%\end{itemize}
%
%\externaldocument{text-205-teoreticke-pojmy}
\subsection{Nest watcher}\label{subsec:nest-watcher}
Tato služba interpretuje stavy jednotlivých hnízd v kurníku pro Home Assistanta.
Hlavní funkcí je načítání a analýza dat z jednotlivých vah v hnízdech, která jsou reprezentována Scale Drivery.\newline
Data jsou načítána několikrát do minuty a ukládána do databáze s časovým údajem, kdy byl záznam vytvořen.
Následně se jednou za minutu vyhodnotí průměrná hodnota během posledních několika vážení.
Na základě tohoto údaje jsme schopni zjisti několik případů
\begin{itemize}
    \item hnízdo je prázdné (hodnota na váze nepřevyšuje 50 g )
    \item v hnízdě se nacházejí vejce (hmotnost jednoho vejce je průměrně 50 g)
    \item v hnízdě sedí slepice (hmotnost slepice se pohybuje okolo 1200 g a více)
\end{itemize}
Pokud je na váze průměrně méně než 50 g, vzhledem k možným chybám měření, takový případ vyhodnotíme jako, že je hnízdo prázdné.
Jestliže se hodnota pohybuje mezi 50 a 1200 gramy, znamená to, že v hnízdě jsou pravděpodobně vejce, a jejich počet je vypočítán vydělením celkové hmotnosti a hmotnosti jednoho vejce.
V případě, že je na váze více jak 1200 g, vyhodnotí služba, že v hnízdě sedí slepice.
Tyto tři zmíněné informace služba následně pomocí MQTT předává do Home Assistanta.
%
%\externaldocument{text-02-teoreticka}
\subsection{Dog alarm}\label{subsec:dog-alarm}
Služba Dog alarm má detekovat nebezpečí ve výběhu a poslat tuto zprávu do Home Assistanta.\newline
Aktuální záběry jsou pomocí GET requestů stahovány z konkrétní instance služby Camera driver, která je přiřazena ke kameře ve výběhu.
Analýza probíhá v určitých intervalech za pomocí umělé inteligence, kde je konkrétně použita metoda detekce objektů.
Jakmile jako výsledek klasifikace vyjde jednoznačně, že v záběru byl spatřen pes nebo jiný predátor, je zpráva poslána pomocí MQTT do Home Assistanta společně s konkrétním záběrem, na němž byl predátor detekován.
Tato služba zároveň přes MQTT posílá do Home Assistanta aktuální záběr z kamery.
%
%\externaldocument{text-02-teoreticka}
\subsection{Chicken watch guard}\label{subsec:chicken-watch-guard}
Úkolem služby Chicken watch guard je sledovat stav a počet slepic v kurníku.\newline
Aktuální záběry jsou stejně jako u Dog alarmu stahovány z konkrétního Camera driveru v kurníku.
Následně po získání záběru proběhne detekce objektů a počet těchto objektů udává počet detekovaných slepic v obraze.
Tato hodnota je publikována pomocí MQTT do Home Assistanta.
Vedlejší funkcí služby je průběžné posílání aktuálního pohledu z kamery v kurníku do Home Assistanta.
%
%\externaldocument{text-205-teoreticke-pojmy}
\subsection{GUI rozhraní}\label{subsec:tvorba-gui-rozhrani}
- vybral se teda ten home assistant a ted ho nakonfit\newline
- konfiguruje se to přes yaml configuration.yaml což je hlavní konfigurák HA\newline
- byla výzva přijít na to jak se přidávají vlastní mqtt senzory viz seznam závad v trellu\newline
- po přidání senzorů jsem si hrál s vizualizací jednotlivých hnízd tak aby to pro uživatele bylo přívětivě\newline
- napsal jsem proto vlastní komponentu pro HA viz foto a trello\newline
- zajímavé zjištění bylo že state který jsem využíval pro předávání textových zpráv má maximální velikost 255 bytů\newline
- takže jsem z jsonu přešel na csv\newline
- pokecat trochu o tom jak vytvořit takovou komponentu jaké to má části a předpoklady a jak se to následně přidává a konfiguruje v HA\newline
- následně pak konfigurace automations pro dveře, přepínačů pro manuální ovladání světla a dvěří, obrázků z kamer které taky chodí pomocí mqtt, a následně ještě dog alert aby se poslalo upozornění pokud je mobilní apka a na dashboardu vyskočil daný obrázek a výstrahou\newline
- závěrem pak výpis teploty, vlhkosti a počtu slepic v kurníku které systém poznal\newline
%
%\externaldocument{text-02-teoreticka}

\subsection{Digitální váha do hnízda}\label{subsec:digitalni-vaha-do-hnizda}
Pro účely projektu bylo potřeba postavit digitální váhu vlastní konstrukce.
Tato váha má být umístěna v každém hnízdě v kurníku, kde slepice snáší vejce, a jejím úkolem je přes usb poskytovat aktuální data o tom jakou hmotností je zatěžována podložka hnízda.

\subsubsection{Funkcionalita}
Po připojení napájení k arduinu, se nejdříve nastaví a připojí serialová komunikace rychlostí 9600 baudů pomocí usb.
Následně je instancován objekt starající se o sprostředkování komunikace s převodníkem HX711 a nastaví se jednotlivé piny dle fyzického zapojení převodníku.
Dále je nastaven kalibrační faktor, což je konstanta používaná knihovnou HX711.h při převodu dat získaných z převodníku na gramy, s nimiž program již nadále pracuje a poskytuje je po serialovém připojení.
Jako poslední krok v rámci přípravy programu je vynulování váhy pomocí metody tare(), která je jedna z metod HX711.h knihovny.
Jakmile program dokončí fázi příprav vstoupí do nekonečného cyklu.
Tento cyklus pravidelně čte data přicházející po serialovém připojení.
V případě, kdy jsou přijmuta data, program počítá se třemi scénáři.
Buď je přijat znak w, znak t nebo jakákoli jiná data.
V případě znaku w (od slova weight) program reaguje odesláním aktuální naměřené hodnoty váhy po seriálovém připojení jako reakce na příkaz.
Přijmutý znak t znamená tare neboli vynulovat váhu.
Na základě toho se programu řekne, že aktuální data přijímaná ze senzoru má brát jako hmotnost 0 g.


\subsubsection{Konstrukce váhy}
Jako řídící jednotka váhy je použito Arduino Nano, jež je zodpovědné za komunikaci s počítačem pomocí serialového připojení a načítání dat o hmotnosti z AD převodníku HX711 přes, který je arduino připojeno k tenzometrickému senzoru.
Hmotnost je měřena tenzometrickým senzorem se jmenovitým zatížením 20 kg.
Toto zapojení je nutné vzhledem ke konstrukci a vlastnostem senzoru.
Tenzometrické senzory jsou analogové a jejich výstupem jsou velmi malé změny v napětí v rozmězí 0.15 až 1 mV.
Tyto signály jsou navíc hodně slabé, proto je nutné použít konkrétně převodník HX711, který signál zesílí a zajistí větší rozlišení díky vysoké citlivosti.
Složitější zapojení není jedinou nevýhodou.
Značnou nevýhodu představuje také fakt, že už vlivem zatížení středního zatížení se profil senzoru mírně deformuje a na základě toho narůstá chyba měření v průběhu času.
Tento problém se dá snadno řešit pravidelným nulováním v případech, kdy zrovna senzor zatížen není.
Na konstrukci váhy je velmi zajímavý tvar a důvody samotné železné konstrukce.
Protože má být váha instalována do jiz existujícího hnízda a slepice bude vlastně sedět rovnou na váze, je třeba navrhnout konstrukci váhy tak, aby slepice se funkce a způsob sezení nezměnil.
Nakonec je jako platforma, na které slepice sedí, zvolena čtvercová miska o rozměrech odpovídajících s rezervou rozměrům celeho hnízda.
Nemohla by to být například pouze rovná deska, protože slepice potřebuje pro pohodlné sezení podestýlku kterou tvoří seno a v případě desky by se seno dostalo mezi stěny hnízda a desku a váha by se tak nemohla volně pohybovat a zasekávala by se.
Takto v misce je mnohem menší šance, že bude nějak větší množství sena vytlačeno a bude přidírat váhu.

\subsubsection{Kalibrace}
Po fyzickém sestavení váhy je třeba provést kalibraci.
Kalibrace námi používaných senzorů je obecně nutná, protože se díky ní eliminují rozdíli mezi senzory, překryjí se nedostatky konstukce váhy a zajistí přesnost a korektnost měření pro konkrétní senzor.
To znamená najít vhodnou hodnotu kalibračního faktoru, která se již za běhu váhy nebude měnit.
Doporučená metoda jak provádět kalibraci je vzít si předmět o němž víme přesně jeho hmotnost a ten použít jako vzorové závaří, na které chceme váhu zkalibrovat.
Konkrétně u námi použitého senzoru do 20 kg je dobré volit závaží v řádech minimálně několika kilo.
Já doma použil 5 pitlů polohrubé mouky jejichž hmotnost jsem si ověřil přesnou kuchyňskou váhou.
Následujícím krokem je naleznutí odpovídající hodnoty kalibračního faktoru, čehož je docilováno buď zvyšováním nebo snižováním kalibračního faktoru na základě toho zda výsledná hmotnost dle výpočtů s využitím aktuálního faktoru je moc nízká nebo moc vysoká.


\subsubsection{Hardware}
\begin{itemize}
    \item Arduino Nano v3.0
    \item AD převodník HX711
    \item Tenzometrický senzor se jmenovitým zatížením do 20 kg
    \item Železná konstrukce váhy
\end{itemize}

\subsubsection{Arduino knihovny}
\begin{itemize}
    \item Arduino.h - defaultní arduino knihovna
    \item HX711.h - pro usnadnění komunikace s AD převodníkem HX711
\end{itemize}

%- vysvětlit přijimane commandy w - dej vahu a t - tare
%zastrcim arduino do elektriky
%rozběhne se loop
%z pinů xy čteme hodnoty
%knihovna zpracovava hodnoty odporu
%knihovna řeší vynulování váhy
%vysvětlit kalibraci
%najít a popsat kalibrační program a vysvětlit k čemu je potřeba
%loop čeka a kontroluje serialovy vstup zda neni pismenko w nebo t
%popis loopu

%je tam převodník pro převod analogu na digital
%odecita odpor
%na konci toho je tenzometrikej můstek
%specifikace senzoru
%plusy a mínusy senzoru deformace, vysvětlit princip

%tenzometricky sezor podobne jako u vcel
%hezky popsano v praci
%uvaha o tom co vybrat zavesna vaha nebo tlacna vaha
%problemy pri implementaci
%problémy s konstrukcí a nekvalitními součástkami
%senzory se deformují a tím vzniká chyba v měření
%chyby vznikají díky analogu i chybami / odpory v elektrickém
%vedení (pevné / pájene spoje vs odnímatelné konektrory)

%knihovny arduina
%defaultní Arduino.h
%HX711.h: knihovna pro komunikaci s module AD Převodníku 24-bit 2 kanály HX711
%
%\externaldocument{text-02-teoreticka}
\subsection{Řídící jednotka}\label{subsec:ridici-jednotka}
Řídící jednotka je vlastní sestavené zařízení, které zprostředkovává pomocí USB připojení data o teplotě a vlhkosti v kurníku.
Další úlohou je kontrola a ovládání dvířek a světla.
Srdem řídící jednotky je Arduino Nano.
Tento programovatelný mikrokontroler slouží jako logický prvek mezi nízkoúrovňovími prvky(relé, senzory) a počítačem, na němž běží služby ovládají nebo využívající data z těchto iprvky.

%## funkcionalita
%- umoznuje ovladat osvetleni kurniku
%- resi otevírání a zavírání dvířek v kurníku
%- ziskava informace ze senzoru o teplote a vlhkosti v kurniku
%- komunikuje se zbytkem systému pomocí rest api
%- s arduinem komunikuje pomocí serialového portu a pomocí zasílání příkazů v podobě jednoho písmene
%- příkazy: o = otevřít dvířka, c = zavřít dvířka, l = rozsvítit světlo, d = zhasnout světlo, j = vrátí json s daty o teplotě a vlhkosti, s = vrátí json s daty o stavech jednotlivých ovládaných prvku (dvěře, světlo)
%
%## technologie
%- c++ / wire (jazyk pro programování v arduino ide)
%- knihovny pro arduino
%- defaultní **Arduino.h**
%- **DHT**: pro ovládání teplotních+vlhkostních sezorů DHT11 a DHT22
%- táhlový motor 12v nevím kolik neutonů
%- zpřevodovaný motorek k navijáku
%- relé 5v 10A
%
%
%## hardware
%- arduino Nano v3.0
%- teplotni a vlhkosti senzor DHT11
%- samotana konstrukce dveri
%- relatko ovladajici svetlo a relátka udávající chod a směr motoru
%- napájeno připojením ke raspberry pi přes usb napětím 5V
%
%\externaldocument{text-205-teoreticke-pojmy}
\subsection{Kamerový systém}\label{subsec:kamerovy-system}
Kamerový systém je jeden z hlavních prvků systému Coopmaster.
Data z jeho kamer využívají skrze službu Camera driver služby Dog Alarm a Chicken Watch Guard.
Do systému jsou aktuálně instalovány dvě kamery jedna pro venkovní výběh druhá pro pohled do kurníku.
Po pečlivém výběru a volbě parametrů jsou vybrány kamery značky Hikvision typ Hilook IPC-B180HA-LU.

\subsubsection{Výběr kamery}
Klíčové vlastnosti pro výběr našich kamer byly
\begin{itemize}
    \item možnost venkovní instalace
    \item drátové připojení
    \item volitelně oba způsoby napájení buď POE nebo přímo ze sítě (záleží na způsobu instalace a umístění)
    \item kvalitní noční vidění
    \item velké zorné pole
    \item vysoká kvalita obrazu (dostatek dat pro umělou inteligenci)
\end{itemize}
Snažil jsem se vybrat kameru, bude relativně cenově dostupná a bude vyhovovat zmíněným parametrům.
Musel jsem se kvůli tomu dovzdělat v mnoha ohledech například detailně prozkoumat technologii PoE, digitálního videa, konstrukcí venkovních elektrických zařízení a mnoha dalších parametrů, které venkovní IP kamery mají.
S výsledným výběrem jsem velmi spokojený.
Kamera poskytuje kvalitní data pro vývoj tohoto systému a práce s ní není nijak komplikovaná jak je detailněji popsáno v sekci o službě Camera driver.

\subsubsection{IP camera Hilook by Hikvision IPC-B180HA-LU}
Klíčové parametry, které využíváme v projektu, jsou
\begin{itemize}
    \item Rozlišení 8 Mpx
    \item Přísvit jak bílou LED a IR s dosahem až 30 m
    \item Zajímavá technologie ColorVu pro snímání barev za horších světelných podmínek
    \item Technologie Motion Detection 2.0 pro osob či vozidel
    \item Technologie WDR pro zmenšení rozdílů při různě silném osvětlení scény
    \item Napájení pomocí PoE nebo přes síťový adaptér na 12V 2A.
    \item Buď REST api a HTTP nebo RSTP protokol pro poskytování obrázků, streamování videa a zvuku
    \item Dobře zpracované webové rozhraní pro konfiguraci
\end{itemize}

%- vraci obrázek ve FULL HD rozliseni
%- pro každý call se načte nový obrázek z kamery
%
%## technologie
%- python
%- rtsp - real-time streaming protocol
%- knihovny
%
%
%## hardware
%- IP camera Hilook by Hikvision IPC-B180HA-LU
%
%\externaldocument{text-205-teoreticke-pojmy}
\subsection{Dvířka}\label{subsec:dvirka}
konstrukce dveri - expoziční \newline
- přímočarý motor s dorazy\newline
- konstrukce z hliníkových profilů a smontnotváno pomocí běžných spojovacích prvků pro práci s těmito profily\newline
- dvířka reprezentována hliníkovým plátem tahaným lankem připěvněným k ramenu který vynáší sílu od motoru k dvířkům\newline
- zdroj 12V \newline
konstrukce dveri - instalováno\newline
- zdroj 12V\newline
- rám z železných U profilů použitých jako vodící lišty pro posuv dvířek z plastové desky a pásků které konstrukci udžují po kupě\newline
- naviják je s dvířky přopojen lankem\newline
- navijecí systém / systém navijáku\newline
- 12V motor s převodovkou, rychlostí otáčení 60rpm\newline
- navíjecí buben vytištěný na 3d tiskárně\newline
- dorazový systém\newline
%
%\subsection{---poznámky TODO ke smazání---}\label{subsec:---poznamky-todo-ke-smazani---}
%- budeme potřebovat váhu pro kontrolu hnízd zda tam slepice je a nebo kolik je tam vajec\newline
%- budeme potřebovat ideálně ip kamery s vhodným IP krytím abychom mohly monitorovat kurník vevnitř a venku\newline
%- budeme potřebovat ovladačku asi arduino pro dveře, světlo, senzory teploty a vlhkosti\newline
%- předchozí věci je potřeba dostat na síť takže nějaké rpi pro předávání komunikace a směrování\newline
%- a všechno to musí řídit něco s dostatečným výkonem pro klasifikace třeba my tu máme nucka s RTX2080\newline
%- uživatelské rozhraní se rozhodlo že je zbytečné vyrábět vlastní a ztrácet tím čas lepší bude použít HA který disponuje všemi funkcemi je open source a má komunitu která ho udržuje
%- první řešení ale bude na stole takže se nemusíme zaobírat detaily kontkrétní instalace, alespoň pro zatím
%- implementace probíhala v jazyce python za využítí vypsaných knihoven viz readmečka modulů\newline
%- jak se konfiguroval logger, flask blueprinty, jak propojit python a arduino, jak na to s cronem/schedulerem, jak posílat mqtt a přijímat(jaká je struktura našich topiců, jak se to pojmenovává)
%- verzuje se to na github
%- na githubu běží workflow které vytváří jednolivé docker image pro každý modul a uploaduje je na můj docker hub kvůli snadnému deployi a distribuci po internetu
%
%\section{První pracovní zapojení a běh}\label{sec:prvni-pracovni-zapojeni-a-beh}
%- zjistilo se že bez poe je to špatná volba \newline
%- kamery žerou dost proudu\newline
%- bylo potřeba odladit nastavení kamer a jejich statické ip adresy, kamery měli zabezpečení na blokování ip address a tak\newline
%- byl problém s kontakty na váze takže se nakonec museli vyměnit lisované za pájené\newline
%- na stole to většinou funguje dobře \newline
%
%
%\section{Nasazení do kurníku}\label{sec:nasazeni-do-kurniku}
%- bylo potřeba natahat elektřinu a internet \newline
%- elektřina nebyla problém vzala se z kůlny zkrz díru ve zdi\newline
%- horší to bylo s netem protože wifi signál do kurníku nedosáhne\newline
%- vyřešilo se to wifi extenderem od tplinku který má zárověň i ethernet výstup díky čemuž nemuselisme dlouhého tahati kabelu a šlo nám to vzduhem přes dvůr a ve stodole pak drátem\newline
%- kamery bylo potřeba napájet a taky připojit přes internet; zprvu jsem si myslel že poe nebude potřeba ale taha 230 by bylo zbytečné námahy takže jsem pořídil poe switch od tplinku a ten připojuje kamery\newline
%- bylo někde potřeba udělat místo kam se nainstaluje celá technologie kurníku\newline
%- zvolila se plechová bedna ze starého domovního rozvaděče do níž se pro jednotlivé prvky vytvožili na míru držáčky a pouzdra ps.: fotky z tisku a pak z rozvaděče\newline
%- jak se zrealizovala váha\newline
%- jak vypadá řídící jednotka pri room assistanta\newline
%- jaké tam jsou dveře pro slepice\newline
%- celé je to propojené s rpi 5 které to v kurníku řídí a s ním pak komunikuje nuc pomocí tail scailu ale to zas v další kapitole\newline
%- bylo potřeba zkalibrovat hmotnosti slepic a vajec
%
%
%\section{Konfigurace vzdáleného přístupu}\label{sec:konfigurace-vzdaleneho-pristupu}
%- rpi propojene s nuckem a dev kompama přes tailscail\newline
%- v docker compose na nuckoj se vyrobila nová network jenom pro home assistanta a jeho propagaci na venek\newline
%- do vzniklé sítě se přidal ještě kontejner Cloudflared který zajišťuje tunel ven na cloudflare a přes jeho firewall a proxyny do internetu\newline
%- cloudflare tunel bylo potřeba namapovat na doménu kterou vlastníme\newline
%- pronajmul jsem si doménu u doméhového registrátora forpsi\newline
%- zmínění jaká plynou nebezpečí z tohoto řešení
%
%
%\section{Rozvržení práce do budoucna}\label{sec:rozvrzeni-prace-do-budoucna}
%- jak by se řešení dalo zoptimalizovat\newline
%- vylepšení modelu pro classifykaci\newline
%- vylepšení a zpřesnění vah\newline
%- kontrola napajedla\newline
%- kontrola krmítka
%- implementace autonomního chování do Room Assistanta
%
%
%\section{Ekonomická stránka projektu}\label{sec:ekonomicka-stranka-projektu}
%
