\externaldocument{text-205-teoreticke-pojmy}
\subsection{Room assistant}\label{subsec:room-assistant}
Room assistant je zpřístupňuje komunikaci mezi GUI tedy konkrétně Home Assistantem a Room driverem.\newline
S Home Assistantem je komunikace realizována pomocí MQTT(\ref{sec:mqtt}) a s Room driverem pomocí HTTP(\ref{sec:http-rest}) protokolu.
Základní funkcí je zpracování a přeposlání příkazů do Room Driveru odebíraných z témat
%\begin{itemize}
%    \item coopmaster/room/door/cmnd
%    \item coopmaster/room/lamp/cmnd
%\end{itemize}
Služba očekává, že na tato témata budou chodit zprávy open / close pro dveře a on / off jako příkazy pro světlo.
Další úlohou je periodické načítání a aktualizace informací o stavu dveří, světla a dat z teplotního a vlhkostního senzorů.
Tato data by měl Room assistant pravidelně načítat a posílat přes MQTT do Home Assistanta.
%\begin{itemize}
%    \item coopmaster/room/temperature
%    \item coopmaster/room/humidity
%    \item coopmaster/room/door/state
%    \item coopmaster/room/lamp/state
%\end{itemize}