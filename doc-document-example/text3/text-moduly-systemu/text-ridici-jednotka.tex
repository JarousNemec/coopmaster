\externaldocument{text-02-teoreticka}
\subsection{Řídící jednotka}\label{subsec:ridici-jednotka}
Řídící jednotka je vlastní sestavené zařízení, které zprostředkovává pomocí USB připojení data o teplotě a vlhkosti v kurníku.
Další úlohou je kontrola a ovládání dvířek a světla.
Srdem řídící jednotky je Arduino Nano.
Tento programovatelný mikrokontroler slouží jako logický prvek mezi nízkoúrovňovími prvky(relé, senzory) a počítačem, na němž běží služby ovládají nebo využívající data z těchto iprvky.

%## funkcionalita
%- umoznuje ovladat osvetleni kurniku
%- resi otevírání a zavírání dvířek v kurníku
%- ziskava informace ze senzoru o teplote a vlhkosti v kurniku
%- komunikuje se zbytkem systému pomocí rest api
%- s arduinem komunikuje pomocí serialového portu a pomocí zasílání příkazů v podobě jednoho písmene
%- příkazy: o = otevřít dvířka, c = zavřít dvířka, l = rozsvítit světlo, d = zhasnout světlo, j = vrátí json s daty o teplotě a vlhkosti, s = vrátí json s daty o stavech jednotlivých ovládaných prvku (dvěře, světlo)
%
%## technologie
%- c++ / wire (jazyk pro programování v arduino ide)
%- knihovny pro arduino
%- defaultní **Arduino.h**
%- **DHT**: pro ovládání teplotních+vlhkostních sezorů DHT11 a DHT22
%- táhlový motor 12v nevím kolik neutonů
%- zpřevodovaný motorek k navijáku
%- relé 5v 10A
%
%
%## hardware
%- arduino Nano v3.0
%- teplotni a vlhkosti senzor DHT11
%- samotana konstrukce dveri
%- relatko ovladajici svetlo a relátka udávající chod a směr motoru
%- napájeno připojením ke raspberry pi přes usb napětím 5V