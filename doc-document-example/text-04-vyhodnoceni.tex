\newpage
\chapter{Vyhodnocení}\label{ch:vyhodnoceni}

\addcontentsline{toc}{section}{Závěr}
Závěrem lze říct, že jsme úspěšně vyvinuli systém, který je schopný v reálném čase pomáha plnit každodení úkony, které by hospodář musel jinak dělat sám. Díky naší aplikaci může člověk kontrolovat situaci v kurníku případně i ve výběhu. Systém ho sám informuje, pokud je něco v nepořádku, například v případě, že je ve výběhu vetřelec, který by mohl představovat nebezpečí pro chované slepice. Dále se nám na podobném principu povedlo implementovat kontrolu počtu slepic v kurníku, což lze použít například při automatizaci zavírání dvířek. Systém také poskytuje aktuální záběry z bezpečnostních kamer díky, kterým si může hospodář sám kontrolovat situaci svého chovu a nemusí díky tomu být přítomen fyzicky. V poslední řadě jsme také implementovali část systému, která je schopna kontrolovat stavy jednotlivých hnízd a zjistit zda v něm sedí slepice případně kolik vajec v hnízdě je. Tato data lze také použít ke tvorbě statistik, díky nimž může člověk analizovat různé aspekty svého chovu.


\addcontentsline{toc}{section}{Použitá Literatura}
\printbibliography[title=Použitá Literatura]

\addcontentsline{toc}{section}{Seznam obrázků}
\listoffigures

\addcontentsline{toc}{section}{Seznam tabulek}
\listoftables

\addcontentsline{toc}{section}{Seznam rovnic}
\listoflistedequation