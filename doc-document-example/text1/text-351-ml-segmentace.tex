\externaldocument{text-02-teoreticka}
\subsection{Reidentifikyce objektů pomocí strojového učení}\label{subsec:klasifikace-a-segmentace-objektu-pomoci-strojoveho-uceni}


%Nejlepsi a nejhorsi nosnice
%
%Image segmentation je technika v počítačovém vidění, která má za cíl rozdělit obrázek na několik segmentů nebo oblastí pro jednodušší analýzu. Cílem image segmentation je identifikovat a rozlišit různé objekty nebo části obrázku, což může být užitečné pro různé aplikace, jako je rozpoznávání objektů, detekce hranic, klasifikace a další analytické úlohy.
%V mém případě jsem chtěl využit tuto techniku k tomu, abych identifikoval která slepice snesla vajíčko abych tak získal statistiku o tom v jaké kondici slepice jsou a jak dobře snáší.
%Existuje několik možností jak slepici identifikovat, já jsem se rozhodl pro rozpoznání slepice z obrazu.
%
%Začal jsem krokem  Získání a Příprava Dat
%Shromaždil jsem dataset snímků slepic z kurníku, který pokrývá různé úhly pohledu, osvětlení a pozice. Dataset jsem anotoval. V tomto případě byla tvorba anotacniho souboru významě složitější, protože bylo třeba nejenom identifikovat, že slepice je na obraze, ale identifikovat jeji konkrérní tvar, případně jenom část těla. Znamenalo to tedy ručně projít jednotlivé snímky a myší vyklikat potřebné oblasti.
%
%
%obr segmentace labeling ´´´´´´´´´´´´´´´´´´´´´´
%
%1.	Obecná zásada říká, že platí přímá úměra mezi množstvím trénovacích dat a kvalitou modelu. Toto jsem mohl vyřešit větším počtem obrázku v testovacím datasetu nebo a to je cesta, kterou jsem zvolil já augmentací dat. Na obrázky se nechá aplikovat řada transformací, které přispěly k různorodosti dat.
%
%Rotace: Otáčení obrázků o náhodně vybrané úhly, aby se model stal invariantním vůči natočení objektů.
%Škálování: Změny velikosti obrázků buď zmenšením nebo zvětšením, aniž by se změnil poměr stran, což pomáhá modelu lépe zvládat objekty různých velikostí.
%Oříznutí (cropping): Vyříznutí náhodných částí obrázku. To pomáhá modelu naučit se, že objekty mohou být částečně oříznuty.
%Překlopení (flipping): Horizontální nebo vertikální zrcadlení obrázku pro zvýšení variability dat.
%Změna jasu, kontrastu a saturace: Úprava těchto parametrů dělá model odolnějším vůči různým světelným podmínkám.
%Přidání šumu: Přidání náhodného šumu do obrázků může pomoci modelu být méně citlivý na zašumění v datech.
%Posun (translation): Posunutí obrázku ve vodorovném nebo svislém směru, které pomáhá modelu rozpoznávat objekty při různých umístěních.
%Random Erasing: Náhodné vymazání malých částí obrázku, které může pomoci vylepšit model proti částečnému zakrývání objektů.
%MixUp: Směšování dvou obrázků a jejich odpovídajících anotací ke generování nových tréninkových vzorů.
%Mosaic Augmentation: Kombinování čtyř různých obrázků do jednoho, což rozšiřuje kontext a zvyšuje variabilitu scén.
%
%
%Následovalo trenování Segmentačního Modelu:
%Řešení jsem postavil na YOLOv11 z oficiálního repozitáře. Diky knihovnám z projektu Ultralytics je volání přímočaré. Na vstup jsem dal cestu k obrázkovému datasetu a anotačnímu souboru v COCO formátu. Další parametry jsem pro výuku ponechal defaultní. Počet epoch jsem ponechal na 50. Již od 40 epochy model dosahoval přijatelné úrovně přesnosti při rozpoznávání a segmentaci slepic.
%
%obrs označené slepice pomocí ai ´´´´´´´´´´´´´´´´´´´´´´´´´´´´´´
%
%
