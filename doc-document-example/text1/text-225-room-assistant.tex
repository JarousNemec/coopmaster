\section{Room assistant}\label{sec:room-assistant}
Room assistant zpřístupňuje komunikaci mezi systémem Home Assistant a službou Room Driver.\newline
S Home Assistantem je komunikace realizována pomocí MQTT(\ref{sec:mqtt}) a s Room driverem pomocí HTTP(\ref{sec:http-rest}) protokolu.
Tato služba zpřístupňuje do systému Home Assistant data o teplotě a vlhkosti a Home Assistant přes ní ovládá světlo a dvířka.

\subsection*{Funkcionalita}
Po odstartování služby se stane hned několik věcí.
Jako první naběhnou MQTT subscribery pro přijímání příkazů ze systému Home Assistant.
Jeden přijímá příkazy pro dívřka druhý pro světlo.
Vytvoří se vlastně instance dvou tříd LampTimeChecker a DoorTimeChecker.
Tyto instance mají každá svou instanci třídy NestMQTTClient, která zaobaluje základní funkcionalitu ohledně používání protokolu MQTT.\newline

\textbf{NestMQTTClient} zaobaluje metody pro publikování a odebírání zprávy, připojení a odpojení od MQTT brokeru, reakci na navázané spojení s MQTT brokerem a reakci na odebrání nové zprávy.\newline
\textbf{LampTimeChecker} odebírá téma coopmaster/room/lamp/cmnd a zpracovává zprávu pokud je jejím obsahem \textbf{on} nebo \textbf{off}.
Na základě přijatého obsahu zprávy se zavolá statická metoda call\_room\_driver\_command třídy driver\_client, aby odeslala příslušný příkaz do Room Driveru na endpoint, který je předáván jako parametr.

\textbf{DoorTimeChecker} funguje stejně jako NestMQTTClient, ale odebírá téma coopmaster/room/door/cmnd a reaguje pokud je obsah MQTT zprávy \textbf{open} nebo \textbf{close}.
.\newline
Po startu se rozběhne BackgroundScheduler, který má za úkol vždy po uplinutí 5 vteřin spustit nový běh jobu pro aktualizování stavů z řídící jednotky.
Podmínkou pro spuštění nového běhu jobu je, že předchozí běh musí být dokončen.
Tuto funkci má již v sobě implementovanou scheduler z knihovny APScheduler, který využíváme.
Job pro aktualizaci obsahuje jednu metodu a to detect\_hardware\_state.
Tato metoda nejdříve stáhne data z Room Driveru.
Následně načte z konfigurace témata pro stav dvíře, stav světla, teplotu a vlhkost.
Dále jsou pak data z Room Driveru zveřejněna na téma daná kofigurací odkud je přijímá systém Home Assistant a vizualizuje je.

%\subsection*{MQTT témata}
%\begin{itemize}
%    \item coopmaster/room/door/cmnd
%    \item coopmaster/room/lamp/cmnd
%    \item coopmaster/room/temperature
%    \item coopmaster/room/humidity
%    \item coopmaster/room/door/state
%    \item coopmaster/room/lamp/state
%\end{itemize}