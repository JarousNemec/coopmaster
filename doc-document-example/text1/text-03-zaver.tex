\newpage
\chapter{Závěr}\label{ch:zaver}

Závěrem lze říci, že se mi podařilo úspěšně vyvinout systém, který výrazně usnadňuje hospodářům plnění každodenních úkolů v reálném čase.
Díky mé aplikaci může hospodář kdykoli kontrolovat situaci v kurníku i ve výběhu, aniž by musel být fyzicky přítomen.
Systém ho automaticky informuje o jakýchkoli nesrovnalostech, například pokud se ve výběhu objeví vetřelec, který by mohl představovat nebezpečí pro slepice.
Úspěšně jsem také implementoval kontrolu počtu slepic v kurníku, což je užitečné například při automatizaci zavírání dvířek.

Dále systém poskytuje aktuální záběry z bezpečnostních kamer, což umožňuje hospodáři mít neustálý přehled o svém chovu.
Implementoval jsem také část systému schopnou kontrolovat stavy jednotlivých hnízd a zjistit, zda v nich sedí slepice nebo kolik vajec je v hnízdě.
Tato data lze využít k tvorbě statistik, díky nimž může hospodář analyzovat různé aspekty svého chovu.

Na druhou stranu jsem narazil na některé výzvy.
Nepodařilo se mi plně optimalizovat systém pro detekci vetřelců za zhoršených světelných podmínek, což může ovlivnit jeho spolehlivost v nočních hodinách.
Také přesnost při rozpoznávání počtu vajec v hnízdech potřebuje další vylepšení, aby byla zajištěna maximální spolehlivost dat.

Celkově můj systém představuje významný krok vpřed v automatizaci a monitorování chovu slepic, ačkoli stále existuje prostor pro další zdokonalení a optimalizaci některých funkcí.






