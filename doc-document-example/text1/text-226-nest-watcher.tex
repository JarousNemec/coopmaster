\section{Nest watcher}\label{sec:nest-watcher}
Tato služba interpretuje stavy jednotlivých hnízd v kurníku pro Home Assistanta.
Hlavní funkcí je načítání a analýza dat z jednotlivých vah v hnízdech, která jsou reprezentována Scale Drivery.\newline
Data jsou načítána několikrát za minutu a ukládána do databáze s časovým údajem, kdy byl záznam vytvořen.
Záznam v databázi obsahuje následující sloupce:
\begin{itemize}
    \item id: Jednoznačný identifikátor záznamu v databázi.
    \item nest\_id: Identifikační číslo hnízda (v naší implementaci aktuálně 1 až 6).
    \item weight: Naměřená hmotnost v gramech pro dané hnízdo.
    \item timestamp: Čas vytvoření záznamu.
\end{itemize}
Následně se jednou za minutu vyhodnotí průměrná hodnota během posledních několika vážení.
Tento kontinuální sběr a analýza dat nám umožňuje monitorovat statistiky pro jednotlivá hnízda.
Výsledky můžeme využít například ke zlepšení podmínek chovu.
Například v případě, kdy bychom zjistili, že některé hnízdo není pravidelně navštěvováno stejně jako ostatní, znamená to, že v hnízdě je něco, co slepice pravidelně odrazuje.\newline
\newline
Na základě vyhodnocené průměrné hmotnosti jsme schopni zjistit několik případů:
\begin{enumerate}
    \item \textbf{Hnízdo je prázdné}: Pokud průměrná hmotnost nepřevyšuje 50 gramů (vzhledem k možným chybám měření), vyhodnotíme, že hnízdo je prázdné.
    V hnízdě není ani slepice, ani vejce.
    \item \textbf{V hnízdě se nacházejí vejce}: Jestliže se hodnota hmotnosti pohybuje mezi 50 a 1200 gramy, znamená to, že v hnízdě jsou pravděpodobně vejce.
    Hmotnost jednoho vejce je přibližně 50 gramů, takže počet vajec lze vypočítat vydělením celkové hmotnosti hmotností jednoho vejce.
    \item \textbf{V hnízdě sedí slepice}: Pokud je průměrná hmotnost na váze více než 1200 gramů, vyhodnotí služba, že v hnízdě sedí slepice.
    Hmotnost slepice se pohybuje kolem 1200 gramů a více.
\end{enumerate}

Obrázek~\ref{fig:weight_egg_chart_timeline} znázořňuje hodnotu hmotnosti v čase pro hnízdo s nest\_id 1.

\begin{figure}[h]
    \centering
    \includegraphics[width=\textwidth]{img/weight_egg_chart_timeline}
    \caption{Graf vývoje hmotnosti hnízda v čase}
    \label{fig:weight_egg_chart_timeline}
\end{figure}

Tyto tři zmíněné informace služba následně pomocí MQTT předává do systému Home Assistant, aby je zpracoval a vizualizoval chovateli.
Tím je umožněno přijímat notifikace či provádět akce na základě stavů hnízd.
Díky měřením hmotnosti z jednotlivých hnízd můžeme odvodit i další vzorce, které vypovídají o aktuálním stavu daného hnízda.
Kromě třech výše uvedených případů můžeme z nasbíraných dat identifikovat například tyto situace:

\begin{itemize}
    \item \textbf{Doba setrvání slepice v hnízdě}: Analýzou délky období, kdy je hmotnost hnízda nad 1200 gramů, lze zjistit, jak dlouho slepice v hnízdě zůstává.
    To může poskytnout informace o jejím chování, pohodě a případně upozornit na možné zdravotní problémy, pokud setrvává déle než obvykle.
    \item \textbf{Frekvence snášení u jednotlivých hnízd}: Porovnáním počtu snesených vajec v různých hnízdech lze zjistit, zda některá hnízda nejsou preferována více než jiná.
    To může vést k úpravám uspořádání kurníku nebo kontrolám hnízd, která jsou méně využívána.
    \item \textbf{Poruchy senzorů}: Pokud hmotnost zůstává nezměněna po dlouhou dobu nebo vykazuje nerealistické hodnoty, může to indikovat technickou závadu.
    Pravidelná kontrola a kalibrace senzorů zajistí spolehlivost dat.
    \item \textbf{Denní vzorce aktivity}: Analýzou hmotnostních dat během dne lze určit období nejvyšší aktivity slepic.
    To může být užitečné pro plánování krmení, čištění kurníku nebo jiných činností, které by mohly slepice rušit.
\end{itemize}

Celkově systém obsluhuje šest hnízd, přičemž se data z každého hnízda monitorují a vyhodnocují.
Toto komplexní sledování umožňuje efektivní řízení kurníku, zajišťuje optimální péči o slepice a včasný sběr vajec.
Přístup založený na datech zvyšuje naši schopnost udržovat zdravé slepice a maximalizovat jejich produktivitu.

%\subsection*{Předpokládaná datová struktura v databázi:}
%
%\begin{table}[h!]
%    \centering
%    \begin{tabular}{lll}
%        \textbf{Pole} & \textbf{Datový typ} & \textbf{Popis}                                    \\
%        timestamp     & DATETIME            & Datum a čas měření                                \\
%        nid           & INT                 & Identifikační číslo hnízda (1 až 6)               \\
%        weight        & FLOAT               & Naměřená hmotnost v gramech                       \\
%        nest\_status  & VARCHAR             & Stav hnízda (např. "prázdné", "vejce", "slepice") \\
%    \end{tabular}
%    \caption{Datová struktura v databázi}
%\end{table}
%
%Tato datová struktura umožňuje efektivní ukládání a vyhodnocování dat z jednotlivých hnízd. Pole \texttt{nest\_status} může být automaticky vypočítáno na základě hodnoty v poli \texttt{weight} podle výše uvedených kritérií. Tímto způsobem můžeme snadno filtrovat a analyzovat data pro sledování stavu každého hnízda v čase.
%
%Rozšířená analýza nám poskytuje hlubší vhled do chování slepic a umožňuje nám přijímat informovaná rozhodnutí pro zlepšení podmínek v kurníku, zvýšení produktivity a zajištění pohody zvířat.

%Data jsou načítána několikrát do minuty a ukládána do databáze s časovým údajem, kdy byl záznam vytvořen.
%Následně se jednou za minutu vyhodnotí průměrná hodnota během posledních několika vážení.
%Na základě tohoto údaje jsme schopni zjisti několik případů
%\begin{itemize}
%    \item hnízdo je prázdné (hodnota na váze nepřevyšuje 50 g )
%    \item v hnízdě se nacházejí vejce (hmotnost jednoho vejce je průměrně 50 g)
%    \item v hnízdě sedí slepice (hmotnost slepice se pohybuje okolo 1200 g a více)
%\end{itemize}
%Pokud je na váze průměrně méně než 50 g, vzhledem k možným chybám měření, takový případ vyhodnotíme jako, že je hnízdo prázdné.
%Jestliže se hodnota pohybuje mezi 50 a 1200 gramy, znamená to, že v hnízdě jsou pravděpodobně vejce, a jejich počet je vypočítán vydělením celkové hmotnosti a hmotnosti jednoho vejce.
%V případě, že je na váze více jak 1200 g, vyhodnotí služba, že v hnízdě sedí slepice.
%Tyto tři zmíněné informace služba následně pomocí MQTT předává do Home Assistanta.
