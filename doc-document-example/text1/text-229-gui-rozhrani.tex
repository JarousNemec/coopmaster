\section{Uživatelská aplikace}\label{sec:tvorba-gui-rozhrani}
Hlavním důvodem proč jsem zvolil systém Home Assistant, aby kooperoval s mým systémem Coopmaster, je právě jeho robustnost, snadná konfigurace a široká možnost vizualizace dat.
Vytvořit takto robustní a kvalitní systém by trvalo nesmyslně dlouho.
Proto jsem se rozhodl využít již existující řešení, aby mi pomohlo vyřešit akurátní uživatelské rozhraní. \newline

\subsection*{Co je to Home Assistant}
Konfigurace systému Home Assistant se realizuje skrze soubor \texttt{configuration.yaml}, což je hlavní konfigurační soubor celého systému.
Práci s YAML jsem si rychle osvojil, jelikož je to efektivní způsob, jak strukturovaně a přehledně nastavit jednotlivé komponenty a integrace.

\subsection*{Výzvy s MQTT senzory}

Jednou z hlavních výzev bylo pochopit a aplikovat způsob, jak přidávat vlastní MQTT senzory do HA. Tato fáze byla složitá, jak ukázal i seznam závad v nástroji Trello, kde jsem sledoval všechny úkoly a problémy.
Po několika pokusech a testech jsem úspěšně integroval senzory, což umožnilo sledovat různé metriky ze systému Coopmaster.

\subsection*{Uživatelské rozhraní a vizualizace}

Po integraci senzorů jsem se zaměřil na vizualizaci dat tak, aby byla uživatelsky přívětivá.
Chtěl jsem, aby uživatelé měli intuitivní přístup k informacím o jednotlivých hnízdech a celkovém stavu kurníku.
Proto jsem vytvořil vlastní komponentu pro Home Assistant, jak ukazují i fotografie a naše Trello deska.

\subsection*{Limity textové předávání a přechod na CSV}

Zajímavou výzvou bylo zjištění limitu velikosti stavu (\textit{state}) v HA, který je pro textové zprávy maximální velikosti 255 bytů.
Tato omezení mě přiměly ke změně formátu přenášených dat z JSON na CSV, což pomohlo efektivně předávat větší objemy informací bez ztráty jejich kvality.

\subsection*{Vytváření vlastní komponenty}

Vytvoření vlastní komponenty pro HA vyžadovalo znalost architektury systému.
Komponenta se skládá z několika částí, včetně definice vlastností, funkcí pro zpracování dat a integrace do existujícího UI. Implementace a konfigurace této komponenty v HA umožnily přizpůsobení funkcionalit projektu konkrétním potřebám Coopmasteru.

\subsection*{Automatizace a další konfigurace}

Nakonec jsem se zaměřil na konfiguraci automatizací pro různé systémy Coopmaster, jako jsou dveře nebo přepínače pro manuální ovládání světel a dveří.
Snímky z kamer, přenášené pomocí MQTT, byly začleněny do HA, což umožnilo rychlé zobrazení obrázků při výstrahách, jako například u Dog Alert systému, který posílá upozornění doprovázené relevantními snímky jak na mobilní aplikaci, tak na dashboardu.

\subsection*{Shrnutí dat a informací}

Závěrem jsem zajistil, aby HA pravidelně zobrazoval klíčové metriky, jako teplotu, vlhkost a počet slepic v kurníku.
Tyto informace jsou neustále aktualizovány a umožňují uživatelům mít kompletní přehled o stavu jejich chovu.
Celý tento proces ukázal, jak důležité je precizní plánování a inovativní přístup ve fázi konfigurace a vývoje uživatelského rozhraní, aby bylo dosaženo vysoké úrovně automatizace a efektivity v projektu Coopmaster.


%- vybral se teda ten home assistant a ted ho nakonfit\newline
%- konfiguruje se to přes yaml configuration.yaml což je hlavní konfigurák HA\newline
%- byla výzva přijít na to jak se přidávají vlastní mqtt senzory viz seznam závad v trellu\newline
%- po přidání senzorů jsem si hrál s vizualizací jednotlivých hnízd tak aby to pro uživatele bylo přívětivě\newline
%- napsal jsem proto vlastní komponentu pro HA viz foto a trello\newline
%- zajímavé zjištění bylo že state který jsem využíval pro předávání textových zpráv má maximální velikost 255 bytů\newline
%- takže jsem z jsonu přešel na csv\newline
%- pokecat trochu o tom jak vytvořit takovou komponentu jaké to má části a předpoklady a jak se to následně přidává a konfiguruje v HA\newline
%- následně pak konfigurace automations pro dveře, přepínačů pro manuální ovladání světla a dvěří, obrázků z kamer které taky chodí pomocí mqtt, a následně ještě dog alert aby se poslalo upozornění pokud je mobilní apka a na dashboardu vyskočil daný obrázek a výstrahou\newline
%- závěrem pak výpis teploty, vlhkosti a počtu slepic v kurníku které systém poznal\newline