\section{Uživatelská aplikace}\label{sec:tvorba-gui-rozhrani}
Hlavním důvodem proč jsem zvolil systém Home Assistant~(\ref{sec:home-assistant}), aby kooperoval s mým systémem Coopmaster, je právě jeho robustnost, snadná konfigurace a široká možnost vizualizace dat.
Vytvořit takto robustní a kvalitní systém by trvalo nesmyslně dlouho.
Proto jsem se rozhodl využít již existující řešení, aby mi pomohlo vyřešit akurátní uživatelské rozhraní.

\subsection*{Požadavky na uživatelské rozhraní}
Na začátku práce~\ref{ch:tvorba-zadani} jsme si určili uživatelské požadavky, které by měl systém splňovat.
Nyní tyto požadavky převedeme na to, co pro nás znamenají při tvorbě uživatelské rozhraní.

\subsubsection*{Vzdálený přístup}
Systém Home Assistant je od začátku používán skrze webové rozhraní.
Vzdáleným přístupem tedy systém disponuje a nemusíme ho tedy nijak složitě implementovat.
Problémem je fakt, že běžící Home Assistant je přístupný pouze z lokální sítě.
Pro zřízení vzdáleného přístupu odkudkoli tedy využijeme privátního tunelu, které poskytuje například společnost Cloudflare.
Podrobně je výběr technologie pro vzdálený přístup popsán v sekci~\ref{sec:zpristupneni-aplikace-z-internetu}.
%todo ozdrojovat cloudflare
%todo zdroj pro mqtt image https://www.home-assistant.io/integrations/image.mqtt/

\subsubsection*{Pohledy z bezpečnostních kamer}
Pro splnění tohoto parametru bude třeba využít nějaké komponenty, která bude schopna zpracovat data z MQTT.
K využití se nám nabízí hotová komponenta přímo v systému Home Assistant.
Komponenta se jmenuje Picture a slouží ke zobrazení obrázku, který jí dodáván jako stav jakékoli entity.\newline
Abychom splnili tento bod bude tedy třeba nakofigurovat příslušné MQTT entity a přidat komponenty Picture pro zobrazení obrázků z entit.
%todo odkaz na pouziti picture a mqtt entity pro obrázky

\subsubsection*{Dálkové ovládání a automatizaci světla a dvířek v kurníku}
Tímto bodem rozumějme v prvé řadě ovládací komponenty/tlačítka, která bude využívat chovatel, aby manuálně vyslal příkaz k provedení nějaké akce.
Za druhé se zde dostáváme k dalšímu skvělému využití systému Home Assistant a to je automatizace.
Využitím systému Home Assistant lze automatizovat právě reakce na hodnoty různých čidel a senzorů.
Může jednat o automatizaci od úplných základů jako je zapnutí topení, když je v bytě zima, až po celé inteligentní domácnosti ovládané hlasem.
Takové domácnosti například na základě toho, kdo přišel domů, nastaví barvu světla, spustí jeho oblíbenou hudbu a uvaří kávu.
Já ve svém projektu využiji automatizaci k tomu, aby zavírala dvířka na základě času a počtu slepic, který poskytuje služba Chicken Watch Guard systému Coopmaster.
Dále pak automatizace na základě času zapíná a vypíná světlo.\newline
Pro splnění tohoto požadavku bude tedy třeba nakonfigurovat automatizaci a vytvořit ovládací tlačítka.
%todo přidat odkaz na coopmaster a chicken guarda a na dokumentaci home assistant automatizace

\subsubsection*{Intuitivní vizualizaci stavů jednotlivých hnízd(sedící slepice, v opačném případě počet vajec)}
Tento bod je ze všech nejnáročnější.
Home Assistant totiž nedisponuje žádnou pěknou komponentou přímo pro vizualizaci hnízd v kurníku a ani žádnou jinou, která by se dala po nějakých úpravách využít.\newline
Z tohoto důvodu bude tedy třeba vytvořit vlastní komponentu pro systém Home Assistant, která půjde snadno parametrizovat dle požadavků kurníku a bude umět zpracovat data, která přijdou ze služby Nest Watcher systému Coopmaster.
%//todo odkaz na nějakou stránku kde radí jak vytvořit vlastní komponentu

\subsubsection*{Upozornění na detekování vetřelce ve výběhu}
Zde se bude opět jednat o automatizace.
Požadavkem je, aby v případě poplachu systém poslal notifikaci chovateli a zobrazil na dashboardu aktuální fotografii, kde byl vetřelec detekován.\newline
Abychom splnili tento bod, budeme muset nastavit automatizaci, která zareaguje na určitý stav nějaké MQTT entity a následně vytvoří notifikaci, jež ,pokud chovatel používá mobilní aplikaci, vyskočí i na chovatelově telefonu.
Dále bude použita komponenta s názvem Conditional, která opět při daném stavu MQTT entity zobrazí v dashboardu upozornění na vetřelce a opačném případě nezobrazí nic, což se hodí právě pro různá upozornění.
%//todo odkaz na komponentu conditional

\subsubsection*{Data o teplotě a vlhkosti}
Tyto dva požadavky nebudou co do implementace nic složitého.
Jedná se pouze o dvě hodnoty, které navíc ze služby Room Assistant chodí ve formátu JSON takže není problém je zpracovat.\newline
Bude tedy třeba nakonfigurovat MQTT entitu, kterou následně bude načítat komponenta Entities.
Tato komponenta zobrazuje v seznamu pod sebou entity, jejichž seznam přebírá jako parametr.
Pro zobrazení každé entity je možno upravit ikonu a zobrazovaný název, což přesně odpovídá našim požadavkům.
%todo odkaz na kompoentu entities

\subsection*{Postup konfigurace systému Home Assistant}
Po rozvržení řešení jsem se začal seznamovat se systéme Home Assistant s čímž mi hodně pomohla rozsáhlá a kvalitní dokumentace společně s diskusními fóry, která jsem i nadále hojně využíval.
\subsubsection*{MQTT entity}
Konfiguraci jsem započal definováním všech potřebných MQTT entit v souboru configuration.yaml, do kterých systém ukládá data přijatá z MQTT brokeru a následně je můžeme využít v našem dashboardu.
Používám tři typy MQTT entit.
\begin{itemize}
    \item sensor = přijímá běžné hodnoty senzoru jako čísla a texty (maximální velikost je 255 bytů)
    \item image = umožňuje přijmout obrázek z MQTT brokeru
    \item binary\_sensor = na základě nakonfigurovaných hodnot je jeho stav buď 0 nebo 1
\end{itemize}
Každá MQTT potřebuje nastavit jednotnačné jméno pro identifikaci v systému a topic, ze kterého bude odebírat zprávy.

%todo ukazka na konfiguraci entit a na dokumentaci

\subsubsection*{Základní dashboard}
Jakmile se nakonfigurují entity, následuje tvorba samotného dashboardu.
Za pomocí webového rozhraní jsem tedy přidal tlačítka, vizualizace obrázků a výpisy hodnot entit tak, jak bylo rozebráno dříve v návrhu.
Po přidání komponent bylo třeba přiřadit odpovídající entity ke komponentám opět s využitím webového rozhraní.

\subsubsection*{Automatizace}
Díky intuitivnímu webovému rozhraní systému není problém přidat jakoukoli automatizaci.
Automatizaci si lze představit stejně jako klasickou podmínku v programování.
Je zde sekce Spouštěč, která nastavuje hlavní důvod spuštění automatizace.
Následuje volitelná sekce A Pokud.
Tato sekce udává další omezení pro spuštění scriptu.
Na závěr je zde sekce Pak Provést, která právě specifikuje akci nebo akce, které se mají uskutečnit při splnění podmínek.
S těmito vědomostmi jsem opět nakonfiguroval dle návrhu potřebné automatizace pro světlo, dvířka a notifikace.

%todo najít návod na přidávání automatizace a dat ho taky jako zdroj

\subsubsection*{Tvorba vlastní Home Assistant komponenty}
Zde se již dostáváme k opravdovému programování.
Služba Nest Watcher poskytuje data o tom zda je hnízdo obsazené a kolik je v každém hnízdě vajec.
Tato data jsou ve formátu CSV, kdy každý prvek v poli znamená jedno hnízdo.
Data se do komponenty dostávají z entit.
Tyto entity jsou se nastavují jako parametry během přidávání komponenty na dashboard.
Data, která dostaneme z entity, jsou v CSV formátu, takže se pomocí funkce split rozdělí podle separátoru, kterým je v tomto případě znak středníku.


%todo dodat ukázky kodu
%todo zdroj pro tvorbu vlastní komponenty https://developers.home-assistant.io/docs/frontend/custom-ui/custom-card/


\subsection*{Problémy a poznatky během konfigurace}
maximální velikost state je 255 bytů
json a csv proč
trošku problém s dokumentací