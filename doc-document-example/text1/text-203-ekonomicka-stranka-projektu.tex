Projekt Coopmaster je rozhodně revoluční řešení v chovatelské branži.
Každopádně se rozhodně nehodí jeho využití jako odpověď na snížení ceny za provoz malých domácích chovů.
Hlavním důvodem je už samotná pořizovací cena komponent, která je u mého prototypového řešení okolo 20 ticíc korun.
Je tedy rozhodně zbytečné instalovat systém do malého domácího kurníku, když samotná pořizovací cena předčí cenu vajec vyprodukovaných za 3 roky, a to se ještě nebavím o licencích za software a energie potřebně pro běh systému.\newline
Potenciál tohoto nápadu by se dle mého dal určitě zúročit ve větších podnicích, kde bude možno naplno využít schopností celého řešení.
Například počítání by určitě pomohlo například na pastvách s ovcemi či kravami, kde jsou počty výrazně větší a je již složité a časově náročné ručně spočítat všechny kusy.
Dále by se potenciál kamer a neuronových sítí dal využít k detekci nemocných nebo zraněných zvířat.
Toto jsou faktory, které by rozhodně stálo za to vyzkoušet zlepšit.
Například pokud by systém správně s předstihem detekoval nějakou nemoc na jednom jedinci ve stádě krav, mohlo by to pomoci dříve zareagovat a zachránit zbytek stáda před uhynutím.
V takovém případě by se právě systém velice vyplatil, protože by vlastně zachránil firmu před krachem.\newline
Dále je určitě vhodné zmínit možnost automatizace krmení a čištění sídel zvířat.
To je věc, která se dá plně automatizovat a ušetřit tak čas a případně i pracovní sílu, která by jinak tyto činnosti konala.\newline
Na závěr tedy můžeme říct, že myšlenka tohoto systému se nevyplatí pro malé chovatele, pokud to nejsou techničtí nadšenci jako já.
Své uplatnění, ale najde u větších firem, kterým může v daných situacích i zachránit byznys.

//TODO suma



