
Projekt \textbf{Coopmaster} představuje moderní přístup k automatizaci chovu slepic s využitím nejnovějších technologií.
Cílem je usnadnit práci chovateli a zvýšit efektivitu a kvalitu chovu.
Níže je uvedena ekonomická analýza projektu, která zhodnocuje finanční náklady a potenciální přínosy.

\subsection*{Počáteční investiční náklady}

\begin{itemize}
    \item \textbf{Hardware}:
    \begin{itemize}
        \item IP kamery pro monitoring: 2 ks × 1\,500 Kč = 3\,000 Kč
        \item Teplotní a vlhkostní senzory: 2 ks × 250 Kč = 500 Kč
        \item Řídicí jednotka (např. Raspberry Pi): 1\,500 Kč
        \item Automatické dveře: 2\,000 Kč
        \item Automatické krmítko: 1\,500 Kč
        \item Automatická napáječka: 800 Kč
        \item Další vybavení (kabeláž, montážní materiál): 700 Kč
    \end{itemize}
    Celkem hardware: \textbf{10\,000 Kč}
    \item \textbf{Software}:
    \begin{itemize}
        \item Vývoj softwaru: 100 hodin × 200 Kč/hod = 20\,000 Kč
    \end{itemize}
    Celkem software: \textbf{20\,000 Kč}
    \item \textbf{Instalace a implementace}:
    \begin{itemize}
        \item Instalace a úpravy kurníku: 5\,000 Kč
    \end{itemize}
    Celkem instalace: \textbf{5\,000 Kč}
\end{itemize}

Celkové počáteční investiční náklady jsou tedy:

\[
\text{Hardware (10\,000 Kč)} + \text{Software (20\,000 Kč)} + \text{Instalace (5\,000 Kč)} = \textbf{35\,000 Kč}
\]

\subsection*{Provozní náklady}

\begin{itemize}
    \item \textbf{Spotřeba energie}:
    \begin{itemize}
        \item Spotřeba zařízení: 20 W
        \item Roční spotřeba: \(20~\text{W} \times 24~\text{hod} \times 365~\text{dní} = 175{,}2~\text{kWh}\)
        \item Náklady na energii: \(175{,}2~\text{kWh} \times 5~\text{Kč/kWh} = 876~\text{Kč/rok}\)
    \end{itemize}
    \item \textbf{Údržba}:
    \begin{itemize}
        \item Hardwaru: 500 Kč/rok
        \item Softwaru: 2\,000 Kč/rok
    \end{itemize}
    \item \textbf{Datové služby}:
    \begin{itemize}
        \item Internetové připojení: \(100~\text{Kč/měsíc} \times 12 = 1\,200~\text{Kč/rok}\)
    \end{itemize}
\end{itemize}

Celkem provozní náklady: \textbf{4\,576 Kč/rok}

\subsection*{Potenciální úspory a přínosy}

\begin{itemize}
    \item \textbf{Úspora času}:
    \begin{itemize}
        \item Denní úspora: 2 hodiny
        \item Roční úspora: \(2~\text{hod/den} \times 365~\text{dní} = 730~\text{hodin}\)
        \item Finanční hodnota: \(730~\text{hod} \times 200~\text{Kč/hod} = 146\,000~\text{Kč/rok}\)
    \end{itemize}
    \item \textbf{Snížení cestovních nákladů}:
    \begin{itemize}
        \item Denní ujetá vzdálenost: 80 km
        \item Roční ujetá vzdálenost: \(80~\text{km} \times 365~\text{dní} = 29\,200~\text{km}\)
        \item Náklady na palivo: \(29\,200~\text{km} \times 7~\text{l/100 km} \times 40~\text{Kč/l} = 81\,760~\text{Kč/rok}\)
    \end{itemize}
    \item \textbf{Efektivnější chov}:
    \begin{itemize}
        \item Snížení úhynu slepic: 200 Kč/rok
        \item Zvýšení produkce vajec: 750 Kč/rok
    \end{itemize}
\end{itemize}

Celkové roční úspory: \textbf{82\,710 Kč}

\subsection*{Návratnost investice}

Počáteční investici 35\,000 Kč lze díky ročním úsporám 82\,710 Kč vrátit za přibližně 5 měsíců:

\[
\frac{35\,000~\text{Kč}}{82\,710~\text{Kč/rok}} \approx 0{,}42~\text{roku}
\]

Za pět let provozu lze očekávat čistý přínos:

\[
(82\,710~\text{Kč/rok} \times 5~\text{let}) - (4\,576~\text{Kč/rok} \times 5~\text{let} + 35\,000~\text{Kč}) = 355\,670~\text{Kč}
\]

\subsection*{Závěrečné zhodnocení}

Projekt \textbf{Coopmaster} nabízí ekonomicky efektivní řešení s rychlou návratností investice.
Automatizace chovu přináší významné úspory času a finančních prostředků, zvyšuje komfort a efektivitu péče o slepice.
Doporučuje se zvážit možnosti komercializace systému a další optimalizace nákladů pro ještě lepší ekonomickou efektivitu.


