\externaldocument{text-02-teoreticka}


\chapter{Praktická část}\label{ch:prakticka-cast}
Tato část práce předvede čtenáři konkrétní realizaci myšlenky probírané v této prací.
Zároveň v se v ní čtenář dozví jak v praxi využít poznatky nabyté v teoretické části.


\section{Rozvržení person}\label{sec:rozvrzeni-person}
Před tím než se započne tvorba jakékoli aplikace, je třeba určit, pro koho danou aplikaci tvoříme a jak bychom chtěli, aby jí tento uživatel používal.
V našem případě jde o to, že systém by měl být schopný používat člověk znalý v chovu hospodářských zvířat, ale často méně zdatný v používání informačních technologií a mobilních aplikací.
Na základě toho, že už víme, kdo je cílový uživatel můžeme se zamýšlet jakou funkcionalitu do řešení implementovat.
Jedná se tedy o to, aby aplikace umožňovala uživateli vzdálený dohled na jeho chov a pomáhala mu šetřit čas zautomatizováním každodenních činnosti.

Rozhodlo se tedy, že aplikace bude uživateli poskytovat následující funkce
\begin{itemize}
    \item Vzdálený přístup
    \item Pohledy z bezpečnostních kamer
    \item Dálkové ovládání a automatizace světla a dvířek v kurníku
    \item Intuitivní vizualizace stavů jednotlivých hnízd(sedící slepice, v opačném případě počet vajec)
    \item Upozornění na detekování vetřelce ve výběhu
    \item Poskytování dat o teplotě a vlhkosti
\end{itemize}

\section{Návrh architektury a volba technologií}\label{sec:navrh-architektury-a-volba-technologii}
Díky tomu, že si přesně určíme, kdo je cílový uživatel naší aplikace, a k tomu dáme do hromady, co uživatel od naší aplikace očekává.
Jsme tedy nyní schopni bez větších problémů teoreticky navrhnout architekturu našeho systému a přesně popsat požadovanou funkcionalitu, jakou budou disponovat jeho jednotlivé části.
Tato sekce se bude zabývat analýzou a návrhem architektury systému Coopmaster dle požadavků ze sekce~\ref{sec:rozvrzeni-person}.

\subsection{Architektura dle jednotlivých potřeb uživatele}\label{subsec:microservices}
Jako teoretický model na základě kterého, budeme organizovat a třídit funkcionalitu, jsem zvolil mikroservisní architekturu(sekce~\ref{sec:microservice-architecture}).
Tento způsob rozvržení zodpovědnosti částí systému jsem zvolil, díky velké možnosti rozšíření, zapouzdření funkcionality a snadné úpravě jednotlivých služeb bez nutnosti kompletního restartu systému případně znovunasazení.
Jako programovací jazyk pro tvorbu služeb je zvolen programovací jazyk Python(sekce~\ref{sec:python}).
Python byl vybrán kvůli jeho univerzálnosti, snadné syntaxi a podpoře ze strany vývojářů a komunity.
Pro realizaci, konfiguraci a síťování mikroservistní architektury je použit Docker Engine(sekce~\ref{sec:kontejnerizace}) společně s rozšířením Docker Compose.
%\subsubsection{Zjednodušení implementace uživatelského rozhraní a automatizace úkolů}
Podstatnou otázkou bylo, zda se pouštět do tvorby kompletně vlastního systému pro uživatelského rozhraní a automatizaci.
Po zvážení časové náročnosti, technologické náročnosti a komplexnosti takového systému, mi bylo jasné, že tudy cesta nepovede.
Bylo tedy třeba najít open source systém, který již touto funkcionalitou disponuje.
Nakonec jsem zvolil systém Home Assistant, protože jsem s ním již dříve pracoval a mám s ním dobré zkušenosti.
Home Assistant má na internetu také rozsáhlou podporu a komunitu s množstvým nápadů a již hotových řešení pro inspiraci.
Popis toho, jak je nakonfigurováno uživatelské rozhraní v Home Assistantovi, je popsáno v sekci~\ref{subsec:tvorba-gui-rozhrani}.

\subsubsection{Ovládání světla a poskytování dat o teplotě a vlhkosti}
%todo popis proč a jak to plní uživatelskou storku, popis jaké jsou vazby, jak jednotlivé moduly spolupracují, vysvětlení jak jsem se dobral k tomuto řešení
Jsou použity částí Řídící jednotka, Moduly systému Coopmaster(Room Driver a Room Assistant) a systém Home Assistant.
\begin{figure}[h]
    \centering
    \includegraphics[width=\textwidth]{img/svetlo_teplo_vlhkost}
    \caption{Mechanismus pro ovládání světla a poskytování dat o teplotě a vlhkosti}
    \label{fig:svetlo_teplo_vlhkost}
\end{figure}
\subsubsection{Automatizace dvířek}
%todo popis proč a jak to plní uživatelskou storku, popis jaké jsou vazby, jak jednotlivé moduly spolupracují, vysvětlení jak jsem se dobral k tomuto řešení
Jsou použity částí Kamerový systém, Elektrická dvířka, Moduly systému Coopmaster(Camera Driver, Chicken Watch Guard, Room Driver a Room Assistant) a systém Home Assistant.
\begin{figure}[h]
    \centering
    \includegraphics[width=\textwidth]{img/automatizace_dvirek}
    \caption{Mechanismus bezpečného zavření dvířek}
    \label{fig:automatizace_dvirek}
\end{figure}
\subsubsection{Detekce vetřelců}
%todo popis proč a jak to plní uživatelskou storku, popis jaké jsou vazby, jak jednotlivé moduly spolupracují, vysvětlení jak jsem se dobral k tomuto řešení
Jsou použity částí Kamerový systém, Moduly systému Coopmaster(Camera Driver a Dog Alarm) a systém Home Assistant.
\begin{figure}[h]
    \centering
    \includegraphics[width=\textwidth]{img/detekce_vetrelcu}
    \caption{Mechanismus detekce vetřelců}
    \label{fig:detekce_vetrelcu}
\end{figure}
\subsubsection{Vizualizace stavů hnízd}
%todo popis proč a jak to plní uživatelskou storku, popis jaké jsou vazby, jak jednotlivé moduly spolupracují, vysvětlení jak jsem se dobral k tomuto řešení
Jsou použity částí Digitální Váha, Moduly systému Coopmaster(Scale Driver a Nest Watcher) a systém Home Assistant.
\begin{figure}[h]
    \centering
    \includegraphics[width=\textwidth]{img/vizualizace_stavu_hnizd}
    \caption{Mechanismus vizualizace stavů hnízd}
    \label{fig:vizualizace_stavu_hnizd}
\end{figure}
%Na základě určeného způsobu zapouzdření funkcionality je architektura rozdělena do následujících služeb, které jsou pojmenovány dle jejich účelu
%\begin{itemize}
%    \item Camera driver
%    \item Scale driver
%    \item Room driver
%    \item Health checker
%    \item Room assistant
%    \item Nest watcher
%    \item Dog alarm
%    \item Chicken watch guard
%\end{itemize}

\subsection{Komunikace mezi Backendem a Frontendem}\label{subsec:komunikace-mezi-backendem-a-frontendem}
Je třeba zajistit komunikaci mezi Home Assistantem a Službami.
Tuto úlohu musíme přijmout velice zodpovědně a navrhnout řešení, které půjde opět snadno rozšířit a modifikovat.
Nelze proto použít klasické HTTP(sekce~\ref{sec:http-rest}), z toho důvodu že bychom komunikaci vázali na buď doménové jméno a port nebo ip adresu a port.
Toto řešení má problém v tom, že pokud bychom potřebovali změnit, buď umístění částí služeb nebo port jedné ze služeb, bylo by třeba překonfigurovat i home assistanta.
Dalším problém nastane, když potřebujeme z některé ze služeb poslat například notifikaci do Home Assistanta, je pro to potřeba, aby jednotlivé služby věděli, kde na síti Home Assistant běží, což je věc, která se může měnit, a museli bychom naopak přenastavovat jednotlivé služby.
Jako řešení se nabízí použít messaging konkrétně třeba technologii MQTT(sekce~\ref{sec:mqtt}).
Tato technologie se běžně používá u IoT zařízení a pro naše použití bude vynikajícím řešením.

\subsection{Hardwarová zařízení}\label{subsec:hardwarova-zarizeni}
Tuto sekci je velmi důležité dobře a správně navrhnout, protože náš systém bude pracovat s daty z reálného světa.
Tato data musejí získávat k tomu určená zařízení a na jejich přesnosti závisý efektivita a správnost chování zbytku systému.
Musíme tedy navrhnout a realizovat několik zařízení, která společně budou plnit následující úlohy
\begin{itemize}
    \item snímání stavů hnízd
    \item ovládatní dvířek
    \item snímání teploty a vlhkosti
    \item snímání scén v kurníku a ve výběhu
\end{itemize}
\subsubsection{Snímání stavů hnízd}
Všechny případy, které potřebujeme v kontextu našeho hnízda řešit se dají rozpoznat na základě hmotnosti hnízda.
Pokud je v hnízdě slepice, hnízdo bude hodně zatížené oproti počátečnímu prázdnému stavu.
V případě, že bude hnízdo zatížené málo a nebo středně, může to pro nás ve většině případů spolehlivě znamenat, že se v hnízdě nacházejí vejce.
Nestandartní situace jako neobviklý nepořánek v hnízdě případně výkal, řešit nebudeme, protože by to exponenciálně zvýšilo složitost řešení a nepřineslo by to o moc lepší data.
Kurník ja navíc v našem případě pravidelně udržován, takže šance na výskyt chybového faktoru v podobě nežádoucí zátěže je nízká.
Jako nejjednodužší prostředek pro splnění tohoto úkolu na základě předchozí analýzy se ukázala obyčejná digitální váha.
Tu bude třeba vytvořit vlastní konstrukce vzhledem ke skutečnosti, že ji musíme zabudovat do již existujícího hnízda.
Popis konkrétního zapojení a konstrukce je sepsán níže v sekci~\ref{subsec:digitalni-vaha-do-hnizda}.

\subsubsection{Ovládatní dvířek, snímání teploty a vlhkosti}
Pro ovládatní dvířek a snímání povětrnostních podmínek je nejrozumnější vytvořit řídící jednotku, která tyto požadavky obsáhne.
Další částí částí jsou samotná elektrická dvířka, jejichž funkce bude na základě způsobu elektrického signálu být otevřená nebo zavřená.
Teplota a vlhkost jsou snímány senzory, jež jsou souřástí řídící jednotky.
Řídící jednotka je popsána v sekci~\ref{subsec:ridici-jednotka} a konstrukce dvířek v sekci~\ref{subsec:konstrukce-dvirek}.

\subsubsection{Snímání scén v kurníku a ve výběhu}
Ke snímání scén v daných oblastech je třeba vybrat zařízení s dostatečnou odolností proti povětrnostním vlivům.
Speciálně v kurníku je hodně agresivní prostředí kvůli vysoké vlhkosti a pro dlouhodobý provoz je třeba vybrat kvalitní a odolnou techniku.
Zařízení musí být připojeno přes ethernet a musí být napájené ze sítě nikoli pomocí baterie.
Na základě předchozí analýzy je vhodné použít venkovní ip kameru.
Ip kamera s dostatečným krytím, dobrou kvalitou obrazu a velkým zorným polem bude snadná na instalaci, je cenově dostupná a poskytuje dostatečně kvalitní data pro naše použití při detekci objektů.
Podrobnosti ohledně výběru kamery se nachází v sekci~\ref{subsec:kamerovy-system}

\subsection{Zpřístupnění aplikace z internetu}\label{subsec:zpristupneni-aplikace-z-internetu}
Aby systém mohl sloužit svému účelu musí být vzdáleně přístupný.
To znamená, že server, na kterém systém poběží, musí mít veřejnou ip adresu a v ideálním případě mít tuto adresu propojenou s doménovým názvem pro snadnější použití.
Tato problematika se dá řešit několika způsoby.\newline
Jeden ze způsobů je zařízení si přímo veřejné ip adresy pro svůj server.
Toto řešení je, technicky i teoreticky náročné, protože v případě, se kterým pracujeme my, běží server na lokální síti a veřejná ip adresa tedy není přímo pro náš server, ale celou síť.
To vytváří v určitých situacích vytváří velmi vážná bezpečnostní rizika, díky čemuž je tato metoda velice náročná na hardware a implementaci.
Navíc poskytování veřejné ip adresy je placená služba poskytovatel internetového připojení.\newline
Další způsob je nasazení části systému do některého z cloudových řešení jako je Microsoft Azure nebo Amazon AWS.
Tato metody by v podstatě vyhovovala našim požadavkům, ale pokud bychom se bavili o ceně, tak to rozhodně není levné řešení. \newline
Metoda, kterou využíváme v řešení my, je snadná, jednodužší na implementaci než zřizování veřejné ip adresy a levnější nez využití cloudové služba.
Tento způsob využívá takzvané tunely neboli šifrovaná spojení.
Funguje na principu, při němž se zařízení šifrovaně propojíte s veřejným serverem a tento server bude skrze sebe vystavovat službu, která běží na lokálním počítači například u nás doma.
Výhoda je, že vzdálený server disponuje kvalitním a odborně nastaveným firewalem, ktery zajistí dobrou ochranu našeho lokálního serveru.
Tuto službu poskytuje zdarma a bez omezení například společnost Cloudflare.
Tento způsob nejvíce odpovídá našim požadavkům, které byli cena a jednoduchost společně s rychlost nasazení.\newline
Poslední nutnostní, kterou ovšem je třeba zaplatit, je pronájem vlastního doménového jména.
Doménu si lze pronajmout u doménových registrátorů jako jsou GoDaddy, Hostinger nebo například Forpsi.
Ceny domén se odvíjí na základě jejího druhu a pohybují od 20 do 1500 korun za kus.
My v našem řešení využíváme službu Forpsi, protože s ní máme již předchozí zkušenost.\newline
Dále je tato část rozvedena v sekci~\ref{sec:nasazeni-do-kurniku}, kde je popsán konkrétní způsob nasazení do funkčního projektu.

\section{Popis jednotlivých modulů systému}\label{sec:popis-jednotlivych-modulu}

\externaldocument{text-02-teoreticka}
\subsection{Camera driver}\label{subsec:camera-driver}
Camera driver je služba zodpovědná za sprostředkování komunikace  mezi ip kamerou(sekce~\ref{sec:ipcamera-rtsp}) a službami zpracovánajícími obraz (Dog alarm, Chicken watch guard).\newline
S kamerou komunikuje pomocí protokolu RTSP(sekce~\ref{sec:ipcamera-rtsp}).
Url ip kamery, k níž je driver přiřazen, je službě předávána pomocí environment proměnných(sekce~\ref{sec:environment-variables}).
Pro komunikaci se zbytkem služeb poskytuje Camera driver své REST(sekce~\ref{sec:http-rest}) api.
Konkrétně při zavolání na endpoint driver stáhne nejnovější obrázek z kamery a vrátí ho jako odpověď na volání.

%\begin{itemize}
%    \item GET /api/image
%\end{itemize}

\externaldocument{text-02-teoreticka}

\subsection{Scale driver}\label{subsec:scale-driver}
Scale driver zodpovídá za komunikaci mezi fyzickou váhou a službami, které využívají data o vážení.\newline
Protože jako řídící jednotka váhy je použito Arduino(\ref{sec:arduino}) tento modul komunikuje přes serialový port pomocí protokolu USB a na dotaz přijmutý RESTovým api poskytne jako odpověď hodnotu načtenou z váhy.

\subsubsection{Funkcionalita}
Nejprve naběhne hlavní vlákno, které se stará o poskytování RESTového api na portu 9004.
Následně je spuštěno nové vlákno, které má za úkol číst data z váhy a předávat je aplikačnímu rozhraní jako aktuálně naměřenou hmotnost na váze, jež se bude od teď poskytovat, pokud o ní někdo GET requestem požádá.
Čtení informací z váhy se prování v nekonečném cyklu.
Před začátkem cyklu se zaloguje že je nové vlákno úspěšně spuštěno a následně proběhne otevření komunikace s váhou přes seriový port s využitím knihovny PySerial.
Po úspěšném otervření spojení se váze vždy po uplinutí časového intervalu 2 sekundy, odešle znak w.
Tento znak je ve firmwaru váhy vedeno jako příkaz, při jehož přijetí má váha vrátit aktuální naměřenou hodnotu.
Časové spoždění je implementováno kvůli tomu, že není třeba číst data tak často a zároveň by mohlo docházet k zahlcení váhy.
Jakmile je celá odpověď od váhy přijata programem zpět, jsou data zvalidována a následně, pokud je to možné, převedena na datový typ Int.
Datový typ Int je zvolen z důvodu, že váha posílá data v gramech a je zbytečné v tomto případě používat desetinná čísla, protože gramy poskytují dostatečné rozlišení pro naše účeli.
Pokud se nepodaří převézd přečtenou hodnotu na číslo, je výtupní promněnná nastavena na -1, což zbytku systému značí, že váha není v pořádku.
Na závěr se zalogují aktuálně získaná data a následuje další volání cyklu.

\subsubsection{Technologie}
\begin{itemize}
    \item Jazyk Python a jeho knihovny
    \item REST api
    \item Serialový port
\end{itemize}

\subsubsection{Použíté knihovny}
Flask: Lehký webový framework pro rychlý vývoj webových aplikací.
colorama: Manipulace s barvami v textovém výstupu na terminálu.
waitress: Rychlý WSGI server pro produkční nasazení webových aplikací.
pyserial: Komunikace se sériovými zařízeními přes sériovéporty.
python-dotenv: Načítání konfigurace z .env souborů.

\externaldocument{text-02-teoreticka}
\subsection{Room driver}\label{subsec:room-driver}
Room driver zařizuje komunikaci mezi ostatními službami a řídící jednotkou v kurníku, která ovládá dveře a světlo.\newline
Jako mozek řídící jednotky je použito opět Arduino a tomu je třeba přizpůsobit architekturu služby.
Tento modul má tedy za úkol přes serialový port pomocí protokolu USB posílat příkazy a načítat stavy řídící jednotky na základě requestů příchozích na REST api služby.
Pro služby v systému služba na vystavuje GET a POST endpointy
%\begin{itemize}
%    \item GET /api/temperature (dej teplotu)
%    \item GET /api/humidity (dej vlhkost)
%    \item GET /api/lamp/state (dej stav světla vypnuto/zapnuto)
%    \item GET /api/door/state (dej pozici dvířek otevřeno/zavřeno)
%    \item POST /api/lamp/on (rozsviť)
%    \item POST /api/lamp/off (zhasni)
%    \item POST /api/door/open (otevři)
%    \item POST /api/door/close (zavři)
%\end{itemize}

\externaldocument{text-02-teoreticka}

\subsection{Health checker}\label{subsec:health-checker}
Health checker je malá služba určená pro správce systému.
Poskytuje informace o tom zda všechny potřebné služby běží, aby správce nebyl nucen přihlašovat se vzdáleně na server a manuálně kontrolovat každou službu.
Výpis stavů jednotlivých služeb je poskytován pomocí jednoduché webové stránky.

\subsubsection{Funkcionalita}
Po nastartování služby je spuštěno REST api pro komunikaci se správcem.
Jediným endpointem služby je /status.
Pokud uživatel pomocí prohlížeče pošle GET požadavek na tuto službu, bude mu vrácen seznam služeb a jejich stavů převedený do formátu JSON.
Stavy jsou myšleny status kódy a případné chybové hlášky, které služby po zavolání jejich metody /ping.
Služby jsou v seznamu identifikovány na základě portů.
Každá služby v systému Coopmaster má svůj port a porty jsou po sobě v intervalu od 9000 do 9010.
Tento způsob identifikace a rozdělení portů službám zjednodušuje jejich monitorování.
Algoritmus funguje na základě plánoveče úloh neboli scheduleru, který jednou do minuty spustí kontrolu všech služeb a jejich stavy uloží do seznamu odkud jsou pak načítány při požadavku na api.

%todo: sehnat obrázek jsonu který služba vrací

\input{text1/text-304-room-assistant}

\input{text1/text-305-nest-watcher}

\externaldocument{text-02-teoreticka}
\subsection{Dog alarm}\label{subsec:dog-alarm}
Služba Dog alarm má detekovat nebezpečí ve výběhu a poslat tuto zprávu do Home Assistanta.\newline
Aktuální záběry jsou pomocí GET requestů stahovány z konkrétní instance služby Camera driver, která je přiřazena ke kameře ve výběhu.
Analýza probíhá v určitých intervalech za pomocí umělé inteligence, kde je konkrétně použita metoda detekce objektů.
Jakmile jako výsledek klasifikace vyjde jednoznačně, že v záběru byl spatřen pes nebo jiný predátor, je zpráva poslána pomocí MQTT do Home Assistanta společně s konkrétním záběrem, na němž byl predátor detekován.
Tato služba zároveň přes MQTT posílá do Home Assistanta aktuální záběr z kamery.

\externaldocument{text-02-teoreticka}
\subsection{Chicken watch guard}\label{subsec:chicken-watch-guard}
Úkolem služby Chicken watch guard je sledovat stav a počet slepic v kurníku.\newline
Aktuální záběry jsou stejně jako u Dog alarmu stahovány z konkrétního Camera driveru v kurníku.
Následně po získání záběru proběhne detekce objektů a počet těchto objektů udává počet detekovaných slepic v obraze.
Tato hodnota je publikována pomocí MQTT do Home Assistanta.
Vedlejší funkcí služby je průběžné posílání aktuálního pohledu z kamery v kurníku do Home Assistanta.

\input{text1/text-308-gui-rozhrani}

\externaldocument{text-02-teoreticka}
\subsection{Digitální váha do hnízda}\label{subsec:digitalni-vaha-do-hnizda}
- vysvětlit přijimane commandy w - dej vahu a t - tare
zastrcim arduino do elektriky
rozběhne se loop
z pinů xy čteme hodnoty
knihovna zpracovava hodnoty odporu
knihovna řeší vynulování váhy
vysvětlit kalibraci
najít a popsat kalibrační program a vysvětlit k čemu je potřeba
loop čeka a kontroluje serialovy vstup zda neni pismenko w nebo t
popis loopu

je tam převodník pro převod analogu na digital
odecita odpor
na konci toho je tenzometrikej můstek
specifikace senzoru
plusy a mínusy senzoru deformace, vysvětlit princip

\externaldocument{text-02-teoreticka}
\subsection{Řídící jednotka}\label{subsec:ridici-jednotka}

\externaldocument{text-02-teoreticka}
\subsection{Kamerový systém}\label{subsec:kamerovy-system}

\externaldocument{text-02-teoreticka}
\subsection{Konstrukce dvířek}\label{subsec:konstrukce-dvirek}
Elektrická dvířka systému Coopmaster jsou tím hlavním důvodem proč tento projekt vůbec vznikl.
Prozatím se jedná o prototyp vlastní konstrukce, který se bude před trvalou instalací ještě zdokonalen a konstrukce bude navržena vzhledem k ceně řešení a na míru potřebám chovatele.
Aktuálně jsou dvířka sestavena ze standardních hliníkových profilů.
Pro pohyb dvířek se pro prototyp jeví jako nejsnažší volba táhlový motor.
Tento motor je připevněn ke kovovému plátu, který slouží jako posuvná dvířka.
Toto řešení je nejjednodužší na realizaci, díky snadno proveditelné konstrukci.
Řešení však není ideální pro velkovýrobu, protože hliníkové součásti společně s motorem jsou příliš drahé a je zbytečné si do kurníku instalovat dvířka, která svou cenou převýší cenu vajec, které spotřebujete za rok.

Finálním řešením by se mělo
\subsubsection{Táhlový motor}
Jedná se o motor, jehož součástí je samotný elektromotor a k němu je připevněn šnekový převod do síli.
Šnek je pevně upevněn na šroubovici, která se stará o rozvod síli a točivého momentu po celé délce motoru.
Převod z otáčivého pohybu na pohyb přímočarý je zajištěn maticí, která pevně drží na táhlu.
Rozhodnutí zda se bude táhlo zasouvat nebo vysouvat je dáno polaritou napětí dodávaného elektromotoru, který se na základě toho otáčí buď po nebo proti směru hodinových ručiček.

%todo obrazek dvirek
konstrukce dveri - expoziční
- přímočarý motor s dorazy
- konstrukce z hliníkových profilů a smontnotváno pomocí běžných spojovacích prvků pro práci s těmito profily
- dvířka reprezentována hliníkovým plátem tahaným lankem připěvněným k ramenu který vynáší sílu od motoru k dvířkům
- zdroj 12V
konstrukce dveri - instalováno
- zdroj 12V
- rám z železných U profilů použitých jako vodící lišty pro posuv dvířek z plastové desky a pásků které konstrukci udžují po kupě
- naviják je s dvířky přopojen lankem
- navijecí systém / systém navijáku
- 12V motor s převodovkou, rychlostí otáčení 60rpm
- navíjecí buben vytištěný na 3d tiskárně
- dorazový systém

\subsection{---poznámky TODO ke smazání---}\label{subsec:---poznamky-todo-ke-smazani---}
- budeme potřebovat váhu pro kontrolu hnízd zda tam slepice je a nebo kolik je tam vajec\newline
- budeme potřebovat ideálně ip kamery s vhodným IP krytím abychom mohly monitorovat kurník vevnitř a venku\newline
- budeme potřebovat ovladačku asi arduino pro dveře, světlo, senzory teploty a vlhkosti\newline
- předchozí věci je potřeba dostat na síť takže nějaké rpi pro předávání komunikace a směrování\newline
- a všechno to musí řídit něco s dostatečným výkonem pro klasifikace třeba my tu máme nucka s RTX2080\newline
- uživatelské rozhraní se rozhodlo že je zbytečné vyrábět vlastní a ztrácet tím čas lepší bude použít HA který disponuje všemi funkcemi je open source a má komunitu která ho udržuje
- první řešení ale bude na stole takže se nemusíme zaobírat detaily kontkrétní instalace, alespoň pro zatím
- implementace probíhala v jazyce python za využítí vypsaných knihoven viz readmečka modulů\newline
- jak se konfiguroval logger, flask blueprinty, jak propojit python a arduino, jak na to s cronem/schedulerem, jak posílat mqtt a přijímat(jaká je struktura našich topiců, jak se to pojmenovává)
- verzuje se to na github
- na githubu běží workflow které vytváří jednolivé docker image pro každý modul a uploaduje je na můj docker hub kvůli snadnému deployi a distribuci po internetu

\section{První pracovní zapojení a běh}\label{sec:prvni-pracovni-zapojeni-a-beh}
- zjistilo se že bez poe je to špatná volba \newline
- kamery žerou dost proudu\newline
- bylo potřeba odladit nastavení kamer a jejich statické ip adresy, kamery měli zabezpečení na blokování ip address a tak\newline
- byl problém s kontakty na váze takže se nakonec museli vyměnit lisované za pájené\newline
- na stole to většinou funguje dobře \newline


\section{Nasazení do kurníku}\label{sec:nasazeni-do-kurniku}
- bylo potřeba natahat elektřinu a internet \newline
- elektřina nebyla problém vzala se z kůlny zkrz díru ve zdi\newline
- horší to bylo s netem protože wifi signál do kurníku nedosáhne\newline
- vyřešilo se to wifi extenderem od tplinku který má zárověň i ethernet výstup díky čemuž nemuselisme dlouhého tahati kabelu a šlo nám to vzduhem přes dvůr a ve stodole pak drátem\newline
- kamery bylo potřeba napájet a taky připojit přes internet; zprvu jsem si myslel že poe nebude potřeba ale taha 230 by bylo zbytečné námahy takže jsem pořídil poe switch od tplinku a ten připojuje kamery\newline
- bylo někde potřeba udělat místo kam se nainstaluje celá technologie kurníku\newline
- zvolila se plechová bedna ze starého domovního rozvaděče do níž se pro jednotlivé prvky vytvožili na míru držáčky a pouzdra ps.: fotky z tisku a pak z rozvaděče\newline
- jak se zrealizovala váha\newline
- jak vypadá řídící jednotka pri room assistanta\newline
- jaké tam jsou dveře pro slepice\newline
- celé je to propojené s rpi 5 které to v kurníku řídí a s ním pak komunikuje nuc pomocí tail scailu ale to zas v další kapitole\newline
- bylo potřeba zkalibrovat hmotnosti slepic a vajec


\section{Konfigurace vzdáleného přístupu}\label{sec:konfigurace-vzdaleneho-pristupu}
- rpi propojene s nuckem a dev kompama přes tailscail\newline
- v docker compose na nuckoj se vyrobila nová network jenom pro home assistanta a jeho propagaci na venek\newline
- do vzniklé sítě se přidal ještě kontejner Cloudflared který zajišťuje tunel ven na cloudflare a přes jeho firewall a proxyny do internetu\newline
- cloudflare tunel bylo potřeba namapovat na doménu kterou vlastníme\newline
- pronajmul jsem si doménu u doméhového registrátora forpsi\newline
- zmínění jaká plynou nebezpečí z tohoto řešení


\section{Rozvržení práce do budoucna}\label{sec:rozvrzeni-prace-do-budoucna}
- jak by se řešení dalo zoptimalizovat\newline
- vylepšení modelu pro classifykaci\newline
- vylepšení a zpřesnění vah\newline
- kontrola napajedla\newline
- kontrola krmítka
- implementace autonomního chování do Room Assistanta


\section{Ekonomická stránka projektu}\label{sec:ekonomicka-stranka-projektu}

