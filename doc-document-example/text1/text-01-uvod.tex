\chapter{Úvod}\label{ch:uvod}

V dnešní době je populární využívat moderní technologie a umělou inteligenci.
Objevuje se v různých aplikacích, jako například monitoring pohybu
zákazníků v obchodech nebo ve formě textových modelů, čímž je například
ChatGPT.
Také dnes ubývá lidí, kteří by se chtěli zabývat staráním o hospodářská
zvířata, a tudíž je potřeba, aby něco zaujalo jejich místo.
Zároveň jsem sám člověk, který rád zkoumá nové věci v oblasti
informatiky a velice ho baví automatizování nejrůznějších úloh.
Na základě těchto faktů jsem se rozhodl vytvořit práci popisující
využití moderních technologií při chovu hospodářských zvířat.
Nápad na tuto práci vznikl díky mojí babičce, která chová doma slepice.
Když odjede na dovolenou, stávám se tím já, kdo se o ně musí starat.
Je třeba ráno a večer dojet a slepice zkontrolovat a spočítat.
Tato činnost je časově náročná kvůli cestování a kolikrát i zbytečná,
protože většinou se nic neděje.
A vzhledem k tomu, že jsem povoláním informatik, budu tuto práci věnovat
 tomu, jak jsem vyřešil automatizaci babiččina chovu kura domácího
pomocí vlastního systému Coopmaster.

\newline
Práce se dělí na teoretickou a praktickou část.
V teoretické části jsou popsány základní pojmy a principy, jež jsou důležité pro plné chápání práce.
Praktická část popisuje konkrétní implementaci a nasazení asistenčního systému pro chov hospodářských zvířat.
Čtenář může využít znalosti, které nasbírá během čtení teoretické části a jež mu následně pomůže chápat praktická část, jako návod k realizaci vlastního systému dle jeho potřeb.
\newline

Svou konkrétní implementaci jsem zaměřil na chov kura domácího z
 výše popsaných osobních důvodů a protože to bylo pro mě nejdostupnější
testovací zvíře. Řešení jsem realizoval pomocí mikroservisní
architektury a jednotlivé služby jsou psány v populárním jazyce Python.
Jako grafické uživatelské rozhraní jsem použil open source software pro
chytrou domácnost Home Assistant.
