\chapter{Úvod}\label{ch:uvod}

V dnešní době je populární využívat moderní technologie se zapojením umělé inteligence i v oblastech, kde by to člověk nečekal.
Vyskytuje se všude kolem nás, objevuje se v různých aplikacích běžného života.
Od rozpoznávání registračních značek automobílů, monitoringu pohybu zákazníků v obchodech nebo ve formě textových modelů, které fungují jako různí pomocníci nebo chatboti.
Zároveň dnes ubývá lidí, kteří by chtěli trávit svůj čas chovem hospodářských zvířat, a proto je potřeba, aby něco zaujalo jejich místo.
Nejsou, ale všichni takoví.
Třeba já jsem člověk, kterému na zvířátkách záleží a chce, aby se měli dobře.
Zároveň jsem i programátor, který rád zkoumá nové věci v oblasti informatiky a velice ho baví automatizování nejrůznějších úloh. \par
Všechna tato fakta mi pomohla rozhodnout jaký projekt chci realizovat.
Pomohla vytvořit práci, popisující využití moderních technologií při chovu hospodářských zvířat.
Nápad na tuto práci vznikl díky mojí babičce, která chová doma slepice.
Když odjede na dovolenou, stávám se já tím, kdo se o ně musí starat.
Je třeba ráno i večer dojet slepice zkontrolovat a spočítat.
Tato činnost je časově náročná kvůli cestování a kolikrát i zbytečná, protože většinou se nic neděje.
A vzhledem k tomu, že jsem povoláním informatik, tuto práci věnuji tomu, jak jsem vyřešil automatizaci babiččina chovu kura domácího pomocí vlastního systému Coopmaster.\newline
Moje práce se dělí na teoretickou a praktickou část.
V teoretické části jsou popsány základní pojmy a principy, jež jsou důležité pro plné chápání práce.
V praktické části popisuji konkrétní implementaci a nasazení svého asistenčního systému pro chov hospodářských zvířat.
Tato část zároveň může sloužit k motivaci a inspiraci čtenáře k tvorbě vlastního řešení.
Čtenář využije znalosti, které nasbírá během čtení teoretické části, aby mu následně pomohly chápat praktickou část.\newline
Řešení jsem realizoval pomocí mikroservisní architektury a jednotlivé služby jsou psány v jazyce Python.
Jako grafické uživatelské rozhraní jsem použil open source software pro chytrou domácnost Home Assistant.
