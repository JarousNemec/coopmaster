\chapter{Úvod}\label{ch:uvod}

dnešní době je populární využívat moderní technologie se zapojením umělé inteligence i v oblastech, kde by to člověk nečekal.
Vyskytuje se všude kolem nás, objevuje se v různých aplikacích běžného života.
Od rozpoznávání RZ automobílů, monitoringu pohybu zákazníků v obchodech nebo ve formě textových modelů, které fungují jako různí pomocníci nebo chatboti.
Dnes ubývá lidí, kteří by se chtěli trávit svůj čas chovem hospodářských zvířat, a proto je potřeba, aby něco zaujalo jejich místo.
Nejsou, ale všichni takoví.
Třebá já jsem člověk, kterému na zvířátkách záleží a chce, aby se měli dobře.
Zároveň jsem i programátor, který rád zkoumá nové věci v oblasti informatiky a velice ho baví automatizování nejrůznějších úloh.

Včechna tato fakta se mi pomohla rozhodnout jaký projekt chci realizovat.
Pomohla vytvořit práci popisující využití moderních technologií při chovu hospodářských zvířat.

Nápad na tuto práci vznikl díky mojí babičce, která chová doma slepice.
Když odjede na dovolenou, stávám se já tím, kdo se o ně musí starat.
Je třeba ráno a večer dojet a slepice zkontrolovat a spočítat.
Tato činnost je časově náročná kvůli cestování a kolikrát i zbytečná, protože většinou se nic neděje.
A vzhledem k tomu, že jsem povoláním informatik, budu tuto práci věnovat tomu, jak jsem vyřešil automatizaci babiččina chovu kura domácího pomocí vlastního systému Coopmaster.\newline

Práce se dělí na teoretickou a praktickou část.
V teoretické části jsou popsány základní pojmy a principy, jež jsou důležité pro plné chápání práce.
Praktická část popisuje konkrétní implementaci a nasazení asistenčního systému pro chov hospodářských zvířat.
Čtenář může využít znalosti, které nasbírá během čtení teoretické části a jež mu následně pomůže chápat praktická část, jako návod k realizaci vlastního systému dle jeho potřeb.\newline

Svou konkrétní implementaci jsem zaměřil na chov kura domácího z výše popsaných osobních důvodů a protože to bylo pro mě nejdostupnější testovací zvíře.
Řešení jsem realizoval pomocí mikroservisní architektury a jednotlivé služby jsou psány v populárním jazyce Python.
Jako grafické uživatelské rozhraní jsem použil open source software pro chytrou domácnost Home Assistant.
