\section{Dog alarm}\label{sec:dog-alarm}
Jednou z klíčových komponent systému Coopmaster je služba pro ochranu slepic před potenciálním nebezpečím ve výběhu, nesoucí název Dog Alarm.\newline
Služba je skrze MQTT protokol propojena se systémem Home Assistant, který data ze služby zpracovává a v případě nutnosti notifikuje chovatele.
Jako zdroj dat je použit kamerový systém ve výběhu, který je reprezentován službou Camera Driver.
Z této služby pak Dog Alarm stahuje pomocí GET požadavků fotografie z kamery ve výběhu.\newline
Pro detekci predátorů je využito modelu Yolo11.
Tento model má již předučeno 80 kategorii, které umí v základní verzi detekovat.
Mezi těmito kategoriemi je také pes.
Model je velmi kvalitně vyučený a společnost Ultralytics k němu nabízí rozsáhlou dokumentaci, což je také jeden z hlavních důvodů mojí volby.
Model jsem se před nasazením rozhodl ještě doučit o další fotografie psa pořízené z venkovní kamery, podobně jako je to v sekci~\ref{sec:klasifikace-objektu-pomoci-strojoveho-uceni}, aby se model více přizpůsobil podmínkám vstupních fotografii.\newline
Jakmile AI model pozitivně identifikuje přítomnost psa nebo jiného predátora, služba Dog Alarm okamžitě vygeneruje výstrahu.
Tato výstraha je odeslána pomocí protokolu MQTT do systému Home Assistant.
Spolu se zprávou je připojen specifický záběr, na kterém byl predátor detekován, což poskytuje přehledný a důkazní materiál pro rychlé rozhodování.
Kromě výstražných zpráv, Dog Alarm pravidelně zasílá aktuální záběry z kamery do Home Assistanta, čímž umožňuje uživatelům mít kontinuální přehled o situaci ve výběhu.\newline
Pro shrnutí je Dog Alarm služba, která za použití model Yolo11 detekuje vetřelce (psa) na fotkách z kamery ve výběhu a informuje pomocí MQTT protokolu chovatele do systému Home Assistant.

\subsection*{Popis algoritmu}




\subsection*{Výhody a přínosy}

Implementace Dog Alarm v systému Coopmaster nabízí několik klíčových výhod:
\begin{itemize}
    \item \textbf{Automatizovaný dohled:} Automatická detekce predátorů výrazně zvyšuje bezpečnost slepic tím, že minimalizuje závislost na lidském dohledu.
    \item \textbf{Rychlá reakce:} Díky okamžitému odesílání upozornění může být rychle aktivována ochranná opatření.
    \item \textbf{Dokumentace incidentů:} Uložení snímků predátorů poskytuje důležitou dokumentaci pro další analýzu a případné právní kroky.
\end{itemize}

%Služba Dog alarm má detekovat nebezpečí ve výběhu a poslat tuto zprávu do Home Assistanta.\newline
%Aktuální záběry jsou pomocí GET requestů stahovány z konkrétní instance služby Camera driver, která je přiřazena ke kameře ve výběhu.
%Analýza probíhá v určitých intervalech za pomocí umělé inteligence, kde je konkrétně použita metoda detekce objektů.
%Jakmile jako výsledek klasifikace vyjde jednoznačně, že v záběru byl spatřen pes nebo jiný predátor, je zpráva poslána pomocí MQTT do Home Assistanta společně s konkrétním záběrem, na němž byl predátor detekován.
%Tato služba zároveň přes MQTT posílá do Home Assistanta aktuální záběr z kamery.
