\section{Dog alarm}\label{sec:dog-alarm}
%Služba Dog alarm má detekovat nebezpečí ve výběhu a poslat tuto zprávu do Home Assistanta.\newline
%Aktuální záběry jsou pomocí GET requestů stahovány z konkrétní instance služby Camera driver, která je přiřazena ke kameře ve výběhu.
%Analýza probíhá v určitých intervalech za pomocí umělé inteligence, kde je konkrétně použita metoda detekce objektů.
%Jakmile jako výsledek klasifikace vyjde jednoznačně, že v záběru byl spatřen pes nebo jiný predátor, je zpráva poslána pomocí MQTT do Home Assistanta společně s konkrétním záběrem, na němž byl predátor detekován.
%Tato služba zároveň přes MQTT posílá do Home Assistanta aktuální záběr z kamery.


Jednou z klíčových komponent systému Coopmaster je integrované řešení pro ochranu slepic před potenciálním nebezpečím ve výběhu, nazvané Dog Alarm.
Tato služba je navržena k automatické detekci přítomnosti psů nebo jiných predátorů pomocí moderních technologií zpracování obrazu.

\subsection*{Funkční princip}

Služba využívá kamerový systém instalovaný ve výběhu, který poskytuje aktuální záběry prostřednictvím GET requestů na Camera driver.
Tento komponent je zodpovědný za komunikaci s fyzickým zařízením kamery a poskytuje pravidelné snímky pro analýzu.

\subsection*{Analýza pomocí umělé inteligence}

Pro zpracování záběrů je implementována pokročilá metoda detekce objektů využívající umělou inteligenci (AI). AI model je trénován na rozpoznávání specifických charakteristik psů a dalších potenciálně nebezpečných zvířat.
Detekce probíhá v definovaných intervalech, což zajišťuje pravidelný monitoring výběhu.

\subsection*{Integrace s Home Assistanta}

Jakmile AI model pozitivně identifikuje přítomnost psa nebo jiného predátora, služba Dog Alarm okamžitě vygeneruje výstrahu.
Tato výstraha je odeslána pomocí protokolu MQTT do systému Home Assistant.
Spolu se zprávou je připojen specifický záběr, na kterém byl predátor detekován, což poskytuje přehledný a důkazní materiál pro rychlé rozhodování.

Kromě výstražných zpráv, Dog Alarm pravidelně zasílá aktuální záběry z kamery do Home Assistanta, čímž umožňuje uživatelům mít kontinuální přehled o situaci ve výběhu.

\subsection*{Výhody a přínosy}

Implementace Dog Alarm v systému Coopmaster nabízí několik klíčových výhod:
\begin{itemize}
    \item \textbf{Automatizovaný dohled:} Automatická detekce predátorů výrazně zvyšuje bezpečnost slepic tím, že minimalizuje závislost na lidském dohledu.
    \item \textbf{Rychlá reakce:} Díky okamžitému odesílání upozornění může být rychle aktivována ochranná opatření.
    \item \textbf{Dokumentace incidentů:} Uložení snímků predátorů poskytuje důležitou dokumentaci pro další analýzu a případné právní kroky.
\end{itemize}

V rámci shrnutí je služba Dog Alarm nepostradatelným prvkem v rozsáhlém ekosystému Coopmaster, který podporuje snahu o vytvoření plně automatizovaného, bezpečného a efektivního prostředí pro chov slepic.
