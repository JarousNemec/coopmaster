Testování systému Coopmaster je klíčovým krokem k zajištění jeho spolehlivosti, správnosti a bezpečnosti.
Jako programátor bych se zaměřil na několik úrovní testování, které by zajistily, že všechny komponenty systému fungují podle očekávání.

\section{Jednotkové testy}\label{sec:jednotkove-testy}

Začal bych s jednotkovými testy pro jednotlivé mikroslužby, abych ověřil, že každá část funguje izolovaně správně.
Například bych vytvořil testy pro kamerový ovladač, které by simulovaly přijetí obrázků z kamer a kontrolovaly správné vrácení obrazu.
Podobně bych testoval ovladač váhy pro správné čtení a zpracování dat.
Frameworky pro testování jako \texttt{unittest} nebo \texttt{pytest} v Pythonu by byly ideální pro tuto úroveň.

\section{Integrační testy}\label{sec:integracni-testy}

Poté bych přešel na integrační testy, jejichž cílem je ověřit, zda jednotlivé mikroslužby spolupracují podle očekávání.
Například bych testoval komunikaci mezi Room Assistantem a Home Assistantem přes MQTT, a zajistil, že data o teplotě a vlhkosti jsou správně přenášena.
Integrační testy by měly zahrnovat i scénáře zahrnující hardwarovou interakci, abych se ujistil, že fyzické prvky systému správně komunikují se softwarem.

\section{Systémové testy}\label{sec:systemove-testy}

Systematické testování celého systému by zahrnovalo simulaci reálných uživatelských scénářů, jako je detekce vetřelce pomocí kamery a následné zaslání notifikace.
Zde bych se soustředil na testování, zda celý systém od snímání přes analýzu až po odpověď pro uživatele funguje bez chyby.
Testovací scénáře by také zahrnovaly abnormální situace, jako je výpadek některé z mikroslužeb, a sledování, jak systém reaguje.

\section{Testy výkonu a zátěžové testování}\label{sec:testy-vykonu-a-zatezove-testovani}

Systém by měl být také testován pro výkon při různých zátěžových podmínkách, abych zjistil jeho limitace.
To by zahrnovalo generování velkého množství dat například z kamer a sledování, jak si systém poradí s jejich zpracováním a zda připojení mezi mikroslužbami (přes Docker a MQTT) drží krok.

\section{Testování bezpečnosti}\label{sec:testovani-bezpecnosti}

Jelikož je systém přístupný z internetu, další kritickou oblastí by bylo testování bezpečnosti.
Použití nástrojů pro penetrační testy ke zjištění potenciálních zranitelností, simulace útoků jako jsou DDoS, a kontrola správného nastavení komunikace přes tunely a šifrování pomocí nástrojů jako \texttt{Wireshark} pro monitorování datového provozu.

\section{Uživatelské testování}\label{sec:uzivatelske-testovani}

Nakonec bych se obrátil na testy z pohledu uživatele.
To zahrnuje uživatelskou zkušenost s Home Assistantem - např. přehlednost vizualizací, snadnost ovládání systému z mobilní aplikace.
Zde by bylo důležité získat zpětnou vazbu od skutečných uživatelů systému.

Tím, že bych tuto metodiku testování aplikoval důkladně, bych mohl s jistotou říct, že systém Coopmaster je připraven na nasazení do produkce a splňuje všechny očekávané požadavky na funkcionalitu a spolehlivost.