Testování je klíčovým krokem k zajištění spolehlivosti, správnosti a bezpečnosti aplikací a systémů.
Jako programátor jsem se zaměřil na několik úrovní testování, které mi zajistily, že všechny komponenty systému fungují podle očekávání.
Na svůj projekt jsem ručně aplikoval integrační testy, systémové, zátěžové a na závěr uživatelské testy.
Unit testy, které jsou jinak základem každého správně vyvíjeného softwaru, jsem aplikoval hned při vývoji funkcionality.

\section{Integrační testy}\label{sec:integracni-testy}
Ručně jsem prováděl integrační testy.
Procházel jsem je vždy po dokončení služby, abych zjistil zda všechna komunikační rozhraní fungují správně.
K ověření funkčnosti Flask API rozhraní jsem používal program Postman.
Pro ověření komunikace skrze MQTT jsem použil program MQTT-Explorer.

\section{Systémové a zátěžové testy}\label{sec:systemove-a-smoke-testy}
Dále jsem pouštěl opakovaně pouštěl systémové testy.

Ukazovaly totiž funkčnost celého systému a během času ověřovali funkčnosti v různých situacích.
Díky těmto testům jsem například odhalil špatnou konfiguraci MQTT brokeru, který byl nastaven, aby ukládal všechny zprávy, které skrze něj projdou.

Další chyba kterou testy odhalily byla "Access denied při restartu".
Chyba s restarty byla při nastavování linuxových práv k různým periferiím a souborům.
Problém nastal v případě, kdy bylo potřeba udělit práva pro využití USB portu službě Scale Driver.
Práva se totiž nastavila na konkrétní port, jenže po restartu systému se porty zpřeházely a služba k novému portu neměla přítup.

Hned od začátku vývoje aplikace jsem si byl vědom, že systém poběží neustále.
Proto jsem kladl velký důraz na jeho robustnost.
Od prvních momentů snažil, abych měl aplikaci permanentně nasazenou na testovacím prostředí.


\section{Uživatelské testování}\label{sec:uzivatelske-testovani}
Uživatelské testování jsem prováděl kvůli tomu, abych zjistil, zda je grafické rozhraní v systému Home Assistant dostatečně srozumitelné a přehledné.
Díky těmto testům jsem například opravil nelogičnost v ikonkách tlačítek nebo jsem změnil rozvržení komponent podle toho, v jakém pořadí je chce uživatel na první pohled vidět.