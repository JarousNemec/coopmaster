\section{Room driver}\label{sec:room-driver}
Room driver zařizuje komunikaci mezi službou Room Assistant a řídící jednotkou v kurníku.
Tato jednotka se stará o ovládání dveří, světla a poskytování dat o aktuální teplotě a vlhkosti.
Ovládání nízkoúrovňových komponent a komunikaci s připojeným počítačem zajišťuje Arduino Nano.
Arduino je připojeno pomocí usb sběrnice k počítači, který na základě domluvených pravidel jednoduchého komunikačního protokolu posílá znaky, na než Arduino příslušně reaguje dle programu.
Tato služba má tedy za úkol přes serialový port pomocí USB rozhraní posílat příkazy a načítat stavy řídící jednotky na základě requestů přijmutých na REST api.

\subsection*{Popis algoritmu}
Po startu služby se nejdříve vytvoří instance třídy ArduinoReader.
Jedná se o vlastní třídu, při inicializaci vytvoří instanci serialového spojení k arduinu a to drží po celou dobu běhu aplikace.
Třída má jedinou metodu a tou je run\_command.
Tato metoda přijímá jako parametr jeden znak, který při provolání metody pošle zkrze serialové připojení do arduina a vrátí data typu String, kterými odpoví arduino, zpět.
Jakmile se povede provést inicializaci serialového spojení je spuštěno REST api.
Toto api poskytuje endpointy pro vyslání příkazů k zavření a otevření dvíře, zhasnutí a rozsvícení světla, vracení informací o aktuálních stavech jednotky a vrácení informací o teplotě a vlhkosti.
Informace o aktuálním stavu jsou dobré pro případ, kdy se budou muset služby restartovat a bude třeba načíst reálný aktuální stav.
V případě, kdy přijde požadavek na konkrétní endpoint, jako první věc se provede odeslání příslušného znaku jako příkaz arduinu.
Následně se počká na jeho odpověď, pokud je požadavek typu POST.
Odpověď je následně převedena do JSON objektů a vrácena v HTTP odpovědi s typem těla odpovědi JSON.

\subsection*{Příkazy ovládacího protokolu}
\begin{itemize}
    \item o = otevřít dvířka
    \item c = zavřít dvířka
    \item l = rozsvítit světlo
    \item d = zhasnout světlo
    \item j = vrátí json s daty o teplotě a vlhkosti
    \item s = vrátí json s daty o stavech jednotlivých ovládaných prvku (dvěře, světlo)
\end{itemize}

\subsection*{Technologie}
\begin{itemize}
    \item Jazyk Python a jeho knihovny
    \item REST api
    \item HTTP protokol
    \item Serialový port
\end{itemize}

\subsection*{Použíté knihovny}
\begin{itemize}
    \item Flask: Lehký webový framework pro rychlý vývoj webových aplikací.
    \item colorama: Manipulace s barvami v textovém výstupu na terminálu.
    \item waitress: Rychlý WSGI server pro produkční nasazení webových aplikací.
    \item pyserial: Komunikace se sériovými zařízeními přes sériovéporty.
    \item python-dotenv: Načítání konfigurace z .env souborů.
\end{itemize}


%# coopmaster-room-driver
%
%Aplikace provazuje core aplikace(konkrétně službu room assistant) s arduinem ovladajicim svetla, dvere a poskytujici informaci o teplote a vzdusne vlhkosti.
%Součástí projektu je i firmware arduina.

%## funkcionalita
%- umoznuje ovladat osvetleni kurniku
%- resi otevírání a zavírání dvířek v kurníku
%- ziskava informace ze senzoru o teplote a vlhkosti v kurniku
%- komunikuje se zbytkem systému pomocí rest api
%- s arduinem komunikuje pomocí serialového portu a pomocí zasílání příkazů v podobě jednoho písmene
%- příkazy: o = otevřít dvířka, c = zavřít dvířka, l = rozsvítit světlo, d = zhasnout světlo, j = vrátí json s daty o teplotě a vlhkosti, s = vrátí json s daty o stavech jednotlivých ovládaných prvku (dvěře, světlo)
%
%## technologie
%- python
%- knihovny pro python
%- **Flask**: Lehký webový framework, flexibilní a rychlý vývoj webových aplikací.
%- **colorama**: Manipulace s barvami v textovém výstupu na terminálu.
%- **waitress**: Rychlý WSGI server pro produkční nasazení webových aplikací.
%- **requests**: Jednoduché HTTP požadavky (GET, POST, atd.).
%- **Werkzeug**: WSGI nástroje pro webové aplikace (routování, správa relací).
%- **python-dotenv**: Načítání konfigurace z `.env` souborů.
%- **pyserial**: Komunikace se sériovými zařízeními přes sériové porty.
