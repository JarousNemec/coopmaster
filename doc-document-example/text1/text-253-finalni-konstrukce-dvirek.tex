\section{Vylepšení konstrukce dvířek}\label{sec:vylepseni-konstrukce-dvirek}
Vylepšení konstrukce dvířek a realizace finálního řešení je jeden z plánů do budoucna pro tento projekt.
Jak již bylo zmíněno v popisu současné konstrukce v sekci~\ref{sec:konstrukce-dvirek}, aktuální konstrukce je příliš drahá na výrobu a také zbytečně robustní.
Pro finální instalaci v následujících větách popíšu návrh na realizaci produkční konstrukce.\newline
Konstrukce by měla být levnější na výrobu i na materiál a zároveň stejně bezpečná a výkonná jako ta stávající prototypová.
Nový rám bude tvořen z železných U profilů použitých zároveň jako vodící lišty pro posuv dvířek.
Posuvný plát bude z plastové desky.

\subsection*{Navíjecí mechanismus}\label{subsec:navijeci-mechanismus}
Způsob ovládání neboli napájení se nemění takže není třeba dělat žádné úpravy na řídící jednotce popsané detailněji v sekci~\ref{sec:ridici-jednotka}.
Pohyb bude dvířkům dodávat malý elektrický naviják vlastní konstrukce.
Díky vlastnímu návrhu bude možno celý rám navijáku navrhnout pomocí CAD programu a následně ho celý vyrobit na 3D tiskárně.
Tento způsob výroby šetří čas a je zajištěna přiměřeně stejná kvalita u všech součástek.
Pohonnou jednotkou navijáku bude 12 V elektromotor model JGA25-370 na stejnosměrné napětí s vestavěnou převodovkou.
Díky převodovce dokáže motor vyvinout kroutící moment až 130N na 1cm dlouhé páce, což je na pohyb dvířek dostačující.
Vývodový hřídel z převodovky elektromotoru bude připevněn k navíjecímu bubnu o průměru okolo 15 mm.
Na navíjecím bubnu bude připevněno i lanko, které se bude používat na vytahování dvířek nahoru.
Pohyb dolů je zajištěn díky gravitaci, díky čemuž stačí povolit lanko a dvířka sama sjedou dolů.
Jako pokračování v ose otáčení motoru a bubnu bude k bubnu připevněn malý šnekový hřídel s maticí.
Matici je zamezeno otáčení opřením o rám navijáku, takže se při otáčení hřídele se bude matice posouvat.
Tohoto pohybu bude využívat poslední součást navijáku a to systém dorazů, aby bylo možno nastavit maximální počet otáček bubnu a nedošlo tak k úplnému odvinutí lanka.
Úplné odvinutí lanka by v případě, že by se motor pořád točil, znamenalo, že se lanko zase začne navíjet opačným směrem, což obrátí navíjecí logiku navijáku.
Řídící jednotka totiž počítá pouze s jedním směrem otáčení a nepočítá s náhlým přehozením směru otáček motoru pro navíjení a odvíjení.


%konstrukce dveří - instalováno
%- zdroj 12V
%- rám z
%- naviják je s dvířky přopojen lankem
%- navíjecí systém / systém navijáku
%- 12V motor s převodovkou, rychlostí otáčení 60rpm
%- navíjecí buben vytištěný na 3d tiskárně
%- dorazový systém