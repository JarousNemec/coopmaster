Moje dosavadní práce na školních projektech a zkušenosti s vývojem aplikaci mě naučili, že před tím než se započne tvorba jakékoli aplikace, je třeba udělat důkladnou analýzu.

Je důležité mít představu, co má aplikace umět, jaké jsou na ni kladeny požadavky a také je dobré vědět, kdo bude uživatel.

Uživatel lépe řečeno dobře nadefinovaná uživatelská role nám pomáhá pochopit jaké má cíle a motivace.
Ať už je to konkrétní člověk nebo fiktivní postava, vcítění se do jejího pohledu na věc je cenným zdrojem zpětné vazby.

Mým zákazníkem a uživatelem je moje babička.
Je to člověk milující zvířátka, nepřítel stolního počítače, ale již několik let online na mobilním telefonu.
Můžeme tedy říci, že v našem případě, děláme systém pro člověka znalého chovu hospodářských zvířat, ale méně zdatného v používání informačních technologií.

Teď se pojďme zamyslet jakou funkcionalitu je třeba do řešení implementovat.
Postupem let, kdy babičce se slepičkami pomáhám, se mi v hlavě začal utvářet seznam.
Její největší starost krom krmení je otevírání a zavírání dvířek, svícení v kurníku, sběr vajíček a starost o kuny, které chodí na vejce.
Dále má pak strach ze psů, kteří se jí tam v minulosti dostali.
Řeší počítání slepiček, aby zjistila jestli jsou všechny doma.
Má problém i se slepicemi, které zanáší kdekoliv ve výběhu.
A v neposlední řadě mívá i takový pocit, jako když člověk odejde z domu a neví jestli zamkl.
V jejím případě by ráda věděla, jestli je všechno v pořádku.

V jedné větě bych mohl říct, že je třeba, aby aplikace umožňovala uživateli vzdálený dohled nad jeho chovem a pomáhala mu šetřit čas zautomatizováním každodenních činností.

Vytvořil jsem si podle zásad agilního vývoje~\cite{agile-manifesto} prioritizovaný seznam požadavků.
Tento seznam jsem setřídil podle priorit, náročnosti a také svých možností a schopností.
Na vrchol tohoto seznamu se dostaly úkoly, které pokrývají následují oblasti a poskytnou uživateli následující funkce

\begin{itemize}
    \item Vzdálený přístup
    \item Pohledy z bezpečnostních kamer
    \item Automatizované ovládání světla a dvířek v kurníku
    \item Intuitivní vizualizaci stavů jednotlivých hnízd (sedící slepice, snesené vajíčko)
    \item Upozornění na detekování vetřelce ve výběhu
    \item Data o teplotě a vlhkosti
\end{itemize}


\newpage