Před tím než se započne tvorba jakékoli aplikace, je třeba udělat důkladnou analýzu.
Je třeba určit, pro koho danou aplikaci tvoříme.
Následovat by měl rozhovor z daným člověkem, abychom pochopili jeho možnosti, schopnosti, požadavky.

Našim cílem by mělo být navrhnout aplikaci tak, aby ji uživatel chtěl používat, řešila to, co potřebuje a aby pomáhala.

Mým zákazníkem a uživatelem je moje babička.
Je to člověk milující zvířátka, nepřítel stolního počítače, ale již několik let online na mobilním telefonu.
Můžeme tedy říci, že v našem případě, děláme systém člověka znalého chovu hospodářských zvířat, ale méně zdatného v používání informačních technologií, ale schopného akceptovat mobilní aplikací.

Ted si pojme zamýslet jakou funkcionalitu je třeba do řešení implementovat.
Postupem let, kdy babičce se slepičkama pomáhám se mi v hlavě začal utvářet seznam.
Jeji největší starost krom krmení je otevírání a zavírání dvířek, svícení v kurníku, sběr vajíček, starost o kuny, které chodí na vejce.
Dále má pak strach ze psů, kteří se ji tam v minulosti dostali, řeší počítání slepiček jestli jsou doma, má i problém se slepicemi, které zanáší kdekoliv ve výbehu.
A v neposlední řadě takový ten pocit, jako když odejde člověk z domu a neví jestli zamknul.
V jejím případě by ráda věděla, jestli je všechno v pořádku.

V jedné větě bych mohl říct, že je třeba, aby aplikace umožňovala uživateli vzdálený dohled na jeho chov a pomáhala mu šetřit čas zautomatizováním každodenních činností.

Vytvořil jsem si podle zásad agilního vývoje  \cite{agile-manifesto} prioritizovaný backlog požadavků.
Tento seznam jsem setřídil podle priorit a svých možností.

Na vrchol tohoto seznamu se dostaly úkoly, které pokrývají následují oblastí a poskytnou uživateli následující funkce

\begin{itemize}
    \item Vzdálený přístup
    \item Pohledy z bezpečnostních kamer
    \item Dálkové ovládání a automatizaci světla a dvířek v kurníku
    \item Intuitivní vizualizaci stavů jednotlivých hnízd(sedící slepice, v opačném případě počet vajec)
    \item Upozornění na detekování vetřelce ve výběhu
    \item Data o teplotě a vlhkosti
\end{itemize}


\newpage