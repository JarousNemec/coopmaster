Před tím než se započne tvorba jakékoli aplikace, je třeba určit, pro koho danou aplikaci tvoříme a jak bychom chtěli, aby jí tento uživatel používal.
V našem případě jde o to, že systém by měl být schopný používat člověk znalý v chovu hospodářských zvířat, ale často méně zdatný v používání informačních technologií a mobilních aplikací.
Na základě toho, že už víme, kdo je cílový uživatel můžeme se zamýšlet jakou funkcionalitu do řešení implementovat.
Jedná se tedy o to, aby aplikace umožňovala uživateli vzdálený dohled na jeho chov a pomáhala mu šetřit čas zautomatizováním každodenních činnosti.

Rozhodlo se tedy, že aplikace bude uživateli poskytovat následující funkce
\begin{itemize}
    \item Vzdálený přístup
    \item Pohledy z bezpečnostních kamer
    \item Dálkové ovládání a automatizace světla a dvířek v kurníku
    \item Intuitivní vizualizace stavů jednotlivých hnízd(sedící slepice, v opačném případě počet vajec)
    \item Upozornění na detekování vetřelce ve výběhu
    \item Poskytování dat o teplotě a vlhkosti
\end{itemize}


\newpage