\section{První pracovní zapojení a běh}\label{sec:prvni-pracovni-zapojeni-a-beh}
- zjistilo se že bez poe je to špatná volba \newline
- kamery žerou dost proudu\newline
- bylo potřeba odladit nastavení kamer a jejich statické ip adresy, kamery měli zabezpečení na blokování ip address a tak\newline
- byl problém s kontakty na váze takže se nakonec museli vyměnit lisované za pájené\newline
- na stole to většinou funguje dobře \newline

\newpage


\section{Nasazení do kurníku}\label{sec:nasazeni-do-kurniku}
- bylo potřeba natahat elektřinu a internet \newline
- elektřina nebyla problém vzala se z kůlny zkrz díru ve zdi\newline
- horší to bylo s netem protože wifi signál do kurníku nedosáhne\newline
- vyřešilo se to wifi extenderem od tplinku který má zárověň i ethernet výstup díky čemuž nemuselisme dlouhého tahati kabelu a šlo nám to vzduhem přes dvůr a ve stodole pak drátem\newline
- kamery bylo potřeba napájet a taky připojit přes internet; zprvu jsem si myslel že poe nebude potřeba ale taha 230 by bylo zbytečné námahy takže jsem pořídil poe switch od tplinku a ten připojuje kamery\newline
- bylo někde potřeba udělat místo kam se nainstaluje celá technologie kurníku\newline
- zvolila se plechová bedna ze starého domovního rozvaděče do níž se pro jednotlivé prvky vytvožili na míru držáčky a pouzdra ps.: fotky z tisku a pak z rozvaděče\newline
- jak se zrealizovala váha\newline
- jak vypadá řídící jednotka pri room assistanta\newline
- jaké tam jsou dveře pro slepice\newline
- celé je to propojené s rpi 5 které to v kurníku řídí a s ním pak komunikuje nuc pomocí tail scailu ale to zas v další kapitole\newline
- bylo potřeba zkalibrovat hmotnosti slepic a vajec

\newpage


\section{Konfigurace vzdáleného přístupu}\label{sec:konfigurace-vzdaleneho-pristupu}
- rpi propojene s nuckem a dev kompama přes tailscail\newline
- v docker compose na nuckoj se vyrobila nová network jenom pro home assistanta a jeho propagaci na venek\newline
- do vzniklé sítě se přidal ještě kontejner Cloudflared který zajišťuje tunel ven na cloudflare a přes jeho firewall a proxyny do internetu\newline
- cloudflare tunel bylo potřeba namapovat na doménu kterou vlastníme\newline
- pronajmul jsem si doménu u doméhového registrátora forpsi\newline
- zmínění jaká plynou nebezpečí z tohoto řešení

\newpage

\section*{---poznámky TODO ke smazání---}\label{sec:---poznamky-todo-ke-smazani---}
- budeme potřebovat váhu pro kontrolu hnízd zda tam slepice je a nebo kolik je tam vajec\newline
- budeme potřebovat ideálně ip kamery s vhodným IP krytím abychom mohly monitorovat kurník vevnitř a venku\newline
- budeme potřebovat ovladačku asi arduino pro dveře, světlo, senzory teploty a vlhkosti\newline
- předchozí věci je potřeba dostat na síť takže nějaké rpi pro předávání komunikace a směrování\newline
- a všechno to musí řídit něco s dostatečným výkonem pro klasifikace třeba my tu máme nucka s RTX2080\newline
- uživatelské rozhraní se rozhodlo že je zbytečné vyrábět vlastní a ztrácet tím čas lepší bude použít HA který disponuje všemi funkcemi je open source a má komunitu která ho udržuje
- první řešení ale bude na stole takže se nemusíme zaobírat detaily kontkrétní instalace, alespoň pro zatím
- implementace probíhala v jazyce python za využítí vypsaných knihoven viz readmečka modulů\newline
- jak se konfiguroval logger, flask blueprinty, jak propojit python a arduino, jak na to s cronem/schedulerem, jak posílat mqtt a přijímat(jaká je struktura našich topiců, jak se to pojmenovává)
- verzuje se to na github
- na githubu běží workflow které vytváří jednolivé docker image pro každý modul a uploaduje je na můj docker hub kvůli snadnému deployi a distribuci po internetu