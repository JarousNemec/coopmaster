\lstset{
  language=C++,
  basicstyle=\ttfamily\small,
  keywordstyle=\color{blue},
  stringstyle=\color{red},
  commentstyle=\color{green},
}


V tomto dokumentu podrobně popíšeme postup, jak vytvořit jednoduchý systém pro měření hmotnosti pomocí tenzometrického snímače a modulu HX711 připojeného k Arduinu.
Tento projekt bude používat knihovnu HX711 pro usnadnění komunikace a přesné měření.
Pro dosažení našich cílů budeme následovat několik fází, včetně přípravy hardwaru a softwaru, napsání kódu a jeho nahrání do Arduina.

\section*{Import a Definice Pinů}
Nejprve se zaměřme na import potřebných knihoven a definici pinů.
Knihovna HX711.h je klíčová pro komunikaci s modulem HX711, zatímco Arduino.h je základní knihovnou pro všechny Arduino projekty.
Kromě toho definujeme piny, které budeme používat pro připojení k modulu HX711.
Použijeme makra \#define pro stanovení pinů 3 a 2, ke kterým připojíme DOUT a CLK vývody modulu.

\begin{lstlisting}
#include "Arduino.h"
#include "HX711.h"

#define DOUT 3
#define CLK 2
\end{lstlisting}

\section*{Inicializace}
Následně vytváříme instanci třídy HX711 nazvanou scale, kterou využijeme pro všechny operace spojené s měřením.
Stanovujeme počáteční kalibrační faktor na -100.935, který bude pravděpodobně potřebovat úpravu během procesu kalibrace pro dosažení přesnosti.
Tato úprava se provádí manuálně porovnáním s referenční hmotností.
K uložení aktuální naměřené hmotnosti použijeme proměnnou current\_weight.

\begin{lstlisting}
HX711 scale;
float calibration_factor = -100.935;
float current_weight = 0;
\end{lstlisting}

\section*{Nastavení Systému}
Jakmile máme vše připravené, můžeme se pustit do funkce setup().
Prvním krokem je inicializace sériové komunikace na 9600 baudů, což nám umožní komunikovat s počítačem přes Serial Monitor v Arduino IDE. Poté použijeme begin() pro inicializaci modulu HX711 a set\_scale() k nastavení kalibračního faktoru.
Funkce tare() slouží k vynulování váhy, takže při spuštění programu předpokládáme, že na váhách není žádný objekt.

\begin{lstlisting}
void setup() {
  Serial.begin(9600);
  scale.begin(DOUT, CLK);

  scale.set_scale(calibration_factor);

  scale.tare();
}
\end{lstlisting}

\section*{Hlavní Program}
Ve smyčce loop(), která se opakuje nepřetržitě, čteme z bufferu pomocí funkce Serial.read(), což nám dává jeden byte z bufferu.
Funkce scale.get\_units(1) nám poskytuje aktuální hmotnost.
Pokud obdržíme znak w, vytiskneme aktuální váhu na Serial Monitor s přesností na čtyři desetinná místa.
Při obdržení znaku t provádíme tare() pro opětovné vynulování váhy.

\begin{lstlisting}
void loop() {
  char c = Serial.read();

  current_weight = scale.get_units(1);

  if (c == 'w') {
    Serial.println(current_weight, 4);
  }
  else if (c == 't') {
    scale.tare();
  }
}
\end{lstlisting}

\section*{Nahrávání programu do Arduina}
Nahrání programu do Arduino desky je klíčová část tohoto postupu.
Nejprve se ujistíme, že máme nainstalované Arduino IDE, což je vývojové prostředí umožňující psaní, kompilaci a nahrávání kódu do Arduino zařízení.
Zde je detailní postup:

\begin{itemize}
  \item \textbf{Instalace Arduino IDE}: Navštivte oficiální webovou stránku Arduina a stáhněte nejnovější verzi IDE kompatibilní s vaším operačním systémem.
  Nainstalujte jej podle pokynů.
  \item \textbf{Připojení Arduino k počítači}: Použijte USB kabel k fyzickému připojení Arduino desky k počítači.
  Ujistěte se, že je deska správně rozpoznána.
  \item \textbf{Otevření Arduino IDE a kopírování kódu}: Spusťte Arduino IDE a vložte do něj váš kód.
  \item \textbf{Výběr desky a portu}: V menu Tools zvolte správný typ Arduino desky a správný port, kam je vaše Arduino připojeno.
  \item \textbf{Nahrání programu}: Klikněte na tlačítko Upload.
  Program se nejprve zkompiluje a poté nahraje do Arduino desky.
  Proces je úspěšný, pokud neuvidíte žádné chyby.
  \item \textbf{Použití Serial Monitoru}: Pro interakci s vaším programem použijte Serial Monitor, který najdete v Arduino IDE.
\end{itemize}

\section*{Závěr}
Celý tento projekt a postup práce nabízí efektivní způsob využití Arduina pro měření hmotnosti pomocí modulu HX711.
Spojení teoretických znalostí se skutečnou aplikací ilustruje praktickou využitelnost takového systému v různých oblastech od vědy po průmysl.


