\section{Camera driver}\label{sec:camera-driver}
Camera driver je služba zodpovědná za sprostředkování komunikace mezi IP kamerami a službami zpracovánajícími obraz (Dog alarm, Chicken watch guard).\newline

\subsection*{Popis algoritmu}
Po startu služby naběhne REST api na portu 9001 vytvořené za pomoci python frameworku Flask.
Api poskytuje endpoint pro získání aktuálního snímku z každé kamery, která je službě přiřazena v konfiguraci.
S kamerami služba komunikuje pomocí HTTP protokolu.
Byla zvažována i možnost využití protokolu RTSP.
Nicméně naše řešení prozatím nemá žádne vysoké nároky na živý přenos.
Proto se jako lepší volba ukázala cesta využití REST api kamery, které umožnuje jednou začas si sváthnout aktuální snímek.
Jakmile je přijat požadavek na získání aktuálního snímku z konkrétní kamery, služba za pomocí get requestu stáhne obraz z restového api síťové kamery a vrátí ho ve Full HD rozlišení jako odpověď na http dotaz.
Endpoint kamery ze kterého má služba obrázek stahovat, získá z konfigurace předávané pomocí systémových promenných.
Kamery mají vlastní zabezpečení proti neoprávněnému přístupu k datům.
V případě ověřování HTTP komunikace se používá metoda Digest Access Authentication, pomocí které je kamerě předáváno uživatelské jméno a heslo.


%# coopmaster-camera-driver
%- komunikator mezi IP camerou a obraz zpracovavajici aplikaci (dog alarm, chicken watch guard)
%
%## funkcionalita
%- nese si informaci o konfiguraci jednotlivych IP kamerach aktuálně je zde pevně zahardcoděne používání dvou kamer 1. pro dog alarm a 2. pro chicken watch guard
%- vystavuje REST API pro ziskani aktualniho obrazu z chicken kamery a kamery v ohrade
%- vraci obrazek ve FULL HD rozliseni
%- pro každý call se načte nový obrázek z kamery
%
%## technologie
%- python
%- rtsp - real-time streaming protocol
%- knihovny
%
%
%## hardware
%- IP camera Hilook by Hikvision IPC-B180HA-LU