\externaldocument{text-02-teoreticka}

\subsection{Health checker}\label{subsec:health-checker}
Health checker je malá služba určená pro správce systému.
Poskytuje informace o tom zda všechny potřebné služby běží, aby správce nebyl nucen přihlašovat se vzdáleně na server a manuálně kontrolovat každou službu.
Výpis stavů jednotlivých služeb je poskytován pomocí jednoduché webové stránky.

\subsubsection{Funkcionalita}
Po nastartování služby je spuštěno REST api pro komunikaci se správcem.
Jediným endpointem služby je /status.
Pokud uživatel pomocí prohlížeče pošle GET požadavek na tuto službu, bude mu vrácen seznam služeb a jejich stavů převedený do formátu JSON.
Stavy jsou myšleny status kódy a případné chybové hlášky, které služby po zavolání jejich metody /ping.
Služby jsou v seznamu identifikovány na základě portů.
Každá služby v systému Coopmaster má svůj port a porty jsou po sobě v intervalu od 9000 do 9010.
Tento způsob identifikace a rozdělení portů službám zjednodušuje jejich monitorování.
Algoritmus funguje na základě plánoveče úloh neboli scheduleru, který jednou do minuty spustí kontrolu všech služeb a jejich stavy uloží do seznamu odkud jsou pak načítány při požadavku na api.

%todo: sehnat obrázek jsonu který služba vrací