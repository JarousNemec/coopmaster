\chapter{Teoretická část}
Abychom mohli napsat odborný text jasně a srozumitelně, musíme splnit několik základních předpokladů\cite{vut-zkousky}:
\begin{itemize}
    \item musíme mít co říci,
    \item musíme vědět, komu to chceme říci,
    \item musíme si dokonale promyslet obsah,
    \item musíme psát strukturovaně.
\end{itemize}


\section{Musíme mít co říci}
Nejdůležitějším předpokladem dobrého odborného textu je myšlenka.
Je-li myšlenka dost závažná, tak přetrvá, i když je neobratně a zmateně podaná.
Chceme-li však myšlenku podat co nejvýstižněji a ušetřit tak čtenáři čas, musíme dodržet určité zásady, o~kterých pojednáme dále.

\section{Musíme vědět, komu to chceme říci}
Dalším důležitým předpokladem dobrého psaní je psát pro někoho.
Píšeme-li si poznámky sami pro sebe, píšeme je jinak než výzkumnou zprávu, článek, diplomovou práci, knihu nebo dopis.
Podle předpokládaného čtenáře se rozhodneme pro způsob psaní, rozsah informace a míru detailů.

\section{Musíme si dokonale promyslet obsah}
Jakmile víme, co chceme říci a komu, musíme si rozvrhnout látku.
Ideální je takové rozvržení, které tvoří logicky přesný a psychologicky stravitelný celek, ve kterém je pro všechno místo a jehož jednotlivé části do sebe přesně zapadají.
Jsou jasné všechny souvislosti a je zřejmé, co kam patří.

Abychom tohoto cíle dosáhli, musíme pečlivě organizovat látku.
Rozhodneme, co budou hlavní kapitoly, co podkapitoly a jaké jsou mezi nimi vztahy.
Diagramem takové organizace je graf, který je velmi podobný stromu, ale ne řetězci.
Při organizaci látky je stejně důležitá otázka, co do osnovy zahrnout, jako otázka, co z~ní vypustit.
Příliš mnoho podrobností může čtenáře právě tak odradit jako žádné detaily.

\section{Musíme začít psát strukturovaně}
Máme-li tedy myšlenku, představu o~budoucím čtenáři, cíl a osnovu textu, můžeme začít psát.
Při psaní prvního konceptu se snažíme zaznamenat všechny své myšlenky a názory vztahující se k~jednotlivým kapitolám a podkapitolám.
Každou myšlenku musíme vysvětlit, popsat a prokázat.
Hlavní myšlenku má vždy vyjadřovat hlavní věta a nikoliv věta vedlejší.
