\chapter{Teoretická část}
\section{Virtualizace}
Virtualizace~\autocite{virtualizace-Martin-Polednik} je technologie, která se využívá pro efektivnější rozdělení hardwarových zdrojů fyzického  počítače mezi virtualizované/hostované počítače.
Jednotlivé systémy běží v podstatě na jiném hardwaru, díky čemuž jsou jejich procesy izolovány, což přispívá k bezpečnosti řešení.
Virtuální stroj je zárověň jednodužší spravovat, díky tomu, že máme úplnou kontrolu nad daným virtuálním strojem, což pomáhá například při klonování nebo restartu.

\section{Kontejnerizace}
Kontejnerizace~\autocite{virtualizace-Martin-Polednik} je technologie umožňující zapouzdření funkcionality jedné aplikace společně se všemi jejími závislostmi a zdroji do jednotky nazývané česky obrazy.
Konkrétní běžící instance jednoho obrazu se nazývá kontejner.
Tyto kontejnery pak lze spouštět v prakticky neomezeném množsví nezávisle na platformě a výhodou je, že jsou jejich běhy navzájem izolované, takže běh jednoho neovlivní ostatní.
Je to v podstatě forma virtualizace se všemi jejími výhodami, s tím rozdílem, že daný kontejner obsahuje jen nezbytné věci pro běh dané aplikace.
\subsection{Docker Engine}
Jedna z technologií používaných pro kontejnerizaci je Docker Engine~\autocite{kontejnerizace-docker}.
Používá se pro tvorbu, správu, orchestraci, verzování a nasazování jednotlivých kontejnerů.
\subsection{Docker Compose}
Docker compose~\autocite{kontejnerizace-docker-compose} je nástroj pro Docker, který nám umožňuje tvořit komplikované ekosystémy jednotlivých kontejnerů, jež spolu v rámci něho mohou spolupracovat.
Pomáhá nám například se síťováním nebo konfigurací kontejnerů.
\section{Mikroservisní architektura}
\section{Jazyk Python}
\section{Jazyky Wire a C++ }
\section{Git}
\section{Github workflow}
\section{Mqtt}
\section{Home Assistant}
\section{Arduino}
\section{Ip kamera a RTSP}
\section{Cloudflare tunneling}
\section{Flask}
\section{Strojové učení}
\section{Yolo Ultralytics}
\section{Tenzometrický senzor}
\section{Poe}
\section{Wifi extender}
\section{TailScale vpn}
\section{Raspberry PI}
\section{Raspberry PI OS}
\section{Yaml}
\section{GUI}




