\externaldocument{text-02-teoreticka}

\subsection{Scale driver}\label{subsec:scale-driver}
Scale driver zodpovídá za komunikaci mezi fyzickou váhou a službami, které využívají data o vážení.\newline
Protože jako řídící jednotka váhy je použito Arduino(\ref{sec:arduino}) tento modul komunikuje přes serialový port pomocí protokolu USB a na dotaz přijmutý RESTovým api poskytne jako odpověď hodnotu načtenou z váhy.
Pro služby v systému tento driver poskytuje data o hmotnosti opět pomocí REST api.

jak to funguje
konfigurace
scheduler
otevře se tcp spojení
gobler
Je tam sleep kvůli tomu aby se nečetlo moc času
posíláme příkaz w jako weight soucast programu ve vaze
ctu v gramech aktuální hodnotu váhy
proc používáme gramy

ošetření vstupní hodnoty, cteme pomoci readline nakonci jsou netisknutelne znaky, vysvětlit error handling
vracime -1 kdyz je chyba

zalogujeme výstup pro případnou investigaci problémů v produkci

a vracíme zpět hodnotu v gramech


\subsubsection{Funkcionalita}
v pravidelnych intervalech načíteá / aktualizuje hodnoty
hmotností, které přes seriové porty poskytují jednotlivé váhy v hnizdech
vystavuje REST API pro ziskani aktuálně načtené hodnoty
vraci hodnotu v gramech

\subsubsection{Technologie}
python
c++

\subsubsection{Použíté knihovny}
Flask: Lehký webový framework pro rychlý vývoj webových
aplikací.
colorama: Manipulace s barvami v textovém výstupu na
terminálu.
waitress: Rychlý WSGI server pro produkční nasazení webových
aplikací.
pyserial: Komunikace se sériovými zařízeními přes sériové
porty.
python-dotenv: Načítání konfigurace z .env souborů.
knihovny arduina
defaultní Arduino.h
HX711.h: knihovna pro komunikaci s module AD Převodníku
24-bit 2 kanály HX711

\subsubsection{Hardware}
arduino Nano v3.0
AD převodník HX711
tenzometrický senzor
samotana konstrukce hnizda s vahou (podestylka/koberec)

\subsubsection{Konstrukce váhy}
tenzometricky sezor podobne jako u vcel
hezky popsano v praci
uvaha o tom co vybrat zavesna vaha nebo tlacna vaha
problemy pri implementaci
problémy s konstrukcí a nekvalitními součástkami
senzory se deformují a tím vzniká chyba v měření
chyby vznikají díky analogu i chybami / odpory v elektrickém
vedení (pevné / pájene spoje vs odnímatelné konektrory)

%\begin{itemize}
%    \item GET /api/weight
%\end{itemize}