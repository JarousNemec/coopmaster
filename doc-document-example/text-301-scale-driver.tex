\externaldocument{text-02-teoreticka}

\subsection{Scale driver}\label{subsec:scale-driver}
Scale driver zodpovídá za komunikaci mezi fyzickou váhou a službami, které využívají data o vážení.\newline
Protože jako řídící jednotka váhy je použito Arduino(\ref{sec:arduino}) tento modul komunikuje přes serialový port pomocí protokolu USB a na dotaz přijmutý RESTovým api poskytne jako odpověď hodnotu načtenou z váhy.

\subsubsection{Funkcionalita}
Nejprve naběhne hlavní vlákno, které se stará o poskytování RESTového api na portu 9004.
Následně je spuštěno nové vlákno, které má za úkol číst data z váhy a předávat je aplikačnímu rozhraní jako aktuálně naměřenou hmotnost na váze, jež se bude od teď poskytovat, pokud o ní někdo GET requestem požádá.
Čtení informací z váhy se prování v nekonečném cyklu.
Před začátkem cyklu se zaloguje že je nové vlákno úspěšně spuštěno a následně proběhne otevření komunikace s váhou přes seriový port s využitím knihovny PySerial.
Po úspěšném otervření spojení se váze vždy po uplinutí časového intervalu 2 sekundy, odešle znak w.
Tento znak je ve firmwaru váhy vedeno jako příkaz, při jehož přijetí má váha vrátit aktuální naměřenou hodnotu.
Časové spoždění je implementováno kvůli tomu, že není třeba číst data tak často a zároveň by mohlo docházet k zahlcení váhy.
Jakmile je celá odpověď od váhy přijata programem zpět, jsou data zvalidována a následně, pokud je to možné, převedena na datový typ Int.
Datový typ Int je zvolen z důvodu, že váha posílá data v gramech a je zbytečné v tomto případě používat desetinná čísla, protože gramy poskytují dostatečné rozlišení pro naše účeli.
Pokud se nepodaří převézd přečtenou hodnotu na číslo, je výtupní promněnná nastavena na -1, což zbytku systému značí, že váha není v pořádku.
Na závěr se zalogují aktuálně získaná data a následuje další volání cyklu.

\subsubsection{Technologie}
\begin{itemize}
    \item Jazyk Python a jeho knihovny
    \item REST api
    \item Serialový port
\end{itemize}

\subsubsection{Použíté knihovny}
Flask: Lehký webový framework pro rychlý vývoj webových aplikací.
colorama: Manipulace s barvami v textovém výstupu na terminálu.
waitress: Rychlý WSGI server pro produkční nasazení webových aplikací.
pyserial: Komunikace se sériovými zařízeními přes sériovéporty.
python-dotenv: Načítání konfigurace z .env souborů.